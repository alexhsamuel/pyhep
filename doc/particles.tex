\chapter{Particle properties}

The \module{hep.pdt} module provides code to access a \emph{particle
data table}, which contains measured properties of particles studied in
high energy physics.  

\pyhep includes a particle data table in \code{hep.pdt.default}.  This
table contains particle data from the XXXX edition of the \textit{Review
of Particle Properties} published by the Particle Data Group (PDG).  The
function \function{hep.pdt.loadPdtFile} can be used to load particle
data from a file in the PDG's text file format.

A particle data table is represented by a \class{hep.pdt.Table}
instance, which acts as a dictionary keyed with the plain-text names of
particles.  The text (ASCII) names are set by the PDG.  These names are
generally the same as the conventional particle designations, with Greek
letters spelled out and underscores used to denote subscripts.  For
particle names associated with more than one charge state, the charge
muse be indicated.  Neutral antiparticles conventionally designated with
an overbar are denoted by the ``\code{anti-}'' prefix.  For example, a
positron is spelled ``\code{e+}'', and the short-lived kaon eigenstate
``\code{K_S0}''.  The usual \method{keys} method will return the
plain-text names of all particles in the table.  A particle's name may
have alternate common text spellings; these are stored as aliases, and a
particle may be accessed in the table by any of its aliases.

The values in a particle data table are \class{hep.pdt.Particle}
instances.  Each represents a single species of particle, with unique
quantum numbers.  (Some special particle codes representing bound
states, particles that have not been established, Monte Carlo internal
constructs, etc. are also included.)  

A \class{Particle} object has these attributes.  Central values are
given for measured properties.

\begin{itemize}

  \item \member{name} is the particle's text name.

  \item \member{aliases} is a sequence of alternate text names for the
  particle.

  \item \member{charge_conjugate} is the \class{Particle} object for the
  particle's charge conjugate.

  \item \member{mass} is the particle's nominal mass, in GeV.

  \item \member{width} is the particle's width, in GeV.

  \item \member{charge} is the particle's electric charge.

  \item \member{spin} is the particle's spin, in units of half h-bar.

  \item \member{is_stable} is true if the particle is considered
  stable. 

  \item \member{id} is the particle's Monte Carlo ID number in the PDG's
  numbering scheme.

\end{itemize}

The \class{Table} object also as a method \method{findId}, which returns
the particle corresponding to a Monte Carlo ID number.

The following is a demonstration of using the default particle data
table to look up some particle properties.
\begin{verbatim}
>>> from hep.pdt import default as particle_data
>>> print particle_data["e-"].mass
0.000510999
>>> print particle_data["J/psi"].spin
1.0
>>> print particle_data.findId(22).name
gamma
\end{verbatim}

%-----------------------------------------------------------------------

\chapter{Using \evtgen}

\pyhep provides a simple interface to the \evtgen event generator.
You can easily generate randomized decays of particles, and examine the
decay products.  The \evtgen interface is in the \module{hep.evtgen}
module.

To create a particle decay, follow these steps:
\begin{enumerate}

 \item Create a \class{Generator} instance.  The two arguments to its
 constructor are the path to the particle data listing file, which
 contains particle property information, and the path to the main decay
 file, which contains decays and branching fractions.  The default
 \evtgen particle data and decay files are used if these arguments are
 omitted.  You may specify paths to user decay files, which override the
 main decay file, as additional arguments.  See the \evtgen
 documentation for information about these files.

 \item Create a \class{Particle} object to represent the initial-state
 particle.  Specify the name of the particle, as listed in the particle
 data file, as the argument.  The particle is originally at rest at the
 origin of the lab frame.

 \item Produce the decay by calling the generator's \method{decay}
 method on the particle object.

\end{enumerate}

A \class{Particle} object's momentum is stored in its \member{momentum}
attribute, as a \class{hep.lorentz.FourMomentum} object.  Use
\method{hep.lorentz.lab.coordinatesOf} to obtain its lab-frame
coordinates.  Similarly, its production position is stored in its
\member{position} attribute, as a \class{hep.lorentz.FourVector}.
The name of the Particle's species is in its \member{species} attribute.

Use the \member{decay_products} attribute to access the decay products
of a decayed particle.  That value is a sequence of \class{Particle}
objects representing the particle's decay products.

The following script produces a single decay of an Upsilon(4S), using
a particle data listing file and a decay file in the current directory.
It prints out the decay tree of the Upsilon(4S), with the lab components
of each product's momentum, using the \function{printParticleTree}
function.
\codesample{evtgen1.py}

