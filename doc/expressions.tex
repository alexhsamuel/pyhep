\chapter{Expressions}

We have already seen expressions used as predicates in a table's
\method{select} method, and as formulas for computing the values to
accumulate into histograms.  Generally, operations using expressions
execute much faster than the same logic coded directly as Python, since
expressions are compiled for the specific table into a special format
from which they are evaluated very quickly.

\section{Building and evaluating expressions}

To parse an expression into a Python expression object representing its
parse tree, use \function{hep.expr.asExpression}.
\begin{verbatim}
>>> import hep.expr
>>> ex = hep.expr.asExpression("p_x ** 2 + p_y ** 2")
\end{verbatim}
The expression object's string representation looks similar to the
original formula:
\begin{verbatim}
>>> print ex
(p_x ** 2) + (p_y ** 2)
\end{verbatim}
The expression object's \function{repr} shows its tree structure:
\begin{verbatim}
>>> print repr(ex)
Add(Power(Symbol('p_x', None), Constant(2)), Power(Symbol('p_y', None), Constant(2)))
\end{verbatim}

To evaluate an expression, you must provide the values of all symbols.
For the expression above, the symbols \var{p_x} and \var{p_y} must be
specified.  You may either call the expression object directly, passing
symbol values as keyword arguments:
\begin{verbatim}
>>> ex(p_x=1, p_y=2)
5.0
\end{verbatim}
or you may call its \method{evaluate} method with a map providing values
of the symbols mentioned in the expression:  
\begin{verbatim}
>>> ex.evaluate({"p_x": 1, "p_y": 2})
5.0
\end{verbatim}

\section{Compiling expressions}

Expression objects, as well as expression coded directly in Python,
execute quite a bit slower than similar mathematical expressions
implemented in a compiled language like C or C++.  However, \pyhep can
often execute an expressions much faster, by compiling it to an internal
binary format and then evaluating it with an optimized, stack-based
evaluator implemented in C++.

To compile an expression to this optimized form, use
\function{hep.expr.compile}.  It returns an expression object that can
be used the same way as the original expression, but which executes
faster.
\begin{verbatim}
>>> cex = hep.expr.compile(ex)
>>> cex(p_x=1, p_y=2)
5.0
\end{verbatim}

There is no need to use \code{asExpression} before \code{compile}; just
pass it the expression formula directly.
\begin{verbatim}
>>> cex = hep.expr.compile("p_x ** 2 + p_y ** 2")
>>> cex(p_x=1, p_y=2)
5.0
\end{verbatim}


\subsection{Expression types}

\pyhep attempts to determine the numeric type of the result of an
expression.  To do this, it needs to know the types of the symbol values
in the expression.  For constants, this is obvious, but since Python is
an untyped language, the expression compiler cannot automatically
determine the types of symbolic names in an expression, and treats them
as generic objects.  

\pyhep expressions understand the types \code{int}, \code{float}, and
\code{complex}.  The value \code{None} indicates that the type is not
known, so the value should be treated as a generic Python object.

For example, in these expressions, \pyhep can infer the type of the
expression value entirely from the types of constants.
\begin{verbatim}
>>> print hep.expr.asExpression("10 + 12.5").type
<type 'float'>
>>> print hep.expr.asExpression("3 ** 4").type
<type 'int'>
\end{verbatim}
However, a symbol's type is assumed to be generic, so the whole
expression's type cannot be inferred.
\begin{verbatim}
>>> print hep.expr.asExpression("2 * c + 10").type
None
\end{verbatim}

The expression compiler can do a much better job if it knows the
numerical type of the expression's symbols.  When you call
\function{compile}, you can specify the types of symbols as keyword
arguments.  For example,
\begin{verbatim}
>>> cex = hep.expr.compile("2 * c + 10", c=int)
>>> print cex.type
<type 'int'>
\end{verbatim}
If you provide a second non-keyword argument, this type is used as the
default for all symbols in the expression.
\begin{verbatim}
>>> cex = hep.expr.compile("a**2 + b**2 + c**2", float)
>>> print cex.type
<type 'float'>
\end{verbatim}
By specifying symbol types, you can construct compiled expressions that
execute much faster than expressions with generic types.

You can also construct an uncompiled expression with types specified for
symbols using the \function{hep.expr.setTypes},
\function{hep.expr.setTypesFrom}, and \function{hep.expr.setTypesFixed}
functions.


\section{Using expressions with tables}

Since a table row object is a map from column names to values, you may
specify a row object as the argument to \method{evaluate}.  In the
expression, the name of a column in the table is replaced by the
corresponding value in that row.  Using the \file{tracks.table} table we
created earlier,
\begin{verbatim}
>>> import hep.table
>>> tracks = hep.table.open("tracks.table")
>>> print ex.evaluate(tracks[0])
0.848292267373
\end{verbatim}

This works well for small numbers of rows.  However, if the expression
is to be evaluated on a large number of rows in the same table, it
should be compiled.  Use the table's \method{compile} method, which sets
the symbol types according to the table's schema and performs other
necessary expansions before compiling the expression.  The compiled
expression object behaves just like the original expression object,
except that it runs faster.
\begin{verbatim}
>>> cm = tracks.compile(ex)
>>> print cm.evaluate(tracks[0])
0.848292267373
\end{verbatim}
You may also pass an expression as a string to \method{compile}.
Functions provided by \pyhep which work with expressions, such as a
table's \method{select} method or \function{hep.hist.project} will
compile expressions automatically, where possible.  


\section{Expression syntax}

An expression is specified using Python's ordinary expression syntax,
with the following assumptions:
\begin{itemize}
 \item Arbitrary names may be used in expressions as variable
 quantities, functions, etc.  Other than the built-in names listed
 below, all names must be resolved when the expression is evaluated.

 \item The forward-slash operator for integers is true division, i.e. it
 produces a \code{float} quotient.  Use the double forward-slash
 operator (e.g. ``x // 3'') to obtain the C-style truncated integer
 division.
\end{itemize}

The following names are recognized in expressions:
\begin{itemize}
 \item Built-in Python constants \code{True}, \code{False}, and
 \code{None}.  

 \item Built-in Python types \code{int}, \code{float}, \code{complex},
 and \code{bool}.

 \item Built-in Python functions \code{abs}, \code{min}, and \code{max}.

 \item Constants from the \module{math} module: \code{e} and \code{pi}.

 \item Functions from the \module{math} module: \code{acos},
 \code{asin}, \code{atan}, \code{atan2}, \code{ceil}, \code{cos},
 \code{cosh}, \code{exp}, \code{floor}, \code{log}, \code{sin},
 \code{sinh}, \code{sqrt}, \code{tan}, and \code{tanh}.

 \item From the \module{hep.lorentz} module, \code{Frame} and
 \code{lab}.
\end{itemize}

In addition, expressions may use these numerical convenience functions.
(They are also available in Python programs in the \module{hep.num}
module.)

\begin{funcdesc}{gaussian}{mu, sigma, x}
Returns the probability density at \var{x} from a gaussian PDF with mean
\var{mu} and standard deviation \var{sigma}.
\end{funcdesc}

\begin{funcdesc}{get_bit}{value, bit}
Returns true iff. bit \var{bit} in \var{value} is set.
\end{funcdesc}

\begin{funcdesc}{hypot}{*terms}
A generalization of \function{math.hypot} to arbitrary number of
arguments.  Returns the square root of the sum of the squares of its
arguments.
\end{funcdesc}

\begin{funcdesc}{if_then}{condition, value_if_true, value_if_false}
Returns \var{value_if_true} if \var{condition} is true,
\var{value_if_false} otherwise.  Note that in a compiled expression
(only), the second and third arguments are evaluated lazily, so that
if \var{condition} is true, \var{value_if_false} is not evaluated, and
otherwise \var{value_if_true} is not evaluated.
\end{funcdesc}

\begin{funcdesc}{near}{central_value, half_interval, value}
Returns true if the absolute difference between \var{central_value} and
\var{value} is less than \var{half_interval}. 
\end{funcdesc}


\section{Other ways to make expressions}

You may specify a constant instead of a string when constructing an
expression with \function{asExpression} or \function{compile}.  The
resulting expression simply returns the constant.
\begin{verbatim}
>>> ex = hep.expr.asExpression(15)
>>> print ex
15
>>> print ex.type
<type 'int'>
\end{verbatim}

You may also specify a function that takes only positional arguments.
The resulting expression calls this function, using symbols for the
function arguments matching the parameter names in the function
definition.  If the function has a parameters \code{xyz}, the expression
will evaluate the symbol \code{xyz} and call the function with this
value.  For example,
\begin{verbatim}
>>> def foo(x, a):
...   return x ** a
...
>>> ex = hep.expr.asExpression(foo)
>>> print ex
foo(x, a)
>>> ex(a=8, x=2)
256
\end{verbatim}

The type of the value returned from such a function is not known. 
\begin{verbatim}
>>> ex = hep.expr.asExpression(foo)
>>> print ex.type
None
\end{verbatim}
You may specify it by attaching an attribute \member{type} to the
function, containing the expected type of the function's return value.
\begin{verbatim}
>>> foo.type = float
>>> ex = hep.expr.asExpression(foo)
>>> print ex.type
<type 'float'>
\end{verbatim}

You may also construct expressions pragmatically from the classes used
by \pyhep to represent expressions internally.  Each class represents a
single operation.  Invoke \code{help(hep.expr.classes)} for a list of
these classes and their interfaces.

Here's an example to give you an idea.
\begin{verbatim}
>>> from hep.expr import Add, Divide, Constant, Symbol
>>> mean = Divide(Add(Symbol("a"), Symbol("b")), Constant(2))
>>> print mean
(a + b) / 2
>>> mean(a=16, b=20)
18.0
\end{verbatim}


