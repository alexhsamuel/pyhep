\chapter{Files and directories}

\pyhep provides a uniform interface for accessing and manipulating
directories and their contents.  The same interface can be used not only
with file system directories, but directories in Root and HBOOK files as
well.  

Directory objects satisfy Python's map protocol, very similar to
built-in \code{dict} objects.  The keys in a directory are the names of
items in the directory.  To find the names in a directory, simply call
the \method{keys} method.  For supported file types (discussed below),
the corresponding values are the contents of the files.  To load a file,
simply get the value using the subscript operator (square brackets) or
\method{get} method.

Here are some examples of using \pyhep's directory objects.  Suppose we
are in a directory with these contents, which are various types of files
containing 
\begin{verbatim}
> ls -l
total 16
-rw-r-----  1 samuel samuel 1257 Dec  6 15:39 histogram.pickle
-rw-r-----  1 samuel samuel 3980 Dec  6 15:39 plot.pickle
-rw-r-----  1 samuel samuel   50 Dec  7 19:11 readme.txt
drwxr-x---  2 samuel samuel 4096 Dec  6 15:43 recodata
> ls -l recodata/
total 12
-rw-r-----  1 samuel samuel 4384 Dec  6 15:43 reco.root
-rw-r-----  1 samuel samuel  683 Dec  6 15:34 tracks.table
\end{verbatim}

Now, let's start Python.  First, we import the file system directory
module, \module{hep.fs}, and construct a directory object, an instance
of \class{FileSystemDirectory}, for the current working directory.
\begin{verbatim}
>>> import hep.fs
>>> cwd = hep.fs.getdir(".")
>>> cwd
FileSystemDirectory('/home/samuel/data', writable=True)
\end{verbatim}

Using the \method{keys} method, we can get the names of files in the
directory.
\begin{verbatim}
>>> print cwd.keys()
['readme.txt', 'plot.pickle', 'recodata', 'histogram.pickle']
\end{verbatim}

Directory objects have an additional \method{list} method, which prints
out directory entries and their types.
\begin{verbatim}
>>> cwd.list()
histogram.pickle    : pickle
plot.pickle         : pickle
readme.txt          : text
recodata            : directory
\end{verbatim}

Directory methods that produce or operate on the keys in the directory
can be given extra arguments.  For instance, the \code{recursive} option
will show keys in subdirectories as well.
\begin{verbatim}
>>> cwd.list(recursive=True)
histogram.pickle            : pickle
plot.pickle                 : pickle
readme.txt                  : text
recodata                    : directory
recodata/reco.root          : Root file
recodata/reco.root/hist1    : 1D histogram
recodata/reco.root/hist2    : 1D histogram
recodata/tracks.table       : table
\end{verbatim}
Notice here that \method{list} now lists the contents of the
\file{recodata} subdirectory.  One of the files in that directory is a
Root file named \file{reco.root}.  Since a Root file has an internal
directory structure, \pyhep treats it as a directory itself and the
listing descends into that file too.

Retrieving an object is as simple as looking up a key in a dictionary.
\pyhep determines the file type from its extension, and loads the object
into memory, creating a Python object of the appropriate type to
represent it.  For instance, a file with the extension \file{.pickle} is
assumed to contain a Python pickle (see the documentation for the
standard \module{pickle} and \module{cPickle} for details).  For
example, to retrieve the contents of \file{histogram.pickle},
\begin{verbatim}
>>> histogram = cwd["histogram.pickle"]
>>> histogram
Histogram(EvenlyBinnedAxis(30, (0.0, 30.0), name='energy', units='GeV'), bin_type=float, error_model='asymmetric')
\end{verbatim}

Obtain an object representing a subdirectory, whether an actual file
system subdirectory or a virtual subdirectory in a Root or HBOOK file,
in the same way.
\begin{verbatim}
>>> subdir = cwd["recodata"]
>>> print subdir.keys()
['reco.root', 'tracks.table']
>>> rootfile = subdir["reco.root"]
>>> print rootfile.keys()
['hist1', 'hist2']
\end{verbatim}
Likewise for retrieving an object from inside a Root file.
\begin{verbatim}
>>> histogram = rootfile["hist2"]
\end{verbatim}

You don't have to hang on to the intermediate directory objects if you
only want one object deep in a hierarchy of subdirectories, Root, and
HBOOK files.
\begin{verbatim}
>>> histogram = cwd["recodata"]["reco.root"]["hist2"]
\end{verbatim}
Even simpler, you can specify multiple levels with a single indexing
operation by separating keys with forward slashes.
\begin{verbatim}
>>> histogram = cwd["recodata/reco.root/hist2"]
\end{verbatim}
Note that you \emph{can not} use \file{..} to move up in the directory
tree, or specify absolute paths, as these operations break the model of
nested map objects.

To store an object to disk, simply assign a new item to the directory
object as you would with a dictionary.
\begin{verbatim}
>>> cwd["reco-hist.pickle"] = histogram
>>> print cwd.keys()
['readme.txt', 'plot.pickle', 'reco-hist.pickle', 'recodata', 'histogram.pickle']
\end{verbatim}
Of course, the file type inferred from the key's extension must support
storing the type of the value object you provide.  Python pickles can
store most Python objects, including collections such as tuples and
dictionaries, and most \pyhep objects.

Generally, the Python object is not associated with the file once it's
loaded.  If you change the Python object and want the changes reflected
in the file, you must store it back.  Directories are an exception to
this: if you add a new item to the directory, it goes immediately on
disk.  Tables are also an exception to this.

%-----------------------------------------------------------------------

\section{File types}

When you retrieve a key from a file system directory (\textit{i.e.} a
directory object corresponding to an actual directory in the file
system, not a directory in a Root or HBOOK file), \pyhep examines the
key's extension to determines how to handle the value.  

Each type has an associated type name.  When retrieving or storing items
in a directory, you may override the determination of the file type with
the keyword argument \code{type}, specifying the type name.  For
example, if you have a Root file named \file{histograms.dat} in a
directory, you may retrieve it using
\begin{verbatim}
>>> histograms = directory.get("histograms.dat", type="Root file")
\end{verbatim}

The file types understood by \pyhep, and their corresponding extensions,
are listed below.

\begin{itemize}
 \item \code{"directory"} (no extension): A directory.  The value is a
 directory object.  

 \item \code{"HBOOK file"} (extension \file{.hbook}): An HBOOK file.
 The value is a directory object representing the root of the RZ
 directory tree inside the file.  Additional information about HBOOK
 files is presented below.

 \item \code{"pickle"} (extension \file{.pickle}): A file containing a
 Python pickle.  The value is whatever Python object was pickled.

 \item \code{"Root file"} (extension \file{.root}): A Root file.  The
 value is a directory object representing the root of the directory tree
 inside the file.  Additional information about Root files is presented
 below.

 \item \code{"symlink"} (no extension): A symbolic link.  The value is a
 string containing the target of the link.  Since this type has no
 associated extension, it can only be used with the \code{type} keyword
 argument described above.

 \item \code{"table"} (extension \file{.table}): A \pyhep table.  The
 value is a handle to the open table.  In contrast to how other file
 types are handled, the table is not loaded into memory.  Changes to the
 table are reflected in the table file.

 \item \code{"text"} (extension \file{.txt}): A text file.  The value is
 a character string of the file's contents.

 \item \code{"unknown"} (all other extensions): Represents all file
 types not recognized by \pyhep.
\end{itemize}

Keys in Root and HBOOK directories are handled differently.  The file
types corresponding to these keys is determined from metadata stored in
the files themselves, not from an extension.  Additional file types are
support in these directories, including \code{"1D histogram"} and
\code{"2D histogram"}.

To determine the file type of a key, use the directory's
\method{getinfo}, method described below.

%-----------------------------------------------------------------------

\section{Methods for accessing keys and values}

As we say above, the \method{keys} method lists all keys in the
directory.  Directory objects also support the standard \method{values}
and \method{items} methods, as well as \method{iterkeys},
\method{itervalues}, and \method{iteritems}.  

For all of these, you may use these keyword arguments to restrict the
keys, values, or items that are returned:
\begin{itemize}
 \item \code{recursive}: If true, include recursively the contents of
 subdirectories. 

 \item \code{not_dir}: If true, don't include items that are directories
 (either in the file system or inside Root or HBOOK files).  

 \item \code{pattern}: Only include items whose keys match the specified
 regular expression.  See the \module{re} module for a description of
 Python's regular expression syntax.

 \item \code{glob}: Only include items whose keys match the specified
 glob pattern.  See the \module{glob} module for a description of
 Python's glob syntax.

 \item \code{only_type}: Only include an item if its type is as
 specified.  Item types are specified by strings; see below for more
 information.

 \item \code{known_types}: True by default, which specifies that only
 items of known types are included.  If you set this to false, all items
 are included.  Note that \pyhep will raise an exception if you try to
 access the value corresponding to a key of unknown type.

\end{itemize}

These options can also be used with \method{list}, which prints a
listing of keys in a directory and their types.

For example,
\begin{verbatim}
>>> print cwd.keys(only_type="pickle")
['plot.pickle', 'reco-hist.pickle', 'histogram.pickle']
>>> cwd.list(glob="hist*", recursive=True)
histogram.pickle            : pickle
recodata/reco.root/hist1    : 1D histogram
recodata/reco.root/hist2    : 1D histogram
\end{verbatim}

As with a dictionary, the \function{len} function returns the number of
keys in the directory.  
\begin{verbatim}
>>> print len(cwd)
5
\end{verbatim}
Iterating over a directory object iterates over its keys.  
\begin{verbatim}
>>> for key in cwd:
...   print key
...
readme.txt
plot.pickle
reco-hist.pickle
recodata
histogram.pickle
\end{verbatim}
The \code{has_key} method and \code{in} operator return true if the
specified key is in the directory.
\begin{verbatim}
>>> print cwd.has_key("readme.txt")
True
>>> print "missing.pickle" in cwd
False
\end{verbatim}

To retrieve an object, use the Python indexing notation (square
brackets) or the \method{get} method.  If the key is not found in the
dictionary, indexing will raise \class{KeyError}; the \method{get}
method will return a default value, which you can specify as the second
argument (the default is \code{None}).
\begin{verbatim}
>>> print cwd["missing.pickle"]
Traceback (most recent call last):
...
KeyError: 'missing.pickle'
>>> print cwd.get("missing.pickle")
None
>>> print cwd.get("missing.pickle", 42)
42
\end{verbatim}

Use the \function{hep.fs.getdir} function to return a directory object
for an arbitrary path.  The path may be absolute or relative to the
current working directory.  The path may descend into Root and HBOOK
files.  For example,
\begin{verbatim}
>>> data_dir = hep.fs.getdir("/nfs/data")
>>> histograms = hep.fs.getir("histograms.root/reco")
\end{verbatim}
You can also use \function{hep.fs.get} to load files of other types by
specifying the path to the file.

%-----------------------------------------------------------------------

\section{Accessing file information}

A directory object's \method{getinfo} method returns an \class{Info}
object containing information about the file.  The \class{Info} object
is constructed without loading the file.

All \class{Info} objects have an attribute \member{type}, which is the
file type for that file.  See above for a discussion of file types.  An
\class{Info} object may have additional attributes, depending on the
type of directory object it was obtained from.

File system \class{Info} objects also contain attributes \member{path},
\member{file_size}, \member{user_id}, \member{group_id},
\member{access_mode}, and \member{modification_time}.

The code below demonstrates the use of \method{getinfo} to print a
listing of a file system directory.
\begin{verbatim}
>>> def listFSDir(directory):
...   for key in directory.keys(known_types=False):
...     info = directory.getinfo(key)
...     print ("%-24s (%-12s)  %8d bytes, mode %06o, owner %4d"
...            % (key, info.type, info.file_size, info.access_mode, info.user_id))
...
>>> listFSDir(cwd) 
readme.txt               (text        )        50 bytes, mode 000640, owner  500
plot.pickle              (pickle      )      3980 bytes, mode 000640, owner  500
reco-hist.pickle         (pickle      )      1252 bytes, mode 000640, owner  500
recodata                 (directory   )      4096 bytes, mode 000750, owner  500
histogram.pickle         (pickle      )      1257 bytes, mode 000640, owner  500
\end{verbatim}

%-----------------------------------------------------------------------

\section{Storing data}

To store data in a file, simply set a key in the dictionary object.  For
file system directories, the file type is determined from the key's
extension.  The file type must support the data type you provide.  For
example,
\begin{verbatim}
>>> text = "Hello, world."
>>> cwd["hello.txt"] = text
>>> cwd["hello.pickle"] = text
>>> cwd["hello.root"] = text
Traceback (most recent call last):
...
AttributeError: 'str' object has no attribute 'keys'
\end{verbatim}
Here, the first assignment stores the text in a plain text file, and the
second stores it as a pickled Python string.  The third assignment
fails, since it doesn't make sense to create a Root file with a string.

You can also use the \method{set} method to set keys.  With
\method{set}, you can specify keyword arguments.  For example, you can
override the file type determination with the \code{type} argument.
For example, this assignment tell \pyhep to store a plain text file,
even though the extension \file{.log} is not known.
\begin{verbatim}
>>> cwd.set("hello.log", text,type="text")
\end{verbatim}

You can other standard map methods to modify the directory's contents as
well: \method{setdefault}, \method{update}, and \method{popitem}.  To
delete files, use the \code{del} statement or the directory's
\method{delete} method.  The \method{clear} method empties the
directory.

These methods can take additional keyword arguments that controls how
directories are modified:
\begin{itemize}
 \item \code{deldirs} (true by default): Allow automatic recursive
 deletion of directories and their contents.

 \item \code{makedirs} (true by default): Create missing intermediate
 subdirectories automatically when setting a key (like ``\code{mkdir
 -p}''). 

 \item \code{replace} (true by default): If true, allow keys to be
 replaced.  Otherwise, raise an exception when setting a key that
 already exists.

 \item \code{replacedirs} (false by default): Like \code{replace} but
 applies to subdirectories.
\end{itemize}

You can also create a directory by setting a key.  Since a directory is
represented by a map, it looks like this:
\begin{verbatim}
>>> cwd["subdir"] = {}
\end{verbatim}
You may populate the map you assign; the items are stored as files in
the new directory.
\begin{verbatim}
>>> cwd["subdir"] = { "contents.txt": "calibration constants", 
...                   "constants.pickle": (10, 11, 12) }
\end{verbatim}
To create a directory if it doesn't exist, or obtain it if it does, use
\method{setdefault}:
\begin{verbatim}
>>> output_dir = cwd.setdefault("output", {})
\end{verbatim}

%-----------------------------------------------------------------------

\section{Working with HBOOK files}

You can use directory objects, as described above, to access contents of
HBOOK files.  An HBOOK file is represented by a directory object, as is
a subdirectory in an HBOOK file.

For example, to create a new HBOOK file in the directory represented by
\code{data_dir} containing two histograms,
\begin{verbatim}
>>> data_dir["histograms.hbook"] = { "hist1": histogram1,
...                                  "hist2": histogram2 }
\end{verbatim}
To create a new, empty HBOOK file or return the existing one if it
exists,
\begin{verbatim}
>>> hbook_file = data_dir.setdefault("histograms.hbook", {})
\end{verbatim}

\subsection{The \module{hep.cernlib.hbook} module}

The \module{hep.cernlib.hbook} module contains \pyhep's implementation
of HBOOK file access.  HBOOK is a part of the CERNLIB library, and
provides a structured file format containing histograms and n-tuples.
Note that not all HBOOK features are supported.

Some details particular to HBOOK directory objects:
\begin{itemize}
 \item \pyhep's module \module{hep.cernlib.hbook} is linked statically
 against CERNLIB 2002.

 \item HBOOK's names are case-insensitive.  \pyhep always uses lower
 case for names in HBOOK files.

 \item To load a histogram from an HBOOK directory, simply obtain it by
 name using the subscript operator or \method{get}.  This returns a
 Python object representing the histogram, which may be modified freely.
 Note that unlike HBOOK itself, where histograms are always stored in a
 global ``PAWC'' memory region, \pyhep constructs ordinary Python
 objects for histograms.  There is no need to manage ``PAWC''
 explicitly.

 \item Not all histogram features supported in HBOOK are also supported
 in \pyhep, and visa versa.  Therefore, if a histogram is saved to an
 HBOOK file and later loaded, it may differ in some of its
 characteristics.  The basic histogram binning, and bin contents and
 errors (including overflow and underflow bins), are stored correctly,
 however.  Note that \pyhep does not provide profile histograms.

 \item An HBOOK file is not closed until the file object is destroyed,
 i.e. all references to it are released.  Especially when writing an
 HBOOK file, be careful to release all references to the file object.

 \item When \pyhep closes an HBOOK file, it ``purges cycles'',
 \textit{i.e.} removes old revisions of all entries stored in the file.
 To disable this behavior, specify \code{purge_cycles=False} as a
 keyword argument when creating or accessing the HBOOK file.

 \item In an HBOOK file, each object is given not only a name
 identifying it in the directory that contains it, but a numerical RZ
 identification as well.  When you retrieve an object from an HBOOK file,
 \pyhep stores this value in the object's \member{rz_id} attribute.
 When you store an objet from an HBOOK file, you may specify the value
 to use either by setting the object's \member{rz_id} attribute or by
 using an \code{rz_id} keyword argument; otherwise, \pyhep chooses an
 available value.  An info object obtained from an HBOOK directory's
 \method{getinfo} method also has an \member{rz_id} attribute.

\end{itemize}


\subsection{N-tuples}

An HBOOK n-tuple is represented in \pyhep by a table.  The table
satisfies the same protocol as the default table implementation (see
\module{hep.table}), except in the method to create or open tables.
Note that because of HBOOK's limitations, certain table features are not
supported.

To access an n-tuple in an HBOOK file, use the file object's subscript
operator or \method{get} method, just access the n-tuple's name, just as
you would for a histogram.  Unlike a histogram, the table object is
still connected to the n-tuple in the HBOOK file.  A new row appended to
the table is incorporated immediately into the n-tuple.  Also, the table
object carries a reference to the HBOOK file, in its \member{file}
attribute, so the HBOOK file is not closed as long as there is an
outstanding table object for an n-tuple in the file.

To create a new n-tuple in an HBOOK file, use the
\function{hep.cernlib.hbook.createTable} function.  The arguments are the
name of the n-tuple, the HBOOK directory object in which to create the
n-tuple, and the schema (as for \function{hep.table.create}).  You may
use the optional \var{rz_id} argument to specify the n-tuple's RZ ID.

By default, a column-wise n-tuple is used for the table; to create a
row-wise n-tuple, pass a false value for the optional \var{column_wise}
argument to \function{createTable}.  When creating a column-wise n-tuple,
the schema may only contain columns of types \constant{"int32"},
\constant{"int64"}, \constant{"float32"}, and \constant{"float64"}.  The
schema for a row-wise n-tuple may use only \constant{"float32"} columns.

This program creates an HBOOK file containing a row-wise n-tuple filled
with random values.  It then re-opens the file, creates a histogram from
the values, and stores it in the file.
\codesample{hbook1.py}

%-----------------------------------------------------------------------

\section{Working with Root files}

Root set of libraries and programs for high energy physics analysis.
Among other things, Root provides a file format for histograms, n-tuples
(which are called ``trees'' in Root), and other data objects.  \pyhep
provides partial data and file compatibility with Root.

You can use directory objects, as described above, to access contents of
Root files.  An Root file is represented by a directory object, as is a
subdirectory in an Root file.

For example, to create a new Root file in the directory represented by
\code{data_dir} containing two histograms,
\begin{verbatim}
>>> data_dir["histograms.root"] = { "hist1": histogram1,
...                                 "hist2": histogram2 }
\end{verbatim}
To create a new, empty Root file or return the existing one if it
exists,
\begin{verbatim}
>>> root_file = data_dir.setdefault("histograms.root", {})
\end{verbatim}

The API and capabilities of \module{hep.root} are very similar to those
of \module{hep.cernlib.hbook}.  A program written for one can be used
with the other with minimal modification, and it is easy to write
functions, scripts, or programs that can work files from either format.


\subsection{The \module{hep.root} module}

The \module{hep.root} contains \pyhep's implementation of Root file
access.  

\pyhep's module \module{hep.root} is built and linked against Root shared
libraries which are distributed with \pyhep.  The module does not depend
on any other version of Root installed on your system.

Some details particular to Root directory objects:
\begin{itemize}
 \item A Root file is not closed until the file object is destroyed,
 i.e. all references to it are released.  Especially when writing a Root
 file, be careful to release all references to the file object.

 \item Not all histogram features supported in Root are also supported
 in \pyhep, and visa versa.  Therefore, if a histogram is saved to a
 Root file and later loaded, it may differ in some of its
 characteristics.  For instance, any additional attributes added to the
 histogram will be lost.  The basic histogram binning, and bin contents
 and errors (including overflow and underflow bins), are stored
 correctly, however.

 \item In a Root file, each object is given not only a name identifying
 it in the directory that contains it, but a title.  When you retrieve
 an object from an Root file, \pyhep stores this value in the object's
 \member{title} attribute.  When you store an object from an Root file,
 you may specify the title to use either by setting the object's
 \member{title} attribute or by using an \code{title} keyword argument.
 An info object obtained from an Root directory's \method{getinfo}
 method also has an \member{title} attribute.

\end{itemize}


\subsection{Trees}

An Root tree is represented in \pyhep by a table.  The table satisfies
the same protocol as the default table implementation (see
\module{hep.table}), except in the method to create or open tables.
Note that because of Root's limitations, certain table features are not
supported.

To open a tree in a Root file as a table, use the file object's
subscript operator or \method{get} method, just as you would for a
histogram.  Note that unlike a histogram returned from \method{load},
though, the table object that \method{load} returns is still connected
to the tree in the Root file.  A new row appended to the table is
incorporated in the tree.  Also, the table object carries a reference to
the Root file, in its \member{file} attribute, so the Root file is not
closed as long as there is an outstanding table object for a tree in the
file.

To create a new tree in a Root file, use the
\function{hep.root.createTable} function.  The arguments are the name of
the new tree, the Root directory in which to create it, and the schema
(as for \function{hep.table.create}).  You may specify a title for the
table with the \var{title} argument.

This program creates a Root file containing a row-wise n-tuple filled
with random values.  It then re-opens the file, creates a histogram from
the values, and stores it in the file.
\codesample{root1.py}

