\chapter{Tables}

\pyhep provides a flat-format database facility, similar to ``n-tuples''
in other HEP analysis systems.  

A \pyhep table is similar to a table in a typical database system.  A
table is a sequence of rows, each of which has the same number of
columns and values for each column of the same type.  The number and
types of columns constitutes the table's schema.  

\pyhep's tables are subject to the following restrictions:

\begin{itemize}

 \item A table's schema is set when it is created, and cannot
 subsequently be changed.  (Creating another table with a different
 schema using the same data is straightforward.)

 \item The size in bytes of each table row is a constant.  This limits
 the types of table columns, but allows the implementation of fast seeks
 in a large table.

 \item Rows are appended to the table.  An entire row must be appended
 at one time.  Rows may not be inserted into the middle of a table,
 modified, or removed.

\end{itemize}

\section{Table implementations}

Several table implementations are provided.  Other than for creating new
tables or opening existing tables, the interfaces of these
implementations are identical.

\begin{enumerate}

 \item A extension-class implementation written in C++ in the
 \module{hep.table} module.  The table resides in a disk file, and rows
 are loaded into memory as needed.  The implementation is simple and
 efficient.  Other programs may write and access these tables via a
 simple C++ interface.

 \item An implementation which uses HBOOK n-tuples (column-wise or
 row-wise) stored in HBOOK files.  Not all features of HBOOK n-tuples are
 supported.

 \item An implementation which uses ROOT tress stored in ROOT files.
 Not all features of ROOT trees are supported.

\end{enumerate}

The extension-class implementation in \module{hep.table} is used in this
tutorial. 

\section{Creating and filling tables}

A table schema is represented by an instance of
\class{hep.table.Schema}.  The schema collects together the definitions of the columns in the table.  Each column is identified by a name, which is
a string composed of letters, digits, and underscores.  

A new instance of \class{Schema} has no columns.  Add columns using the
\method{addColumn} method, specifying the column name and type.  For instance, 
\begin{verbatim}
>>> import hep.table
>>> schema = hep.table.Schema()
>>> schema.addColumn("energy", "float64")
>>> schema.addColumn("momentum", "float64")
>>> schema.addColumn("hits", "int32")
\end{verbatim}

The second argument to \function{addColumn} specifies the storage format
used for values in that column.  These column types are supported in
\code{hep.table} (note that other table implementations may not support
all of these):
\begin{itemize}
 \item three signed integer types: \constant{"int8"}, \constant{"int16"},
 and \constant{"int32"}

 \item two floating-point types, \constant{"float32"} and
 \constant{"float64"}

 \item two floating-point complex types, \constant{"complex64"} and
 \constant{"complex128"}
\end{itemize}

More concisely, you may specify columns as keyword arguments, so the
above schema may be constructed with,
\begin{verbatim}
>>> schema = hep.table.Schema(energy="float64", momentum="float64", hits="int32")
\end{verbatim}

You can also load or save schemas in an XML file format using
\function{hep.table.loadSchema} and \function{hep.table.saveSchema}.
The schema above would be represented by the XML file
\begin{verbatim}
<?xml version="1.0" ?>
<schema>
 <column name="energy" type="float64"/>
 <column name="momentum" type="float64"/>
 <column name="hits" type="int32"/>
</schema>
\end{verbatim}

Call the \function{create} function to create a new table.  The
parameters will vary for each implementation; for the \module{hep.table}
implementation, the parameters are the file name for the new table, and
the table's schema.  
\begin{verbatim}
>>> table = hep.table.create("test.table", schema)
\end{verbatim}
The return value is a \emph{connection} to the newly-created table,
which resides on disk.

To add a row to the table, call the table's \method{append} method.  You
may pass it a mapping argument (such as a dictionary) that associates
column values with column names, and/or you may provide column values as
keyword arguments.  One way or another, you must specify values for all
columns in the table.  The return value of \method{append} is the index
of the newly-appended row.

For instance, to add a row to the table created above,
\begin{verbatim}
>>> row = {
...   "energy": 2.7746,
...   "momentum": 1.8725,
...   "hits": 17,
>>> }
>>> table.append(row)
\end{verbatim}
or equivalently,
\begin{verbatim}
>>> table.append(energy=2.7746, momentum=1.8725, hits=17)
\end{verbatim}

\subsection{Example: creating a table from a text file}

The script below converts a table of values in a text file into a
\pyhep table.  The script assumes that the file contains floating-point
values only, except that the first line of the text file contains
headings that will be used as the names of the columns in the table.
\codesample{table1.py}

Here is a table containing parameters for tracks measured in a
detector.  The first column is the track's energy; the other three are
the x, y, and z components of momentum.
\codesample{tracks.txt}

If you save the script as \file{txt2table.py} and the table as
\file{tracks.table}, you would invoke this command to convert it to a
table: 
\begin{verbatim}
> python txt2table.py tracks.txt tracks.table
\end{verbatim}

\section{Using tables}

To open an existing table, use the \function{open} function.  The first
argument is the table file name.  The second argument is the mode in
which to open the table, similar to the built-in \function{open}
function: \constant{"r"} (the default) for read-only, \constant{"w"} for
write.  

For instance, to open the table we created in the last section,
\begin{verbatim}
>>> tracks = hep.table.open("tracks.table")
\end{verbatim}

The table object is a sequence whose elements are the rows.  Each row is
has a row index, which is equal to its position in the table.  Thus, the
\function{len} function returns the number of rows in the table.
\begin{verbatim}
>>> number_of_tracks = len(tracks)
\end{verbatim}
To access a row by its row index, use the normal sequence index
notation.  This returns a \class{Row} object, which is a mapping from
column names to values.
\begin{verbatim}
>>> track = tracks[19]
>>> print track["energy"]
\end{verbatim}

The table's \member{schema} attribute contains the table's schema; the
schema's \member{column} attribute is a sequence of the columns in the
schema (in unspecified order).  You can examine these directly, or use
the \function{dumpSchema} function to print out the schema.
\begin{verbatim}
>>> hep.table.dumpSchema(tracks.schema)
name             type
-----------------------------
energy           float32
p_x              float32
p_y              float32
p_z              float32
 
\end{verbatim}


\section{Iterating over rows}

For most HEP applications, the rows of a table represent independent
measurements, and are processed sequentially.  An \emph{iterator}
represents this sequential processing of rows.  Using iterators instead
of indexed looping constructs simplifies code, opens up powerful
functional-programming methods, and enables automatic optimization of
independent operations on rows.

Since a table satisfies the Python sequence protocol, you can produce an
iterator over its elements (\textit{i.e.} rows) with the \function{iter}
function.  The Python \code{for} construction does even this
automatically.  The simplest idiom for processing rows in a table
sequentially looks like this:
\begin{verbatim}
>>> total_energy = 0
>>> for track in tracks:
...     total_energy += track["energy"]
...
>>> print total_energy
74.7496351004
\end{verbatim}

Since \code{iter(tracks)} is an iterator rather than a sequence of all
rows, each row is loaded from disk into memory only when needed in the
loop, and is subsequently deleted.  This is critical for scanning over
large tables.  (Note that after the loop completes, the last row remains
loaded, until the variable \var{track} is deleted or goes out of scope.
Also, the table object itself is deleted only after any variables that
refer to it, as well as any variables that refer to any of its rows, are
deleted.)

Table iterators may be used to iterate over a subset of rows in a
sequence.  Most obviously, you could implement this by using a
conditional in the loop.  For instance, to print the energy of each
track with an energy greater than 2.5,
\begin{verbatim}
>>> for track in tracks:
...     if track["energy"] > 2.5:
...         print track["energy"]
... 
\end{verbatim}
While this is straightforward, it forces \pyhep to examine each row
every time the program is run.  By using the selection in the iterator,
\pyhep can optimize the selection process, often significantly.
The selection criterion can be any boolean-valued expression involving
the values in the row.  The selection expression is evaluated for each
row, and if the result is true, the iterator yields that row; otherwise,
that row is skipped.  (This is semantically similar to the first
argument of the built-in \function{filter} function.)  The expression
can be a string containing an ordinary Python expression using column
names as if they were variable name (with certain limitations and
special features), or may be specified in other ways.  Expressions are
discussed in greater detail later.

For instance, the same high-energy tracks can be obtained using the
selection expression \constant{"energy > 2.5"}.  Notice that
\var{energy} appears in this expression as if it were a variable defined
when the expression is evaluated.  The \method{select} method returns an
iterator which yields only rows for which the selection is true.
\begin{verbatim}
>>> for track in tracks.select("energy > 2.5"):
...     print track["energy"]
... 
\end{verbatim}

Python's list comprehensions provide a handy method for collecting
values from a table.  For instance, to enumerate all values of
\var{energy} above 2.5 instead of merely printing them,
\begin{verbatim}
>>> energies = [ track["energy"] for track in tracks.select("energy > 2.5") ]
\end{verbatim}
Note here that \code{tracks.select("energy > 2.5")} returns an iterator
object, so it may only be used (i.e. iterated over) once.  However, if
you really want a sequence of \code{Row} objects, you can use the
\function{list} or \function{tuple} functions to expand an iterator into
an actual sequence, as in
\begin{verbatim}
>>> high_energy_tracks = list(tracks.select("energy > 2.5"))
\end{verbatim}
Such a sequence consumes more resources than an iterator.  You can
iterate over such a sequence repeatedly, or perform other sequence
operations. 

 
\section{Projecting histograms from tables}

If you have a sequence or iterator of values, you can use the built-in
\function{map} function to accumulate them into a histogram.  For
instance, if \var{energies} is a sequence of energy values, you could
fill them into a histogram using,
\begin{verbatim}
>>> energy_hist = hep.hist.Histogram1D(20, (0.0, 5.0))
>>> map(energy_hist.accumulate, energies)
\end{verbatim}

Generally, though, you will want to project many histograms at one time
from a table.  Use the \function{hep.hist.project} function to project
multiple histograms in a single scan over a table.  Pass an iterable
over the table rows to project---the table itself, or an iterator
constructed with the \method{select} method---and a sequence of
histogram objects.  Each histogram should have an attribute
\member{expression}, which is the expression whose value is accumulated
into the histogram for each table row.  The expression are similar to
those used with the \method{select} method, except that their values
should be numerical instead of boolean.

For example, this script projects three histograms -- energy, transverse
momentum, and invariant mass, from the table of tracks we constructed
previously.  Only high-energy tracks (those with energy above 2.5) are
included.  The dictionary containing the histograms is stored in a
standard pickle file.
\codesample{project1.py}
Observe that the last histogram is two-dimensional, and its expression
specifies the two coordinates of the sample using a comma expression.

With this scheme, you can determine later how the values in a histogram
were computed, by checking its \member{expression} attribute.


\section{Using expressions with tables}

Expressions are described in more detail in a later chapter.  This
section presents techniques for optimizing expressions used with tables.

As described above, you can use Python expressions encoded in character
strings with \method{select} method and \function{hep.hist.project}
function.  When such an expressions are evaluated, unbound symbols
(\textit{i.e.} variables) are bound to the value of the corresponding
column.

Since a table row satisfies the Python mapping protocol, you can pass a
table row directly to an expression's \method{evaluate} method.  For
example, this constructs an expression object to compute the transverse
momentum of a track in the tracks table constructed above.
\begin{verbatim}
>>> import hep.expr
>>> p_t = hep.expr.asExpression("hypot(p_x, p_y)")
\end{verbatim}
Its \method{evaluate} method computes transverse momentum for a track by
binding \var{p_x} and \var{p_y} to the corresponding values of the table
row. 
\begin{verbatim}
>>> print p_t.evaluate(tracks[0])
0.92102783203
\end{verbatim}
To find the largest transverse momentum in the track table,
\begin{verbatim}
>>> print max(map(p_t.evaluate, tracks))
2.44177757441
\end{verbatim}

The chapter on expressions, below, describes how to compile an
expression into a format for faster evaluation.  When the expression is
used with a table, you can produce an even faster version by compiling
it with the table's \method{compile} method.  This sets the symbol types
correctly based on the table's schema, and applies additional
optimizations specific to tables.  For example,
\begin{verbatim}
>>> p_t = tracks.compile("hypot(p_x, p_y)")
>>> print p_t.evaluate(tracks[0])
0.92102783203
\end{verbatim}
When you evaluate an expression on many rows of a large table,
performance will be substantially better if you compile the expression
first for that table.  Note that an expression compiled for a table
should only be used with that table.


\subsection{Caching expressions in tables}

You can also configure a table to cache the results of evaluating a
boolean expression on the rows.  The first time you evaluate the
expression on a row, the row is loaded and the expression is evaluated
as usual.  On subsequent times, the table reuses the cached value of the
expression, instead of reloading the row and re-evaluating the
expression. 

To instruct a table to cache the value of an expression, pass the
expression to the table's \method{cache} method.  Do not pass a compiled
expression; pass the uncompiled form instead.  You may pass a string or
function here as well.  Once you add the expression to the table's
cache, the expression cache will automatically be used when you compile
the expression or use it with \method{select} and other table functions.

For example, suppose your table file \file{tracks.table} was extremely
large, and you expect to select repeatedly all tracks with energy above
2.5.  You could cache this selection expression like this:
\begin{verbatim}
>>> cut = "energy > 2.5"
>>> tracks.cache(cut)
\end{verbatim}
\begin{verbatim}
>>> cut = tracks.compile(cut)
\end{verbatim}
Optionally, compile the expression and evaluate it on each row once, to
fill the cache.
\begin{verbatim}
>>> cut = table.compile(cut)
>>> for track in tracks:
...   cut.evaluate(track)
... 
\end{verbatim}

Once you have added a cached expression to the table, the cache will be
used automatically whenever you compile the expression for that table:
the compiled \code{cut} above uses the cache.  The cache will also be
used for subexpressions, so if you were to compile the expression
\code{"mass < 1 and energy > 2.5"} would use the cache, too.

\subsection{Row types}

As we have seen above, the object representing one row of a table
satisfies Python's read-only mapping protocol: it maps the name of a
column to the corresponding value in the row.  While in many ways, they
behave like ordinary Python dictionaries (for instance, they support the
\method{keys} and \method{items} methods), they actually instances of
the \class{hep.table.RowDict} class.
\begin{verbatim}
>>> track = tracks[0]
>>> print type(track)
<type 'RowDict'>
\end{verbatim}
If you ever need an actual dictionary object containing the values in a
row, Python will produce that for you:
\begin{verbatim}
>>> dict(track)
{'energy': 1.1803040504455566, 'p_z': -0.73052072525024414, 
 'p_x': -0.47519099712371826, 'p_y': -0.78897768259048462}
\end{verbatim}

For some applications, however, it is more convenient to use a different
interface to access row data.  You can specify another type to use for
row objects by setting the table's \member{row_type} attribute (or with
the \var{row_type} keyword argument to \function{hep.table.open}).  The
default value, as you have seen, is \class{hep.table.RowDict}.

\pyhep includes a second row implementation,
\class{hep.table.RowObject}, which provides access to row values as
object attributes instead of items in a map.  The row has an attribute
named for each column in the table, and the value of the attribute is
the corresponding value in the row.

For example,
\begin{verbatim}
>>> tracks.row_type = hep.table.RowObject
>>> track = tracks[0]
>>> print track.p_x, track.p_y, track.p_z
-0.475190997124 -0.78897768259 -0.73052072525
\end{verbatim}

You may also derive a subclass from \class{RowObject} and use that as
your table's row type.  This is very handy for adding additional
methods, get-set attributes, etc. to the row, for instance to compute
derived values.

For example, you could create a \class{Track} class that provides the
mass and scalar momentum as ``attributes'' that are computed dynamically
from the row's contents.
\begin{verbatim}
>>> from hep.num import hypot
>>> from math import sqrt
>>> class Track(hep.table.RowObject):
...   momentum = property(lambda self: hypot(self.p_x, self.p_y, self.p_z))
...   mass = property(lambda self: sqrt(self.energy**2 - self.momentum**2))
...
\end{verbatim}
Now set this class as the row type.
\begin{verbatim}
>>> tracks.row_type = Track
>>> track = tracks[0]
>>> print type(track)
<class '__main__.Track'>
\end{verbatim}
You can now access the members of \class{Track}:
\begin{verbatim}
>>> print track.momentum
1.17556488438
>>> print track.mass
0.105663873224
\end{verbatim}

Setting a table's row type to \class{RowObject} or a subclass will not
break evaluation of compiled expressions on row objects.  Expressions
look for a \method{get} method, which is provided by both
\class{RowDict} and \class{RowObject}.

Note that computing complicated derived values in this way is less
efficient than using compiled expressions, as described above.  However,
you can create methods or get-set members that evaluate compiled
expressions.  

Only subclasses of \class{RowDict} and \class{RowObject} may be used as
a table's \member{row_type}.


\section{More table functions}

The function \function{hep.table.project} is a generalization of
\function{hep.hist.project}.  Like the latter, it iterates over table
rows and evaluates a set of expressions on each row.  Instead of
accumulating the expression values into histograms, passes them to
arbitrary functions.  Invoke \code{help(hep.table.project)} for usage
information. 

The \class{hep.table.Chain} class concatenates multiple tables into one.
Simply pass the tables as arguments.  Of course, any columns accessed in
the chain must appear in all the included tables.  For example, this
code chains together all table files (files with the ".table" extension)
in the current working directory.
\begin{verbatim}
>>> import os
>>> tables = [ hep.table.open(path)
...            for path in os.listdir(".") 
...            if path.endswith(".table") ]
>>> chain = hep.table.Chain(*tables)
\end{verbatim}


