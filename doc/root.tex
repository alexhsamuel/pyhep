\section{\module{hep.root} -- Root interface}

\declaremodule{extension}{hep.root}
\modulesynopsis{Root interface}

The \module{hep.root} module provides access from Python to features in
Root, a framework for high energy physics data management and analysis.
The applicable components of Root are included with \pyhep and need not
be provided separately.

\subsection{Root files}

Root includes an implementation of histograms and ntuples (which are
called ``trees'' in Root).  It implements a file format for storing
these in structured disk files.  \pyhep provides read and write access
to histograms and trees stored in Root files.

The \function{open} function access a Root file:

\begin{funcdesc}{open}{path\optional{, mode="r", purge_cycles=1}}
 Create or open a Root file at \var{path}.  

 \var{mode} is the access mode, analogous to Python's built-in
 \function{file} and \function{open} functions.  If \code{"r"}, opens an
 existing file for reading.  If \code{"r+"}, \code{"w"}, \code{"a"}, or
 \code{"a+"}, opens an existing file for reading and writing, or creates
 the file if it doesn't exist.  If \code{"w+"}, creates a file or
 overwrites an existing file.

 If \var{purge_cycles} is set to a true value and the file is open for
 writing, the entire file will be purged when it is closed.  
\end{funcdesc}

The return value from \function{open} is a \class{File} object.  The
file remains open until it is deleted.  

In \pyhep, all paths in Root files are specified relative to the root
directory of the Root file.  The root directory itself is represented by
an empty string, and the path separator is a forward slash.  Note that
there is no notion of working directory or relative paths.

Root files also support revision cycles; see the Root documentation for
details.  Unless \function{root.open} is invoked with
\code{purge_cycles=0} (or the file is opened read-only), \pyhep
``purges'' the entire file when the file is closed.  This removes cycles
except the most recent for each object in the file, whether or not the
object was written or revised in the current session.

\pyhep stores additional information along with tables and histograms in
Root files.  This information, called \emph{metadata}, consists of
useful information about histograms and tables that is not handled
directly by the Root file format.  \pyhep stores metadata in
subdirectories named ``_pyhep_metadata'' in the Root file.  These
subdirectories and their contents may be removed safely without injuring
the primary data in the histograms and tables they are associated with,
but doing so will delete configuration and annotations accessible from
\pyhep.

\subsubsection{\class{File} objects}

A \class{File} object has the following methods and attributes:

\begin{memberdesc}{writeable}
 True if the file is open for writing as well as reading.
\end{memberdesc}

\begin{methoddesc}{listdir}{\optional{path}}
 Return the contents of the directory specified by \var{path}.  The
 return value is a sequence of \code{(name, title, class_name)} tuples,
 where \var{name} is the object's name, \var{title} is the object's
 title, and \var{class_name} is the object's class name.
\end{methoddesc}

\begin{methoddesc}{mkdir}{path\optional{, title}}
 Create a new directory at \var{path}.  Optionally, a title may be
 specified in the \var{title} argument.
\end{methoddesc}

\begin{methoddesc}{rmdir}{path}
 Remove the directory at \var{path}.
\end{methoddesc}

\begin{methoddesc}{save}{path, object}
 Store a histogram \var{object} at \var{path}.  One- and two-dimensional
 histograms are supported.  
\end{methoddesc}

\begin{methoddesc}{load}{path}
 Load a histogram or connect to an ntuple at \var{path}.  If \var{path}
 is a one- or two-dimensional histogram, returns a histogram object with
 its contents.  If \var{path} is a tree, returns a table object (see
 \module{hep.table}) accessing it.
\end{methoddesc}

\begin{methoddesc}{createTable}{path, schema\optional{, title,
separate_branches=0, branch_name="default"}}
 Create a table as a tree at \var{path} in the Root file, using
 columns from \var{schema}, a \class{hep.table.Schema} instance.  

 By default, column values is stored in a leaves grouped in a single
 branch, named "default".  Another name may be specified for the branch
 with the \var{branch_name} argument.  Alternately, if
 \var{separate_branches} is true, each leaf is placed in a separate
 branch, whose name is the column name.

 These column types are supported: \constant{"int8"},
 \constant{"int16"}, \constant{"int32"}, \constant{"float32"}, and
 \constant{"float64"}.
\end{methoddesc}

\begin{methoddesc}{remove}{path}
 Remove the tree or histogram at \var{path}.  Use \method{rmdir} to
 remove a directory.
\end{methoddesc}

\begin{methoddesc}{normpath}{path}
 Return the canonical form of absolute path \var{path} in the Root
 file.
\end{methoddesc}

\begin{methoddesc}{join}{*components}
 Join together \var{components} to form an absolute path.
\end{methoddesc}

\begin{methoddesc}{split}{path}
 Return \code{(dir, name)}, where \var{dir} is the portion of \var{path}
 up to the last directory separator, and \var{name} is the remainder.
\end{methoddesc}

