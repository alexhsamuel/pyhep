\section{\module{hep.cernlib} -- CERNLIB interface}

\declaremodule{extension}{hep.cernlib}
\modulesynopsis{CERNLIB interface}

The \module{hep.cernlib} module provides access from Python to features
in CERNLIB, the CERN program library for high energy physics.  The
applicable components of CERNLIB are linked into \pyhep and need not be
provided separately.

\subsection{\module{hep.cernlib.hbook} -- HBOOK}

\declaremodule{extension}{hep.cernlib.hbook}
\modulesynopsis{HBOOK interface}

HBOOK is a system for managing histograms and ntuples.  It implements a
file format for storing these in structured disk files.  \pyhep provides
read and write access to histograms and ntuples stored in HBOOK files.

The module \module{hep.cernlib.hbook} provides these functions:

\begin{funcdesc}{open}{path\optional{, mode="r", record_length=1024,
purge_cycles=1}}
 Create or open an HBOOK file at \var{path}.  

 \var{mode} is the access mode, analogous to Python's built-in
 \function{file} and \function{open} functions.  If \code{"r"}, opens an
 existing HBOOK file for reading.  If \code{"r+"}, \code{"w"},
 \code{"a"}, or \code{"a+"}, opens an existing file for reading and
 writing, or creates the file if it doesn't exist.  If \code{"w+"},
 creates a file or overwrites an existing file.

 \var{record_length} is the RZ record length to use for this file.  

 If \var{purge_cycles} is set to a true value and the file is open for
 writing, the entire file will be purged when it is closed.
\end{funcdesc}

The return value from \function{open} is a \class{File} object.  The
file remains open until it is deleted.  

\subsubsection{Paths, IDs, and cycles}

In \pyhep, all paths in HBOOK files are specified relative to the root
directory of the HBOOK file.  The root directory itself is represented
by an empty string, and the path separator is a forward slash.  Note
that there is no notion of working directory or relative paths.

The last component of a path is the title of a histogram, ntuple, or
directory.  HBOOK does not require titles to be unique, even in the same
directory.  However, each entry is also assigned a positive integer ID
number (which is called the RZ ID, because HBOOK's file format is built
on top of the RZ input/output system).  \pyhep's functions for creating
tables and directories and saving histograms will choose an available RZ
ID by default, but you can specify a different value to use (except for
directories).

Thus, HBOOK allows you to place multiple items with the same title in
the same directory, but each must have a unique RZ ID.  When loading or
deleting histograms or ntuples, you may specify either the RZ ID instead
of the title in the path.  For example, a histogram with title MYHIST
and RZ ID 17 in directory DIR/SUBDIR can be accessed as
\code{"DIR/SUBDIR/MYHIST"} or \code{"DIR/SUBDIR/17"}.  When saving
histograms, though, you must specify the title in the path.

Note that because of how HBOOK keeps ntuples in memory while they're
being accessed, you may not access two ntuples with the same RZ ID at
the same time, even if the ntuples are stored in different HBOOK files
or directories.  Be sure to delete the \class{Table} object for one
before creating or loading another.

HBOOK files also support revision cycles; see the HBOOK or RZ
documentation for details.  Unless \function{hbook.open} is invoked with
\code{purge_cycles=0} (or the file is opened read-only), \pyhep
``purges'' the entire file when the file is closed.  This removes cycles
except the most recent for each histogram and ntuple in the file.

\subsubsection{\class{File} objects}

A \class{File} object has the following methods and attributes:

\begin{memberdesc}{writeable}
 True if the file is open for writing as well as reading.
\end{memberdesc}

\begin{methoddesc}{listdir}{\optional{path}}
 Return the contents of the directory specified by \var{path}.  The
 return value is a sequence of \code{(id, type, title)} tuples, where
 \var{id} is the RZ ID of the entry; \var{type} is one of
 \code{"1d~histogram"}, \code{"2d~histogram"}, \code{"ntuple"}, or
 \code{"directory"}; and \var{title} is the title of the entry.
\end{methoddesc}

\begin{methoddesc}{mkdir}{path}
 Create a new directory at \var{path}.
\end{methoddesc}

\begin{methoddesc}{rmdir}{path}
 Remove the directory at \var{path}.
\end{methoddesc}

\begin{methoddesc}{save}{path, object\optional{, rz_id}}
 Store a histogram \var{object} at \var{path}.  One- and two-dimensional
 histograms are supported.  An unused RZ ID is chosen automatically,
 unless \var{rz_id} is specified or \var{object} has an \member{rz_id}
 attribute.  The function returns the RZ ID under which 'object' was
 saved. 
\end{methoddesc}

\begin{methoddesc}{load}{path}
 Load a histogram or connect to an ntuple at \var{path}.  If \var{path}
 is a one- or two-dimensional histogram, returns a histogram object with
 its contents.  If \var{path} is an ntuple, returns a table object (see
 \module{hep.table}) accessing it.
\end{methoddesc}

\begin{methoddesc}{createTable}{path, schema\optional{, rz_id, column_wise=1}}
 Create a table as an ntuple at \var{path} in the HBOOK file, using
 columns from \var{schema}, a \class{hep.table.Schema} instance.  If
 \var{rz_id} is not provided, an unused RZ ID is chosen automatically.
 The table is stored as a column-wise ntuple, unless a false value is
 passed for \var{column_wise}.  

 Column-wise ntuples support these column types: \constant{"int32"},
 \constant{"int64"}, \constant{"float32"}, and \constant{"float64"}.

 Row-wise ntuples support only \constant{"float32"} column types.
\end{methoddesc}

\begin{methoddesc}{remove}{path}
 Remove the ntuple or histogram at \var{path}.  Use \method{rmdir} to
 remove a directory.
\end{methoddesc}

\begin{methoddesc}{normpath}{path}
 Return the canonical form of absolute path \var{path} in the HBOOK
 file.
\end{methoddesc}

\begin{methoddesc}{join}{*components}
 Join together \var{components} to form an absolute path.
\end{methoddesc}

\begin{methoddesc}{split}{path}
 Return \code{(dir, name)}, where \var{dir} is the portion of \var{path}
 up to the last directory separator, and \var{name} is the remainder.
\end{methoddesc}

