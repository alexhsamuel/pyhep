\chapter{Plotting}

\pyhep includes classes and function for producing plots of histograms,
functions, and scatter plots.  The \class{hep.hist.plot.Plot} class is
the basic class used for producing plots.  An instance of \class{Plot}
is a figure object and can be used with renderers and layouts, as
described in the previous chapter.

Here's a basic example.  First, create a histogram:
\begin{verbatim}
>>> import hep.hist
>>> hist = hep.hist.makeSample1D("flat", 500)
\end{verbatim}
Now create a \class{Plot} object, which is a figure depicting the
histogram.
\begin{verbatim}
>>> from hep.hist.plot import Plot
>>> plot = Plot(1, hist)
\end{verbatim}
When creating the plot, you specified the number of dimensions
(independent variables) that are depicted in the plot, and the histogram
we want to plot.  Finally, create a window, and render the plot in
the window.
\begin{verbatim}
>>> from hep.draw.xwindow import Window
>>> window = Window((0.2, 0.1))
>>> window.render(plot)
\end{verbatim}
The argument to \class{Window} is the width and height of the window in
meters. 

By default, a bin of a one-dimensional histograms is drawn as crosses.
The horizontal line shows the bin value and bin width, and the vertical
line shows the range of errors on the bin content.  On the left and
right sides are displayed the underflow and overflow bins, labelled
``UF'' and ``OF'' respectively.

You can just as easily generate a PostScript file containing the plot.
Just use a PostScript renderer instead of the \code{Window} object.
\begin{verbatim}
>>> from hep.draw.postscript import PSFile
>>> ps_file = PSFile("plot.ps")
>>> ps_file.render(plot)
>>> del ps_file
\end{verbatim}
By deleting the \class{PSFile} object, you generate the output and close
the file. 

%-----------------------------------------------------------------------

\section{Plots and series}

A \class{hep.hist.plot.Plot} object is a figure that displays a plot of
histograms, functions, and similar data.  A plot can display either data
with one independent variable, such as one-dimensional histograms and
functions of one variable, or data with two independent variables, such
as two-dimensional histograms and scatter plots.  More than one
histogram, function, or scatter plot can be shown at the same time; each
is called a \textit{series}.  All series in the same plot share the same
vertical and horizontal axes (and z range, for two-dimensional plots).

To create a plot object, specify the number of dimensions in the plot,
either 1 or 2.  You may specify the series to plot as additional
arguments.  For example, to create a plot overlaying a histogram and a
function,
\begin{verbatim}
>>> from hep.hist.plot import Plot
>>> plot = Plot(1, histogram, function)
\end{verbatim}
You can also use the \method{append} method to add a series to a plot,
so the following code is equivalent:
\begin{verbatim}
>>> plot = Plot(1)
>>> plot.append(histogram)
>>> plot.append(function)
\end{verbatim}

A \class{Plot} object has a \member{series} attribute, which is a list
of series in the plot.  The series are not the histograms and functions
themselves, but figure objects representing the series.  (Generally, you
wouldn't to render these figures directly.)  
\begin{verbatim}
>>> plot.series[0]
<hep.hist.plot.Histogram1DPlot object at 0xf6d01fcc>
\end{verbatim}
The plot object for a histogram references the histogram itself in its
\member{histogram} attribute.
\begin{verbatim}
>>> plot.series[0].histogram
Histogram(EvenlyBinnedAxis(20, (0.0, 1.0)), bin_type=int, error_model='poisson')
\end{verbatim}

You can remove series from a plot simply by removing the corresponding
element from the \member{series} list.  Similarly, you can reorder
elements in the list to change the stacking order of series in the
plot. 

%-----------------------------------------------------------------------

\section{Plot styles}

\textit{Styles} control the visual attributes of plots.  Style
information is not stored in a histogram, which only contains
statistical data and annotations (such as units).  Instead, style is
stored in a plot.

Each plot has a \member{style} attribute, which is a dictionary
containing style items.  Keys in the style dictionary are names of style
attributes.  Any style attributes may be stored in the style dictionary;
different plot types use different style attributes to control their
output.

A \class{Plot} object has a \member{style} attribute; in addition, the
plot object representing each series in a plot has a \member{style}
attribute as well.  This allows different styles to be used for each
series in a plot (for instance, to assign a different color to each
series).  If a particular style attribute is missing from a series's
style dictionary, it uses the value from the parent \class{Plot}
object's style dictionary.  Thus, if you set a style attribute in a
\class{Plot}, it will apply to each series in that plot, except for a
series which overrides it in its own style dictionary.

Suppose you have a plot showing two histograms.
\begin{verbatim}
>>> plot = Plot(1, histogram1, histogram2)
\end{verbatim}
To draw 2 mm dots at bin contents for both series in the plot, set the
\code{"marker"} and \code{"marker size"} attributes in the plot's
style dictionary.
\begin{verbatim}
>>> plot.style["marker"] = "filled dot"
>>> plot.style["marker_size"] = 0.002
\end{verbatim}
To draw the second series (but not the first) in red, set the
\code{"color"} attribute in that series's style dictionary.
\begin{verbatim}
>>> plot.series[1].style["color"] = hep.draw.Color(0.7, 0, 0)
\end{verbatim}

You can set style attributes for a \class{Plot} when you create it, by
adding keyword arguments.  Similarly, you can set style attributes for a
series by adding keyword arguments to \method{append}.  For example,
this code produces the same plot as the code above does:
\begin{verbatim}
>>> plot = Plot(1, marker="filled dot", marker_size=0.002)
>>> plot.append(histogram1)
>>> plot.append(histogram2, color=Color(0.7, 0, 0))
\end{verbatim}

The sections below list style attributes used by \pyhep's plot classes
to determine the visual style of plots.  For an explanation of how to
specify fonts, colors, marker styles, \textit{etc.}, see the chapter on
drawing. 

\subsection{Styles for plot objects}

These style attributes control the visual style of \code{Plot} objects.

\begin{itemize}
 \item \code{"bottom_margin"}: The size of the margin on the bottom of
 the plot.

 \item \code{"color"}: Specifies the color for the entire plot,
 including axes and labels.

 \item \code{"font"}: Specifies the font to use for titles and labels
 on the plot.

 \item \code{"font_size"}: Specifies the font size to use for titles and
 labels on the plot.

 \item \code{"left_margin"}: The size of the margin on the left side of
 the plot.

 \item \code{"log_scale"}: If true, a logarithmic scale is used for the
 dependent axis (the y axis in one-dimensional plots, the z range in
 two-dimensional plots).

 \item \code{"normalize_bin_size"}: How to normalize bin contents to a
 common bin size.  The contents of bins of different sizes are
 normalized to a common effective bin width (for one-dimensional
 histograms) or bin area (for two-dimensional histograms) to facilitate
 the interpretation of bin contents as an approximation of probability
 density.  The contents of overflow bins are never normalized.

 If the value of this style attribute is a number, the content of each
 bin is normalized to this bin size.  If the value is \code{"auto"} (the
 default), a bin size is automatically chosen (for a histogram with
 evenly-binned axes, the histogram's actual bin size is used).  If the
 value is \code{None}, bins are not normalized, and the actual values of
 bin contents are plotted.

 \item \code{"overflows"}: If true, show underflow and overflow bins for
 the x axis (for one-dimensional plots) or the x and y axis (for
 two-dimensional plots).  For two-dimensional plots, underflow and
 overflow bins can be enabled for the x and y axes individually with the
 \code{"x_axis_overflows"} and \code{"y_axis_overflows"} style
 attributes.

 \item \code{"overflow_line"}, \code{"overflow_line_color"},
 \code{"overflow_line_dash"}, \code{"overflow_line_thickness"}: Whether
 to draw lines separating the overflow bins from the rest of the bin
 data, and the color, dash pattern, and thickness of the line.

 \item \code{"right_margin"}: The size of the margin on the right side
 of the plot.

 \item \code{"title"}: A title to draw on the plot.

 \item \code{"top_margin"}: The size of the margin on the top of the
 plot.

 \item \code{"x_axis_font"}, \code{"x_axis_font_size"}: The font and
 font size to use for labelling the x axis and its ticks.  Likewise for
 the y axis.

 \item \code{"x_axis_line"}, \code{"x_axis_color"}, 
 \code{"x_axis_thickness"}: Whether to draw the x axis, and its color
 and thickness.  Likewise for the y axis.

 \item \code{"x_axis_offset"}: The distance between the x axis and the
 edge of the bin data.  Likewise for the y axis.

 \item \code{"x_axis_position"}, \code{"y_axis_position"}: The position
 of the axes on the plot.  For the x axis, the value may be either
 \code{"bottom"} or \code{"top"}; for the y axis, either \code{"left"}
 or \code{"right"}.

 \item \code{"x_axis_range"}: The range of values to display along the x
 axis, as a \code{(lo, hi)} pair.  Likewise for the y axis.

 \item \code{"x_axis_ticks"}: Specifies the tick marks to draw along the
 x axis.  The value may be the approximate number of tick marks to use
 (chosen heuristally from the axis range); or a sequence of tick
 positions; or \code{None}.  Likewise for the y axis.

 \item \code{"x_axis_tick_size"}, \code{"x_axis_tick_thickness"}: The
 length and thickness of tick marks on the x axis.  Likewise for the y
 axis. 

 \item \code{"z_range"}: For two-dimensional plots, the range of values
 to display on the (virtual) z axis, as a \code{(lo, hi)} pair.

 \item \code{"zero_line"}, \code{"zero_line_color"},
 \code{"zero_line_thickness"}: In one-dimensional plots, whether to draw
 the zero line, and the zero line's color and thickness.

\end{itemize}

In addition, the style attributes that control the border, aspect ratio,
and size of layouts can be specified for plot objects as well.  See the
section on layout styles in the chapter on drawing.

\subsection{Styles for all plot series}

These style attribute control the visual styles of individual series in
plots.  If a series's style dictionary doesn't contain a particular
attribute, the value from the parent \code{Plot} object's style
dictionary is used.  Certain style attributes are used only for some bin
styles.

\begin{itemize}
 \item \code{"color"}: The color in which to draw this series.  

 \item \code{"dash"}: The dash pattern to use for lines.

 \item \code{"errors"}: If true, display bin errors for histograms that
 contain error information.  The representation of bin errors depends on
 the bin style.

 \item \code{"marker"}: The marker to use when drawing points at bin
 contents.  If \code{None}, no markers are drawn.

 \item \code{"marker_size"}: The marker size to use when drawing points
 at bin contents.

 \item \code{"thickness"}: The line thickness to use.

 \item \code{"suppress_zero_bins"}: If true, a bins with zero contents
 will not be drawn.

\end{itemize}


\subsection{Styles for 1D histograms plot series}

These style attributes are specific to plot series of one-dimensional
histograms.

\begin{itemize}
 \item \code{"bins"}: The style to use to draw bins.  These bin styles
 may be used:
 \begin{itemize}
  \item \code{"points"} (the default): Draws a marker and/or cross at
  the value of each bin.

  \item \code{"skyline"}: Draws the traditional outline or filled region
  representing bin contents.

 \end{itemize}

 \item \code{"bin_center"}: In the \code{"points"} bin style, the
 fractional horizontal location within the bin to draw the marker and
 the vertical error bar.  The value should be between zero and one.  The
 default is 0.5, which centers the marker and error bar in the bin.
 When superimposing two series with identical binning and similar bin
 contents, it is useful to offset the error bars of one or both to
 prevent them from overlapping.

 \item \code{"cross"}: In the \code{"points"} bin style, whether to
 draw a horizontal line in each bin.  The position of the line shows the
 bin contents, and the length of the line shows the bin width.

 \item \code{"error_hatch_color"}, \code{"error_hatch_pitch"},
 \code{"error_hatch_thickness"}: In the \code{"skyline"} bin style, bin
 errors are depicted with a 45-degree hatch pattern.  These style
 attributes control the color, spacing, and thickness of the hacth
 pattern.  If the color is \code{None}, the hatch pattern uses the fill
 pattern for errors above the bin contents, and the background color
 below the bin contents.

 \item \code{"fill_color"}: In the \code{"skyline"} bin style, the color
 with which to fill the bins.  If \code{None}, the bins are not filled.

 \item \code{"line_color"}: In the \code{"skyline"} bin style, the color
 for the outline of the bin contetns.  If \code{None}, no outline is
 drawn.

\end{itemize}


\subsection{Styles for 2D histograms plot series}

These style attributes are specific to plot series of two-dimensional
histograms.

\begin{itemize}
 \item \code{"bins"}: The style to use to draw bins.  These bin styles
 may be used:
 \begin{itemize}
  \item \code{"box"}: Draws a box for each bin with area proportional to
  the value of the bin content.

  \item \code{"density"}: Shades each bin with a color reprenting the
  bin content.
 \end{itemize}

 \item \code{"negative_color"}: If a color is specified, an empty bin
 represents zero bin contents, and this color is used to draw the
 contents of negative bins.  If this style attribute is \code{None}, an
 empty bin represents the low end of the z range, and all values are
 represented relative to this value.

 \item \code{"overrun_color"}, \code{"underrun_color"}: Colors to use to
 draw bins whose values are outside the z range of the plot.
\end{itemize}


\subsection{Styles for function plot series}

These style attributes are specific to plot series of one-dimensional
functions.  

\begin{itemize}
 \item \code{"bins"}: The style to use to draw bins.  These bin styles
 may be used:
 \begin{itemize}
  \item \code{"curve"}: Draw the function value as a curve.
 \end{itemize}

 \item \code{"number_of_samples"}: The number of points at which to
 sample the function.

\end{itemize}

%-----------------------------------------------------------------------

\section{Annotations and decorations}

Using the class \class{hep.hist.plot.Annotation}, you can add textual
annotations to a plot.  The annotation contains one or more lines of
text.  You may specify the text when constructing the object, or
incrementally using the \method{append} method or left-shift operator.
If the text contains newline characters, it is split into multiple
lines.

The annotations are drawn left-justified in the upper-left corner of the
plot, or right-justified in the upper-right corner of the plot.  Specify
the position with the \code{"position"} style, which may be either
\code{"left"} or \code{"right"}.  The styles \code{"annotation_font"},
\code{"annotation_font_size"}, \code{"annotation_color"}, and
\code{"annotation_leading"} control how the text is drawn.

Simply add the \class{Annotation} object as a series to the plot.

For example, this code makes a plot of a histogram with an annotation
describing its bin and axis type
\begin{verbatim}
>>> plot = hep.hist.plot.Plot(1, histogram)
>>> annotation = hep.hist.plot.Annotation("title: %s" % histogram.title)
>>> annotation << "bin type: %s" % histogram.bin_type.__name__
>>> annotation << "axis type: %s" % histogram.axis.type.__name__
>>> plot.append(annotation)
\end{verbatim}

The class \class{hep.hist.plot.Statistics} is a subclass of
\class{Annotation} which annotates histogram statistics.  To create one,
provide a sequence of statistic names to include, and one or more
histograms.  Statistic names can be \code{"sum"}, \code{"mean"},
\code{"variance"}, \code{"sd"}, and \code{"overflows"}.  Then simply add
the statistics annotation object as a series to the plot.

For example, to plot a histogram and display its sum (integral), mean,
and standard deviation,
\begin{verbatim}
>>> plot = hep.hist.plot.Plot(1, histogram)
>>> plot.append(hep.hist.plot.Statistics(("sum", "mean", "sd"), histogram))
\end{verbatim}

If you want to include the statistics with other annotations, you can
generate the text of the statistics annotation with the function
\function{hep.hist.plot.formatStatistics}.

%-----------------------------------------------------------------------

\section{Prepackaged plot functions}
