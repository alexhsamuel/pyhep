\section{\module{hep.hist} --- Histograms}

\declaremodule{extension}{hep.hist}
\modulesynopsis{Histograms.}

This module provides a way of collecting and manipulating histograms.  A
histogram is used to measure a statistical distribution by collecting
data sampled from this distribution.  This module supports binned
histograms, in which the data are divided into rectangular bins spanning
the range of possible values.  The values themselves may be one- or
higher-dimensional, and the value's component in each dimension may be
of a different numerical type.

A \class{Histogram} object represents a histogram of a distribution of
one or more dimensions.  Each dimension is represented by an
\class{Axis} object.  The \class{Axis} specifies the minimum and maximum
value of each sample point in that dimensions, and divides the range
into evenly-sized bind.

\begin{funcdesc}{Axis}{num_bins, min, max\optional{, axis_type,
 **kw_args}}
 Create an object representing an evenly-binned dimension in sample
 space with values greater than or equal to \var{min} and less than
 \var{max}.  This range of values is divided into \var{num_bins} bins of
 even size.  Values in this dimension are specified by the numerical
 type \var{axis_type}, which is inferred from \var{min} and \var{max} if
 omitted.  If \var{axis_type} is integral, the number of bins must
 divide evenly into \var{max} - \var{min}.

 Any additional \var{**kw_args} are added to the \class{Axis} object as
 attributes.
\end{funcdesc}

A histogram also records the number of samples which fall above or below
the specifies axis range in each dimension.  A value below the minimum
is stored as an ``underflow'' value along that axis, and a value equal
to or above the maximum is stored as an ``overflow''.

For each bin, a histogram stores the accumulated weight of samples whose
sample values fall in the bin.  Often, samples are accumulated with unit
weight, so that the content of a bin is the number of samples whose
values fall in the bin.  

The histogram also estimates the error on each bin.  The error is
computed as the square root of the sum of the squares of weights of
sample whose sample values fall in the bin.  If samples are accumulated
with unit weight, this is equal to the square root of the number of
samples whose values fall in the bin.  Note that this estimate of the
error is not always statistically valid; use your judgement when
interpreting this error value.

To create a histogram, use

\begin{funcdesc}{Histogram}{*axes\optional{, value_type, **kw_args}}
 Create a histogram with evenly-binned axes that implements the
 histogram protocol.  

 The number of dimensions of the histogram is equal to the number of
 \var{axes} arguments.  Each argument may be either
 \begin{itemize}
  \item An \class{Axis} instance.

  \item A sequence of the form \code{(num_bins, min, max\optional{,
  axis_type, name, units})}.  If specified, \var{axis_type} is the type
  used for bin contents; if omitted, the type for that axis is inferred
  from the values of \var{min} and \var{max}.  If specified, \var{name}
  and \var{units} are set as attributes of the same name on the
  \class{Axis} object.
 \end{itemize}

 The keyword argument \var{bin_type} specifies the numeical type used to
 store the value in each bin.  \emph{If omitted, the default is
 \code{int}}.  If any additional \var{**kw_args} are provided, they are
 set as attributes to the returned histogram object.
\end{funcdesc}

Histograms of a one-dimenstional independent variable are commonly
used.  A subclass of \class{Histogram} for one-dimensional histograms,
\class{Histogram1D}, provides a more convenient interface and more
efficient implementation.

\begin{funcdesc}{Histogram1D}{num_bins, min, max\optional{, axis_type,
name, units, **kw_args}} 
 Arguments are as for \function{Histogram}.  As with
 \function{Histogram}, the numerical type used to represent the contents
 of a bin is \code{int} unless the \var{bin_type} keyword argument is
 provided.
\end{funcdesc}

For example, this call creates a two-dimensional histogram.  The first
dimension, named ``energy,'' takes \code{float} values between 0 GeV and
4 GeV, divided into 40 bins.  The second dimension, named ``drift
chamber hits,'' takes integer values between 0 and 60, with one bin for
each possible value.  Bin contents are stored as integers.  The title of
this histogram and the units for bin contents are also specified.
\begin{verbatim}
histogram = Histogram(
    (40, 0, 4, float, "energy", "GeV"),
    (60, 0, 60, int, "drift chamber hits"),
    title="Fiducial track distribution",
    units="tracks")
\end{verbatim}

This example creates a one-dimenstional histogram over \code{float}
values between -1 and 1.  Bin contents are stored as \code{float}.  
The histogram title is also specified.
\begin{verbatim}
histogram = Histogram1D(20, -1.0, 1.0, bin_type=float, title="cos(theta)")
\end{verbatim}

%-----------------------------------------------------------------------

\subsection{The histogram protocol}

The histogram protocol is used for histograms with arbitrary binning
along one or more axes.  

A position in the histogram is represented by a sequence whose length is
the number of dimensions of the histogram.  Each element of the sequence
is the position along the corresponding axis.  The type of each element
must be the type of the corresponding axis, or coercible to it.  Such a
position sequence corresponds to exactly one bin in the histogram, which
may be an overflow or underflow bin if one or more elements fall outside
the ranges of their corresponding bins.

A bin may also be specified by a bin number, which is also a sequence
whose length is the number of dimensions of the histogram.  Each element
is a bin number between zero and one less than the number of bins of the
corresponding axis, or the strings \code{"underflow"} or
\code{"overflow"}.  For example, \code{(0, 4, "overflow")} is a bin
number for a three-dimensional histogram, specifying the first bin in
the first dimension, the fifth bin in the second dimension, and a value
larger than the axis range in the third dimension.

\begin{memberdesc}{dimensions}
 \readonly The number of dimensions.
\end{memberdesc}

\begin{memberdesc}{axes}
 \readonly A sequence of \class{Axis} instances describing the
 dimensions of the histogram's sample space.
\end{memberdesc}

\begin{memberdesc}{bin_type}
 \readonly The type used to for bin contents.
\end{memberdesc}

\begin{methoddesc}{accumulate}{value\optional{, weight=1}}
 Add \var{weight} to the contents of the bin at \var{value}; update the
 bin error accordingly.
\end{methoddesc}

\begin{methoddesc}{__lshift__}{value}
 A synonym for \method{accumulate}, with unit \var{weight}.
\end{methoddesc}

\begin{methoddesc}{getBin}{bin_number}
 Return the contents of bin \var{bin_number}.
\end{methoddesc}

\begin{methoddesc}{setBin}{bin_number, bin_contents}
 Set the contents of bin \var{bin_number} to \var{bin_contents}. 
\end{methoddesc}

\begin{methoddesc}{setError}{bin_number, bin_contents}
 Set the error on bin \var{bin_number} to \var{bin_contents}. 
\end{methoddesc}

\begin{methoddesc}{getError}{bin_number}
 Return the error on the bin \var{bin_number}.
\end{methoddesc}

\begin{methoddesc}{getBinRange}{bin_number}
 Return the range of values spanned by the bin \var{bin_number}.  The
 result a sequence of pairs, where each pair contains the minimum and
 maximum value for the bin along that axis dimension.
\end{methoddesc}

\begin{methoddesc}{map}{value}
 Return the bin number corresponding to \var{value}.
\end{methoddesc}

An \class{Axis} instance represents a one-dimensional axis of a
histogram.  Bins are numbered from zero to one less than the number of
bins; the bin number for an underflow value is \code{"underflow"}, and
the bin number for an overflow value is \code{"overflow"}.

\begin{memberdesc}{number_of_bins}
 \readonly The number of bins along this axis, not including underflow
 and overflow bins.
\end{memberdesc}

\begin{memberdesc}{range}
 \readonly A pair \code{(min, max)} of the range of values spanned by
 the bins of the histogram.  A value less than \var{min} is considered
 an underflow, and a value greater than or equal to \var{max} is
 considered an overflow.
\end{memberdesc}

\begin{memberdesc}{type}
 \readonly The type for values along this axis.
\end{memberdesc}

\begin{methoddesc}{__call__}{value}
 Return the bin number for a value along the axis.  
\end{methoddesc}

\begin{methoddesc}{getBinRange}{bin_number}
 Return the range \code{(min, max)} for a bin.
\end{methoddesc}

%-----------------------------------------------------------------------

\subsection{The simplified one-dimensional histogram protocol}

\class{Histogram1D} uses a simplified protocol to ease manipulation of
one-dimensional histograms.  It is identical to the general histogram
protocol, except that bin values and bin numbers are represented by
single values instead of sequences, and bin ranges are represented by
\var{(min, max)} pairs instead of sequences of such pairs.

\class{Histogram1D} is a subclass of \class{Histogram}, and provides the
following additional methods and attributes:

\begin{memberdesc}{axis}
 \readonly The sample axis; equivalent to \code{axes[0]}.
\end{memberdesc}

\begin{methoddesc}{__getitem__}{bin_number}
 Equivalent to \function{getBin}.
\end{methoddesc}

\begin{methoddesc}{__setitem__}{bin_number, bin_contents}
 Equivalent to \function{setBin}.
\end{methoddesc}

For example, this code creates and fills a one-dimensional histogram
with \class{Histogram}:
\begin{verbatim}
histogram = Histogram((10, 0.0, 1.0), bin_type=float)
for value in data_samples:
    histogram << (value, )

for bin_number in range(10):
    print "bin %d: %f" % (bin_number, histogram.getBinContent((bin_number, )))
\end{verbatim}
Since \class{Histogram} represents a histogram with a sample space in
arbitrary dimensions, the sample value and bin_number for the
one-dimensional case must be specified as a sequence with one element,
in this case \code{(value, )} and \code{(bin_number, )} respectively.

Using \class{Histogram1D}, this code is simplified:
\begin{verbatim}
histogram = Histogram1D(10, 0.0, 1.0, bin_type=float)
for value in data_samples:
    histogram << value

for bin_number in range(10):
    print "bin %d: %f" % (bin_number, histogram[bin_number])
\end{verbatim}

%-----------------------------------------------------------------------

\subsection{Projecting histograms}

To accumulate into multiple histograms from arbitrary functions of a
sequence of values, use the \function{hep.hist.project} function.

\begin{funcdesc}{project}{events, projections\optional{, weight}}
 Project multiple histograms from a collection of \var{events}.

 The \var{events} argument is a sequence or iterator.  Each item is a
 map containing variable values for that event: each key is the name of
 a variable, and the corresponding value is the event's value for that
 variable.  Each event in \var{events} should have the same keys.

 The \var{projections} argument is a sequence describing the histograms
 to project.  Each sequence element is of the form \code{(expression,
 histogram)}, where \var{expression} is an expression over the variables
 in the events; for each event, it is evaluated and the result is
 accumulated into \var{histogram}.  The expression may be a string, a
 callable, or an expression object (see the description of
 \module{hep.expr} elsewhere in this manual).  

 The \var{weight} argument is an expression over the event variables
 which yields the weight to use for that event.  For each event, the
 same weight value is used for accumulating into all histograms.  If
 \var{weight} is omitted, unit weight is assumed.

 The function returns the sum of weights (which is the number of events,
 if unit weight is used) projected into the histograms.
\end{funcdesc}

Note that \function{project} is designed to work well with tables:
simply pass a table iterator or a sequence of row objects as
\var{events}.  In this case, \function{project} determines that a table
row is in use, and will use special table features to perform the
projections efficiently.  To project a subset of rows in a table, use
the selection feature of table iterators.

%-----------------------------------------------------------------------
