\chapter{Drawing}

\pyhep includes a drawing layer that provides device-independent output
for simple two-dimensional figures and line drawings.  Classes and
functions for producing plots use the drawing layer; these are described
in the next chapter.  The drawing layer is in the module
\module{hep.draw}.

Drawings, plots, and the like are represented in \pyhep by
\textit{figures}.  A figure is a device-independent memory
representation of a drawing in a rectangular region, including all
visual attributes---the \textit{style}---of the drawing.  

A \textit{renderer} produces output by rendering a figure.  Each
renderer represents a particular output channel, such as a display
window or an output file.  The drawing layer supports output to an X11
window, and generation of PostScript files, enhanced Windows metafiles,
and bitmap files.

%-----------------------------------------------------------------------

\section{Units and types}

The unit of measurement in the \pyhep drawing layer is meters,
regardless of the output device; \pyhep attempts to preserve the scale
of output.  All measurements are given as 'float' values.

The module \module{hep.draw} includes the conversion factors
\code{point} and \code{inch} to help you use these units.  For instance,
the length \code{12 * point} is about 17 cm, or a sixth of an inch.

Colors are represented by the \class{Color} class.  Its constructor
arguments are red, green, and blue color components, each between zero
and one.  The function \function{Gray} produces a shade of gray; it
produces a color with the same value for all three components.  The
function \function{HSV} constructs a color from hue, saturation, and
value components.  For example,
\begin{verbatim}
>>> from hep.draw import *
>>> dark_red = Color(0.6, 0.1, 0.1)
>>> medium_gray = Gray(0.6)
>>> print medium_gray
Color(0.600000, 0.600000, 0.600000)
>>> aqua = HSV(0.5, 0.8, 0.5)
>>> print aqua
Color(0.100000, 0.500000, 0.500000)
\end{verbatim}
Also, \module{hep.draw} includes the constants \code{black} and
\code{white}.

The appearance of a line is given by its color, thickness, and dash
pattern.  Specify the thickness in meters as with any other length.  A
dash pattern is a tuple of lengths, that specifies the lengths of
alternating ``on'' and ``off'' segments.  For instance, this dash
pattern produces a line with alternating long (2 mm) and short (1 mm)
dashes, with a fixed small distance (0.5 mm) between the dashes.
\begin{verbatim}
>>> dash = (0.002, 0.0005, 0.001, 0.0005)
\end{verbatim}
For a solid line, specify \code{None} as the dash pattern.  You can also
use these constants for predefined dash patterns: \code{"solid"},
\code{"dot"}, \code{"dash"}, \code{"dot-dash"}, and
\code{"dot-dot-dash"}.

%-----------------------------------------------------------------------

\section{Canvases}

Not documented yet.

%-----------------------------------------------------------------------

\section{Renderers}

A renderer draws a figure to a display or to an output file.  Call the
\method{render} method, passing the figure to render.

\pyhep provides renderers for generating PostScript files, for enhanced
Windows metafiles, bitmap files, and for displaying in X windows.

\subsection{PostScript renderers}

The module \module{hep.draw.postscript} provides renderers for
PostScript files.  

An instance of class \class{PSFile} generates a multi-page ADSC
PostScript file.  Pass the constructor the path to the output file.  

You may specify the page size using the \code{page_size} keyword
argument.  The page size is either a \code{(width, height)} pair.  You
can also specify a page size by name, such as \code{"letter"} or
\code{"A4"}.  Named pages sizes are stored in the dictionary
\code{hep.draw.page_size}.  If the page size is omitted, letter is
assumed.

Each time you invoke the \method{render} method of a \class{PSFile}
object, you generate a new page in the PostScript.  When you are done,
simply delete the \class{PSFile} object to close the file.

This example shows the generatation of a two-page PostScript file.
\begin{verbatim}
>>> import hep.draw.postscript
>>> ps_file = hep.draw.postscript.PSFile("output.ps", page_size="letter")
>>> ps_file.render(figure1)
>>> ps_file.render(figure2)
>>> del ps_file
\end{verbatim}

The class \class{EPSFile} generates an encapsulated PostScript file.
Specify the bounding box size of the image using the \code{size} keyword
argument.  You should only invoke the \method{render} method
once per \class{EPSFile} object.

\subsection{Enhanced metafile renderer}

The module \module{hep.draw.metafile} provides a render for files in the
enhanced Windows metafile (EMF) format.  This is a vector file format
widely supported by Windows applications.  An EMF file essentially
contains a recording of the Windows graphics system calls required to
draw a figure.  The standard file extension for EMF files is
``\code{.emf}''.

The \class{EnhancedMetafile} class renders figures into EMF files.  As
with \class{PSFile}, specify the path to the output file and the image
size when creating an \class{EnhancedMetafile} object.  The
\method{render} method should only be invoked once.

This is how you can quickly save a figure as an EMF file.
\begin{verbatim}
>>> from hep.draw.metafile import EnhancedMetafile
>>> EnhancedMetafile("figure.emf", (0.06, 0.04)).render(figure)
\end{verbatim}

\subsection{Bitmap file renderer}

The module \module{hep.draw.imagefile} provides a function
\function{render} to render a figure as a bitmap image file.  The bitmap
is written using the Imlib library, and all bitmap file types supported
by Imlib can be used.  Depending on how Imlib is installed on your
system, different image file formats will be supported, but at least
PNG, JPG, and TIFF are generally available.  PNG is the recommended
format for line art such as plots of histograms.

Call \function{render} specifying the figure to render and save, the image
file name, and the size of the image in pixels.  The file format is
determined automatically from the file extension.  For example,
\begin{verbatim}
>>> from hep.draw import imagefile
>>> imagefile.render(figure, "figure.png", (640, 480))
\end{verbatim}
You may also specify the virtual size in which to render the figure,
using the \code{virtual_size} argument.

If you specify only the size or only the virtual size, the image other
is computed using the resolution specified by the \code{resolution}
argument, which defaults to 75 dpi.  For example, if you want the image
file to be 5 cm square, assuping 96 dpi,
\begin{verbatim}
>>> imagefile.render(figure, "figure.dpi", virtual_size=(0.05, 0.05),
...                resolution=96/inch)
\end{verbatim}

\subsection{X window renderer}

The class \class{hep.draw.xwindow.Window} is a window that can render
figures in an X window.  Each instance creates a new window.  

When you create a new \class{Window} instance, provide the desired
window size \textit{in meters} when creating a new window.  You can
resize an existing window with its \method{resize} method, or simply by
resizing the window through your window manager.  When creating a
\class{Window}, you may also provide a title for the title bar provided
by your window manager, or set this later by assigning the
\member{title} attribute.

Each time you call the \method{render} method, the window's contents are
replaced by the specified figure.  Note that if the window's contents
are destroyed (for example, if you resize the window), you will have to
call \method{render} again to restore the window's contents.  Also, if
you modify the figure, the window contents do not reflect the changes
until you call \method{render} again.

\class{Window} uses anti-aliasing and subpixel interpolation when
rendering its contents.

The \class{Window} class renders figures in memory and then transfers
the rendered image to your display.  This allows the renderer to use
advanced anti-aliasing and subpixel rendering, and makes sure fonts are
handled uniformly regardless of the configuration of the X server
displaying the window.  However, this means that when you run \pyhep in
a remote X session, the contents of the window are transferred over the
network as a bitmap, which can be slow.  If you tunnel your X connection
through SSH, you will probably find that using compression
(``\code{-C}'' on the \code{ssh} command line) improves display
performance.

\subsection{Figure windows}

A \class{hep.draw.xwindow.FigureWindow} provides a more convenient way
of displaying a figure in a window.  The \class{FigureWindow} knows the
figure it is rendering, and redraws its contents when necessary (when
the window is exposed or resized).

Using a \class{FigureWindow}, it's easy to display a figure.  The second
argument is the window size.
\begin{verbatim}
>>> from hep.draw.xwindow import FigureWindow
>>> window = FigureWindow(figure, (0.16, 0.10))
\end{verbatim}

You can change the figure displayed in the window by setting the
\member{figure} attribute.  If the value is \code{None}, nothing is
drawn in the window.  If the figure changes, call the \method{redraw}
method to force the window to redraw it.

%-----------------------------------------------------------------------

\section{Layouts}

A \textit{layout} is a composite figure that arranges several other
figures.  A layout object is itself a figure, so it can be rendered
directly by renderers, and included in other layouts.  The layout
objects described here are in the \module{hep.draw} module.

A simple layout class is \class{SplitLayout}.  It arranges two figures
next to each other, either horizontally or vertically.  A
\class{SplitLayout} allows you to specify the fractions of the entire
drawing region in which to draw each of the two figures.  When you
create a split layout, specify the orientation of the split, either
\code{"vertical"} or \code{"horizontal"}, and the two figures.  Either
figure may be \code{None}.  You may also specify the fraction of the
first figure (the default is 0.5); the remainder is used to draw the
second figure.

For example, this code creates a layout displaying \code{fig1} on the
left, occupying three-quarters of the layout, and \code{fig2} on the
right, occypting the remaining quarter.
\begin{verbatim}
>>> from hep.draw import *
>>> layout = SplitLayout("vertical", fig1, fig2, fraction=0.75)
\end{verbatim}

You can create more flexible layouts with a \class{BrickLayout} object.
A \textit{brick layout} resembles a row of bricks: the region is divided
into equally-spaced rows, each of which is divided into equally-spaced
cells.  Each row may have a different number of cells.  To specify the
arrangement of cells, provide a sequence containing the number of cells
in each row.  The number of elements in the sequence is the number of
rows.  For instance, to create a layout with three rows, of which the
first and third are divided in half and the middle is divided into
thirds,
\begin{verbatim}
>>> layout = BrickLayout((2, 3, 2))
\end{verbatim}
You may also specify style attributes as keyword arguments; as with
other figures, a layout's style attributes are stored in a dictionary
attribute \member{style}.  For example, the \code{margin} style
attribute controls the size of an empty margin inserted between rows and
cells in a row.
\begin{verbatim}
>>> layout = BrickLayout((2, 3, 2), margin=12*point)
\end{verbatim}

You can index the figures in a brick layout by column and row index, for
instance 
\begin{verbatim}
>>> layout[0, 0] = fig1
>>> layout[1, 2] = fig2
\end{verbatim}
Any entry that contains \code{None} (the default) is left empty.

A \class{GridLayout} is a subclass of \class{BrickLayout} in which all
rows have the same number of cells.  This produces a grid of
evenly-sized cells.  Specify the number of columns and rows when
creating an instance.
\begin{verbatim}
>>> layout = GridLayout(2, 3)
\end{verbatim}

%% All layout classes also have an attribute \member{figures}, which is a
%% list of figures displayed in the layout.  The order of figures depends
%% on the layout.  Also, the \member{appender} method provides convenient
%% way to set figures sequentially in a complex layout.  It produces a
%% function that, when called repeated, sets sequential positions of
%% \member{figures} to its argument.  Each \member{appender} filles the
%% figures starting from index zero, so you should hang on to it and call
%% the same one repeatedly.  For example,
%% \begin{verbatim}
%% >>> append_figure = layout.appender
%% >>> append_figure(fig1)
%% >>> append_figure(fig2)
%% \end{verbatim}
