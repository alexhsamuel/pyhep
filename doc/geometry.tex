\chapter{Lorentz geometry and kinematics}

\pyhep provides an implementation of Lorentz vectors, momenta, and
transformations.  The implementation classes do not simply provide
quadruplets of numbers equipped with a Minkowski inner product; instead,
they represent coordinate-independent objects.  Ultimately, though, the
coordinates of a vector must be specified and obtained; this is done
using a \emph{reference frame}, which specifies the coordinate system to
use. 

Coordinates are given in the order $(t, x, y, z)$, the metric signature
$(+,-,-,-)$ is employed, and the speed of light $c$ is assumed to be
unity.

\section{Reference frames}

A four-vector is a geometric object, which can be used to represent, for
instance, the space-time position of an event, or the energy-momentum of
a particle.  The typical representation of a four-vector is a quadruplet
of four coordinates, but the coordinate values for a particular
four-vector depends on the basis used, or equivalently on the reference
frame in which the coordinates are specified.

The Python class \class{hep.lorentz.Frame} represents a reference
frame.  The principle of relativity implies that there is no absolute
way of specifying a frame; the frame may only be specified in relation
to another.  In \pyhep, a frame is specified in relation to a special
frame, the canonical \emph{lab frame}.  The lab frame is
\member{hep.lorentz.lab}, an instance of \class{Frame}.  

\section{Vectors}

A four-vector is represented by an instance of
\class{hep.lorentz.Vector} class.  A \class{Vector} instance represents
the geometric object, which is independent of reference frame, so you
can specify or obtain its coordinate values only in reference to a
frame.  You may use the lab frame for this, or any other frame which you
create.

To create a vector by specifying its coordinates in a particular frame,
use the \method{Vector} method of that frame.  For instance,
\begin{verbatim}
from hep.lorentz import lab
vector = lab.Vector(5.0, 1.0, 0.0, -2.0)
\end{verbatim}
creates a four-vector whose time coordinate is 5.0 and whose space
coordinates are (1.0, 0.0, -2.0) in the lab frame.

To obtain the coordinates of a four-vector in a reference frame, use the
\method{coordinatesOf} method of that frame.  For instance,
\begin{verbatim}
t, x, y, z = lab.coordinatesOf(vector)
\end{verbatim}

You may negate (invert) a vector or scale it by a constant.  You may
also add or subtract two vectors.  Since these are geometric operations,
no frame is specified.  For example,
\begin{verbatim}
vector3 = - vector1 / 2 + 3 * vector2
\end{verbatim}
Use the \code{\^} operator to obtain the inner product of two vectors.
Each vector also has an attribute \member{norm}, its Lorentz-invariant
normal.  For example,
\begin{verbatim}
c = (vector1 ^ vector2) / (vector1.norm * vector2.norm)
\end{verbatim}

You may also use \class{hep.lorentz.Momentum}, a subclass of
\class{Vector} that represents a four-momentum.  It provided an
additional attribute \member{mass}, which is equivalent to
\member{norm}, plus an attribute \member{rest_frame}, which is a
\class{Frame} object representing the rest frame of a particle with that
four-momentum. 

\section{Transformations and frames}

A \class{hep.lorentz.Transformation} object represents a general Lorentz
transformation.  It can be used to transform either a geometric object,
such as a four-vector, or a reference frame.

Typically, a transformation is specified as a rotation or a boost.  A
rotation is specified by the Euler angles $\phi, \theta, \psi$ in a
particular reference frame.  A boost is specified by the vector
$\vec\beta$ in a particular reference frame.  The frame object's
\method{Rotation} and \method{Boost} methods, respectively, create these
transformations.   For example, 
\begin{verbatim}
from hep.lorentz import lab
from math import pi
rotation = lab.Rotation(pi / 4, pi / 4, 0)
boost = lab.Boost(0.0, 0.0, 0.5)
\end{verbatim}
The arguments to \method{Rotation} are the Euler angles, and the
arguments to \method{Boost} are the components of $\vec\beta$.

Transformations may be composed using the \code{\*} operation.  Be
careful about the frame in which you specify each one; generally, for
sequential transformations, you will want to apply the previous
transformation to your starting frame of reference before specifying the
next one.

You can apply a transformation to a four-vector using the \code{\^}
operator; this returns a different geometric four-vector.  

FIXME

A transformation can also be used to create a new reference frame.

