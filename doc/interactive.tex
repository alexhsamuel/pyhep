\chapter{Interactive PyHEP}

%-----------------------------------------------------------------------

This chapter describes how to use interactive \pyhep.  Interactive
\pyhep is simply the ordinary Python interpreter with certain \pyhep
modules preloaded, commonly-used names imported into the global
namespace, and some additional interactive functions and variables are
provided.


\section{Invoking interactive \pyhep}

The \code{pyhep} script launches interactive \pyhep.  This is installed
by default in \code{/usr/bin}; check your installation for your specific
location.  

In addition to the Python interpreter's usual startup message, the
\pyhep version is displayed.

\section{Imported names}

These names are imported into the global namespace:
\begin{itemize}

 \item The contents of the standard \module{math} module.

 \item The contents of the \module{hep.num} module.

 \item If your version of Python doesn't include built-in \code{bool},
 \code{True}, and \code{False}, these names are added.

 \item The \code{hep.lorentz.lab} reference frame object, as \code{lab}.

 \item The \code{default} particle data dictionary from
 \module{hep.pdt}, as \code{pdt}.

\end{itemize}


\section{Interactive functions and variables}

The names of interactive \pyhep functions always begin with `\code{i}',
and the names of variables always begin with `\code{i_}'.

The \function{ihelp} function displays some help information for
interactive \pyhep.


\subsection{Plotting functions}

Interactive \pyhep plots histograms in a pop-up plot window, similar to
PAW.  The plot window may be divided into rectangular ``zones'', each
displaying one plot.  All plots are shown in the same window, replacing
other plots where necessary.

\begin{funcdesc}{iplot}{histogram\optional{, **style}}
 Show a new plots of a histogram or scatter plot in the plot window.
 The argument is a histogram or \class{Scatter} object.  Any keyword
 arguments are used as style attributes for the new plot.
\end{funcdesc}

\begin{funcdesc}{iseries}{histogram\optional{, **style}}
 Add a series to the current plot in the plot window.  The argument is a
 histogram or \class{Scatter} object.  Any keyword arguments are used as
 style attributes for the new plot.
\end{funcdesc}

\begin{funcdesc}{igrid}{columns, rows}
 Divide the plot window into a grid of rectangular plot zones.  The
 arguments are the number of columns and rows in the grid.  Each call to
 \function{iplot} uses the next plot zone, proceeding left-to-right and
 then top-to-bottom.  Note that calling \function{igrid} always
 removes all plots and clears the plot window.
\end{funcdesc}

\begin{funcdesc}{iselect}{column, row}
 Select a plot zone.  Its arguments are the column and row coordinates
 of the zone.  The next plot is drawn in this zone. 
\end{funcdesc}

\begin{funcdesc}{ishow}{figure}
 Show \code{figure} in the plot window.  If you have divided the figure
 with \function{igrid}, shows \code{figure} in the current cell.
\end{funcdesc}

\begin{funcdesc}{iprint}{file_name}
 Print the contents of the plot window to a file.  The type of the file
 is inferred from the file name extension: ``.ps'' produces a PostScript
 file, and ``.eps'' produces an encapsulated PostScript file.
\end{funcdesc}

The global variable \code{ifig} always refers to the current \code{Plot}
object, the current figure (if you added one with \function{ishow}, or
\code{None} (if the current plot zone is empty).  The global variable
\code{i_plots} is a sequence of sequences of \code{Plot} objects for all
zones in the plot window.

The global variable \code{iwin} is a draw object for the plot window.

The global variable \code{istyle} is a style dictionary that is used for
plots and series created with \function{iplot} and \function{iseries}.
If you set a style attribute in \code{istyle}, it will become the
default for subsequent plots.  You can also call the \method{setall}
method of \code{istyle}, which sets a style attribute not only in this
dictionary, but in all of the plots currently shown in the plot window. 
For example, to set a log scale for the plots currently shown as well as
subsequent plots,
\begin{verbatim}
>>> istyle.setall("log_scale", True)
\end{verbatim}
To set a style attribute in the current plot only, access its
\member{style} member through \code{ifig}, like this:
\begin{verbatim}
>>> ifig.style["log_scale"] = True
\end{verbatim}
\pyhep keeps track of when a plot's style dictionary is changed, and
redraws the plot automatically.

You may modify any plot displayed in any plot zone.  Select the zone
with \function{iselect}, access and modify the plot with \code{i_plot}
(for instance, adjust its style or add a series), and then call
\function{iredraw} to redraw it.

\subsection{Table functions}

\begin{funcdesc}{iproject}{table, expression\optional{, selection,
      number_of_bins, range}}
 Project a histogram from a table.  The first argument is the table
 object, and the second argument is the expression to accumulate in the
 histogram.  The optional \code{selection} argument is a selection
 expression; only table rows for which this expression is true are
 projected into the histogram.  The number of bins and histogram range
 are automatically determined, but to override these, use the
 \code{number_of_bins} and \code{range} (a pair of values) arguments,
 respectively.  The return value is a histogram object.

 To project a one-dimensional histogram, use an ordinary expression.  To
 project a two-dimensional histogram, specify two expressions in
 \code{expression} separated by commas; for instance \code{"p_x, p_y"}.
 You can also use the \function{iproj1} function to project a
 one-dimensional histogram from a table.
\end{funcdesc}

\begin{funcdesc}{idump}{rows, *expressions}
 Dumps values from a table.  The first argument is a table or an
 iterator over table rows.  One or more additional arguments are
 expressions whose values are to be displayed.  The output is a table
 with the row number followed by the values of the specified
 expressions.
\end{funcdesc}

