\chapter{Histograms}

A histogram is a tool for measuring an \textit{N}-dimensional
statistical distribution by summarizing samples drawn from it.

The histogram divides a rectangular region of the \textit{N}-dimensional
samples space into rectangular sub-regions, called \textit{bins}.  Each
bin records the number of recorded samples whose values fall in that
sub-region; when a sample is added into the histogram, or
\textit{accumulated}, the bin's value is incremented.

The histogram designates an \textit{axis} for each dimension of the
sample space, which specifies the range of coordinate values in that
dimension which the histogram will accept.  Coordinate values that fall
below or above that range are called \textit{underflows} and
\textit{overflows}, respectively.

Each axis also specifies the edges of the bins in the corresponding
dimension, which forms rectangular bins.  For \textit{evenly-binned}
axes, the positions of the bin edges are evenly spaced along the range
of the axis.  For \textit{unevenly-binned} axes, the bin edges may be
positioned arbitrarily; this allows smaller bins to be used in regions
of the coordinate value where samples are more likely to fall more
densely.

In some cases, samples are not drawn from the distribution which is
being measured, but from a modified distribution (for example, when
importance sampling).  The modified distribution can be transformed back
to the original distribution by assigning each sample a \textit{weight}.
The value of the corresponding bin will then be increased by the sample
weight.  When no weight is specified, unit weight is assumed.

If the samples are statistically independent, the contents of the bins
will provide an approximation of the probability density function from
which the samples are drawn.  This approximation is subject to
statistical uncertainty or \textit{error}.  The error is drawn from a
Poisson distribution; for large statistics, a Gaussian distribution is a
good approximation.  

%-----------------------------------------------------------------------

\section{Introductory example}

This example creates a one-dimensional histogram with five evenly-spaced
bins between zero and ten, fills some sample values into it, and dumps
the resulting contents.

Import the histogram module.
\begin{verbatim}
>>> import hep.hist
\end{verbatim}
Create a histogram object.
\begin{verbatim}
>>> histogram = hep.hist.Histogram1D(5, (0, 10))
\end{verbatim}
Show the histogram object we just created.
\begin{verbatim}
>>> print histogram
Histogram(EvenlyBinnedAxis(5, (0, 10)), bin_type=int, error_model='poisson')
\end{verbatim}
Accumulate samples into the histogram.
\begin{verbatim}
>>> for sample in (0, 4, 5, 6, 7, 7, 9, 10, 11):
...   histogram << sample
...
\end{verbatim}
Dump the histogram contents.
\begin{verbatim}
>>> hep.hist.dump(histogram)
Histogram, 1 dimensions
int bins, 'poisson' error model
 
axes: EvenlyBinnedAxis(5, (0, 10))
 
      axis 0             bin value / error
---------------------------------------------
[   None,      0)         0 +  1.148 -      0
[      0,      2)         1 +  1.360 -  1.000
[      2,      4)         0 +  1.148 -      0
[      4,      6)         2 +  1.519 -  2.000
[      6,      8)         3 +  1.724 -  2.143
[      8,     10)         1 +  1.360 -  1.000
[     10,   None)         2 +  1.519 -  2.000
 
\end{verbatim}

%-----------------------------------------------------------------------

\section{Creating histograms}

The class \class{hep.hist.Histogram} provides an arbitrary-dimensional
histogram with binned axes.  Optionally, a histogram can store errors
(uncertainties) of the contents of each bin, or can compute these
automatically from the bin contents.

To construct a \class{Histogram}, provide a specification of each axis,
one for each dimension.  You can specify an \code{Axis} object for each
axis (described later), but the easiest way to specify an evenly-binned
axis is with a tuple containing the number of bins, and a '(lo, hi)'
pair specifying the low edge of the first bin and the high edge of the
last bin.  For example,
\begin{verbatim}
>>> import hep.hist
>>> one_d_histogram = hep.hist.Histogram((20, (0, 100)))
\end{verbatim}
creates a one-dimensional histogram with 20 bins between values 0
and 100.  Why all the parentheses?  The outermost pair are for the function
call; the next pair group parameters of the (single) axis; the innermost
group the low and high boundaries of the axis range.

Similarly,
\begin{verbatim}
>>> two_d_histogram = hep.hist.Histogram((20, (-1.0, 1.0)), (100, (0.0, 10.0)))
\end{verbatim}
creates a two-dimensional histograms with 20 bins on the first axis and
100 bins on the second axis.

To create a one-dimensional histogram you may use
\function{Histogram1D}, which is equivalent except the axis's
parameters are specified directly as arguments instead of wrapped in a
sequence, so that one pair of parentheses may be omitted.  Thus, the
first example above may be written,
\begin{verbatim}
>>> one_d_histogram = hep.hist.Histogram1D(20, (0, 100))
\end{verbatim}

You can access the sequence of axes for a histogram from its \code{axes}
attribute.  You'll see that the tuple arguments you specified have been
converted into \code{EvenlyBinnedAxis} objects.
\begin{verbatim}
>>> two_d_histogram.axes
(EvenlyBinnedAxis(20, (-1.0, 1.0)), EvenlyBinnedAxis(100, (0.0, 10.0)))
\end{verbatim}
For one-dimensional histograms, you may also use the \code{axis} attribute.
\begin{verbatim}
>>> one_d_histogram.axis
EvenlyBinnedAxis(20, (0, 100))
\end{verbatim}

For evenly-binned axes such as these, the \code{range} and
\code{number_of_bins} attributes contain the axis parameters.  The
\code{dimensions} attribute contains the number of dimensions of the 
histogram.  This is always equal to the length of the \code{axes}
attribute.
\begin{verbatim}
>>> one_d_histogram.axis.number_of_bins
20
>>> one_d_histogram.axis.range
(0, 100)
>>> one_d_histogram.dimensions
1
\end{verbatim}

If you specify additional keyword arguments when creating the histogram,
they are added as attributes to the new histogram object.  For example,
\begin{verbatim}
>>> histogram = hep.hist.Histogram(
...   (40, (0.0, 10.0), "momentum", "GeV/c"),
...   (32, (0, 32), "track hits"),
...   title="drift chamber tracks")
>>> histogram.title
'drift chamber tracks'
\end{verbatim}
This creates a two-dimensional histogram of 40 bins of track energy
between 0 and 10 GeV/c, and 32 integer bins counting track hits.  Note
that this histogram has $(40 + 2) \times (32 + 2) = 1428$ bins,
including overflow and underflow bins on each axis.  The histogram has
an additional attribute \member{title} whose value is
\code{"drift chamber tracks"}.  You can, of course, set or modify
additional attributes like \member{title} after creating the histogram.

\subsection{Axis type}

By specifying the range for each axis, you also implicitly specify the
numeric type for values along the axis.  (If the low and high values you
specify for the range are of different types, a common type is obtained
by Python's standard \code{coerce} mechanism.)  So, for instance, if
you specify two integer values for the low and high boundaries for the
axis, the axis will expect integer values, but if you specify one
integer and one floating-point value, the axis will expect
floating-point values.

You can find out the type of an axis by consulting its \code{type}
attribute.  For example,
\begin{verbatim}
>>> histogram = hep.hist.Histogram((20, (-1.0, 1.0)), (20, (-10, 10)))
>>> histogram.axes[0].type
<type 'float'>
>>> histogram.axes[1].type
<type 'int'>
\end{verbatim}

If the type for an axis is \code{int} or \code{long} (because you
specified integer or long values for the axis range), the difference
between the high and low ends of the range must be an even multiple of
the number of bins.  If this is not desired, use \code{float} values for
the axis.  For example, this construction raises an exception:
\begin{verbatim}
>>> histogram = hep.hist.Histogram1D(8, (-50, 50))
[... stack trace ...]
ValueError: number of bins must be a divisor of axis range
\end{verbatim}
because 50 - (-50) = 100 is not a multiple of 8.  However, any of these
will work:
\begin{verbatim}
>>> histogram = hep.hist.Histogram1D(10, (-50, 50))
>>> histogram = hep.hist.Histogram1D(8, (-60, 60))
>>> histogram = hep.hist.Histogram1D(8, (-50.0, 50.0))
\end{verbatim}

\subsection{Additional axis parameters}

In addition to the number of bins and axis range, you may optionally
provide additional elements to a tuple specifying an axis (or as the
arguments to \function{Histogram1D}).  The third argument is the string
name of the axis quantity, and the fourth argument is a string
describing the units of this quantity.  For example,
\begin{verbatim}
h = hep.hist.Histogram1D(30, (0.05, 0.20), "mass", "GeV/$c^2$")
\end{verbatim}
creates a histogram of 30 bins of mass between 0.05 and 0.2 GeV/$c^2$.

\LaTeX markup may be used in the axis name and units; these will be
rendered appropriately when plotting the histogram.  See the section on
\LaTeX markup for details of the syntax.

\subsection{Bin type and error model}

A histogram stores for each bin the total number of samples, or the sum
of weights of samples, that have fallen in this bin.  The keyword
argument \var{bin_type} specifies the numerical type used to store these
totals.  The histogram's \code{bin_type} attribute contains this type.
\begin{verbatim}
>>> histogram = hep.hist.Histogram1D(10, (-50, 50))
>>> histogram.bin_type
<type 'int'>
>>> histogram = hep.hist.Histogram1D(10, (-50, 50), bin_type=float)
>>> histogram.bin_type
<type 'float'>
\end{verbatim}

By default, bin contents are stored as \code{int} values.
\emph{Therefore, if you plan to use non-integer weights, you must
specify} \code{bin_type=float} \emph{when creating the histogram!}
Otherwise, weights will be truncated to integers when filling.  If your
weights are all less than one, you fill find your histogram to be empty.

A histogram can also estimate the statistical counting uncertainty on
each bin.  This error is represented by a 68.2\% confidence interval
around the bin value.  The interval is represented by a pair of values
specifying the sizes of the low and high ``error bars'',
\textit{e.g.} if the bin value is \code{20} and the errors are
\code{(5.5, 4.5)}, the 68.2\% confidence interval is the range (14.5,
24.5).

Several models are available are available that control how the errors
are stored or computed:
\begin{itemize}
 \item \code{"none"}: Each bin is assumed to have zero uncertainty.

 \item \code{"gaussian"}: The errors are computed assuming symmetrical
 Gaussian counting errors.  The bin value is interpreted as a number of
 counts, and the low and high errors are both the square root of the bin
 content.

 \item \code{"poisson"}: The errors are computed assuming Poisson
 counting errors.  The bin value is interpreted as a number of counts,
 and the low and high errors are chosen to cover 68.2\% of the Poisson
 cumulative distribution around the bin value.  The confidence interval
 is chosen to cover 34.1\% on either side of the central value where
 possible.  However, for a central value of zero, one, or two counts,
 this would produce a confidence interval with a lower edge below zero,
 so the lower edge is fixed at zero and the upper edge is chosen to
 capture 68.2\%.

 \item \code{"symmetric"}: For each bin, the histogram stores a single
 value representing both the lower and upper errors.  The error value
 may be specified explicitly for each bin with the \method{setBinError}
 method.  

 When a sample is accumulated into the histogram, the sample weight is
 added in quadrature to the bin error.  The error value on a bin (used
 as both the upper and lower error) are is given by $\sigma=\sqrt{\sum
 w_i^2}$ where $w_i$ are the sample weights accumulated into the bin.

 \item \code{"asymmetric"}: For each bin, the histogram stores two
 values representing the lower and upper errors.  The two error values
 may be specified explicitly for each bin with the \method{setBinError}
 method.  

 Errors are computed from weights as in the \code{"symmetric"} error
 model.  The only difference is that you may specify manually different
 values for the lower and upper errors using the \method{setBinError}
 method.
\end{itemize}

Specify the error model for a histogram with the \var{error_model}
keyword argument.  The default is \constant{"poisson"} if the bin type is
\code{int} or \code{long}, or \constant{"gaussian"} otherwise.  If you will
specify bin errors explicitly using the \method{setBinError} method, you
must specify the \constant{"symmetric"} or the \constant{"asymmetric"}
error model.  The \constant{error_model} attribute contains a histogram's
error model.

For example,
\begin{verbatim}
>>> histogram = hep.hist.Histogram1D(10, (-50, 50))
>>> histogram.error_model
'poisson'
>>> histogram = hep.hist.Histogram1D(10, (-50, 50), bin_type=float)
>>> histogram.error_model
'gaussian'
>>> histogram = hep.hist.Histogram1D(10, (-50, 50), error_model="symmetric")
>>> histogram.error_model
'symmetric'
\end{verbatim}

%-----------------------------------------------------------------------

\section{Other kinds of axes}

The histogram axes we've used up to now have been evenly binned,
\textit{i.e.} all the bins on each axis are the same size.  It is also
possible to create a histogram with unevenly-binned axes, using the
\class{hep.hist.UnevenlyBinnedAxis} class.  

Create an \class{UnevenlyBinnedAxis} instance, specifying a list of bin
edges as its argument.  You can then specify this axis when creating a
histogram.  Note that each histogram axis is independently specified, so
that one, several, or all may be unevenly binned.

For example, to create a 2-D histogram with an unevenly-binned x axis
and one evenly-binned y axis,
\begin{verbatim}
>>> import hep.hist.axis
>>> x_axis = hep.hist.UnevenlyBinnedAxis(
...     [0, 2, 4, 5, 6, 8, 10, 15, 20], name="count")
>>> histogram = hep.hist.Histogram(x_axis, 
...     (10, (0., 5.), "time", "sec"))
\end{verbatim}

%-----------------------------------------------------------------------

\section{Filling histograms}

To ``fill'' a histogram is to accumulate samples from the sample
distribution into it.  A sample is represented by a sequence of
coordinate values, where the length of the sequence is equal to the
number of dimensions of the histogram.   Each item in the sequence is
the coordinate value along the corresponding axis of the histogram.

For each bin, the histogram keeps track of the number of accumulated
samples whose coordinates fall within the bin.  Optionally, samples may
be specified a ``weight''; in that case, the histogram keeps track of
the sum of weights of samples.  The numeric type used to store the
number of samples or sum of weights for each bin is given by the
\code{bin_type} attribute.  The histogram also tracks the total
number of accumulations (the ``number of entries'') that you have made.

The \method{accumulate} method accumulates an event into the histogram.
Specify the coordinates of the sample, and optionally the event weight
(which otherwise is taken to be unity).  The left-shift operator
\code{<<} is shorthand for \method{accumulate} with unit weight.

If any of the coordinate values passed to \method{accumulate} is
\code{None}, the sample is not accumulated into the histogram and the number
of entries is not changed.

For example, to accumulate a the sample whose coordinates are \code{x}
and \code{y} into a two-dimensional histogram \code{histogram},
\begin{verbatim}
>>> histogram.accumulate((x, y))
\end{verbatim}
or
\begin{verbatim}
>>> histogram << (x, y)
\end{verbatim}
To accumulate the same sample with weight \code{weight},
\begin{verbatim}
>>> histogram.accumulate((x, y), weight)
\end{verbatim}

For one-dimensional histograms, you may specify the sample coordinate by
itself, instead of as a one-element sequence.  So, to accumulate the
sample with coordinate \code{x} into one-dimensional \code{histogram},
you may use any of these:
\begin{verbatim}
>>> histogram.accumulate((x, ))
>>> histogram.accumulate(x)
>>> histogram << (x, )
>>> histogram << x
\end{verbatim}

Plotting histograms is described later in the plotting chapter.  A quick
way of displaying histogram contents in text format is with the
\function{hep.hist.dump} function.

This script below creates a one-dimensional histogram with eleven
integer bins between 2 and 12, inclusive, and fills the histogram with
the result of simulating 1000 rolls of two dice.
\codesample{histogram1.py}

%-----------------------------------------------------------------------

\section{Accessing histogram contents}

You may specify a particular bin of a histogram with a ``bin number'',
which is a sequence of bin positions along successive axes.  The length
of the sequence is equal to the histogram's number of dimensions.  Each
item is the index of the bin along the corresponding axis.

Along each axis, the coordinate in the bin number ranges between zero
and one less than the number of bins on the axis.  It may also take the
values \code{"underflow"} and \code{"overflow"}, which denote the
underflow and overflow bins, respectively

For example, consider this histogram:
\begin{verbatim}
>>> histogram = hep.hist.Histogram((20, (-1.0, 1.0)), (24, (0, 24)))
\end{verbatim}
The corner bin numbers are for this histogram \code{(0,~0)},
\code{(9,~0)}, \code{(0,~23)}, and \code{(9,~23)}.  
The bin whose number is \code{(12,~"underflow")} is for any samples
whose first coordinate is between 0.2 and 0.3, and whose second
coordinate is less than 0.  The bin whose number is
\code{("underflow",~"overflow")} is for any sample whose first
coordinate is less than -1 and whose second coordinate is greater than
24.

To get the bin number corresponding to a sample point, use the
\method{map} method, passing the sample coordinates. 

Just as with sample coordinates, for a one-dimensional histogram you may
specify either the bin number as a one-element sequence, or simply the
bin number along the (only) axis.

To obtain the contents of a bin, use the \method{getBinContent} method,
passing the bin number.  To obtain the 68.2\% confidence interval on a
bin, use the \method{getBinError} method, which returns two values
specifying how far the interval extends below and above the central
value.

For example,
\begin{verbatim}
>>> histogram = hep.hist.Histogram1D(10, (0.0, 1.0), error_model="gaussian")
>>> histogram.accumulate(0.64, 17)
>>> histogram.map(0.64)
(6,)
>>> histogram.getBinContent((6, ))
17
>>> histogram.getBinError((6, ))
(4.1231056256176606, 4.1231056256176606)
\end{verbatim}
In the \code{"gaussian"} error model, the errors on the bin are the
square root of the bin contents, here, $\sqrt{17}$=4.123106.  Since the
histogram is one-dimensional, we just as easily could have used,
\begin{verbatim}
>>> histogram.getBinContent(6)
\end{verbatim}

To set the contents of a bin, use the \method{setBinContent} method,
specifying the new value as the second argument.  To set the error
estimate on a bin, use the \method{setBinError} method and specify a
pair for the sizes of lower and upper errors.  Note that you may only
call the \method{setBinError} method of a histogram with
\code{"asymmetric"} or \code{"symmetric"} error model, and in the latter
case, the average of the lower and upper error values you specify is
used as the single symmetric error estimate.

To obtain the range of coordinate values spanned by a single bin, use
the \method{getBinRange} method, passing the bin number.  The return
value is a sequence, each of whose items is a \code{(lo, hi)} pair of
coordinate values along on axis spanned by the bin.  For example, to
print the bin range and value for bins in a one-dimensional histogram,
\begin{verbatim}
>>> for bin in range(histogram.axis.number_of_bins):
...     (lo, hi), = histogram.getBinRange(bin)
...     content = histogram.getBinContent(bin)
...     print "bin (%f,%f): %f" % (lo, hi, content)
\end{verbatim}
Note that the return value from \method{getBinRange} is a one-element
sequence, since the histogram is one-dimensional.  The single element is
the pair \code{(lo, hi)} of the bin's range along the histogram's axis.

A histogram's \member{number_of_samples} attribute contains the number
of times the \method{accumulate} method (or the \code{<<} operator) was
invoked.

%-----------------------------------------------------------------------

\section{More histogram operations}

The following functions are provided in \module{hep.hist}.  Invoke
\code{help(hep.hist.}\textit{function}\code{)} for a description of a
function's parameters.

\begin{itemize}
 \item The function \function{scale} produces a copy of a histogram with
 its contents scaled by a constant factor.  

 \item The function \function{integrate} returns returns the sum over
 all bins of a histogram.  Specify \code{overflows=True} to include
 underflow and overflow bins in the integral.  

 \item The function \function{normalize} returns a copy of a histogram,
 scaled such that its integral a fixed integral value (by default, one).

 \item The function \function{normalizeSlices} returns a copy of a
 histogram in which slices along a specified axis are normalized
 independently to a fixed integral value (by default, one).

 \item The functions \function{add} and \function{divide} create a new
 histogram by adding or dividing, respectively, the corresponding bins
 in two histograms.  The histograms must have the same axis ranges and
 binning.

 \item The function \function{getMoment} computes the Nth moments of a
 histogram in each of its dimensions.

 \item The \function{mean} and \function{variance} functions compute
 those two statistics. 

 \item The function \function{slice} produces an (N-1)-dimensional
 histogram by slicing or projecting out one dimension of an
 N-dimensional histogram.  

 \item The function \function{rebin} produces a copy of a histogram with
 groups of adjacent bins combined together.

 \item The function \function{transform} transforms a histogram axis by
 an arbitrary monotonically increasing function by creating a copy of
 the histogram with adjusted bin boundaries.

 \item The function \function{getRange} determines the range of bin
 values in a histogram.  Optionally, the range can accommodate the error
 intervals on each bin.

 \item The function \function{dump} prints the contents of a histogram
 to standard output or another file.

 \item The function \function{project} accumulates samples
 simultaneously into several histograms from an array of sample events.
 This function is described later in the chapter on tables.

\end{itemize}

You may also add or divide two histogram with the ordinary addition and
division operators, respectively, and you may scale a histogram with the
ordinary multiplication operator.  For example,
\begin{verbatim}
>>> combined_histogram = 3 * histogram1 + histogram2
\end{verbatim}

To iterate over all bins in a histogram, use these functions.  Each
takes an optional second argument \code{overflow}; if true, underflow
and underflow bins are included (by default false).
\begin{itemize}
  \item \function{AxisIterator} takes an \code{Axis}.  It returns an
  iterator that yields the bin numbers for that axis.

  \item \function{AxesIterator} takes a sequence of \code{Axis} objects,
  such as the value of a histogram's \member{axes} attribute.  It return
  s an iterator that yields the bin numbers for the multidimensional bin
  space specified by the axes.

  \item \function{BinValueIterator} takes a histogram.  It returns an
  iterator that yields the contents of each bin in the histogram.

  \item \function{BinErrorIterator} takes a histogram.  It returns an
  iterator that yields the error estimate on each bin in the histogram.

  \item \function{BinIterator} takes a histogram.  It returns an
  iterator that yields triplets \code{(bin_number, contents, error)} for
  each bin in the histogram.

\end{itemize}

For example, the code below shows how you might find the largest bin
value in a histogram, including underflow and overflow bins.  (You could
also use the \function{getRange} function.)
\begin{verbatim}
>>> max(hep.hist.BinValueIterator(histogram, overflow=True))
\end{verbatim}

%-----------------------------------------------------------------------

\section{Scatter plots}

A \class{hep.hist.Scatter} object collects sample points from a
bivariate distribution.  The samples are typically displayed as a
``scatter plot''.

The sample points are not binned in any way; rather, the two coordinates
of each samples is stored.  Unlike with a histogram, the memory usage of
a \class{Scatter} object increases as additional samples are
accumulated.  However, the resulting object contains the full covariance
information for the two coordinate variables, and can be plotted
precisely.

To create a \class{Scatter}, specify information about the two axes.
Each may be an instance of \class{hep.hist.Axis}, or more simply a
sequence '(type, name, units)', where 'type' is the Python type of
coordinate values along this axis, and 'name' and 'units' are strings
describing the coordinate values (which may be omitted).  If no axes are
specified, they are taken to have 'float' coordinates

For example, this code creates a \class{Scatter} object.
\begin{verbatim}
>>> scatter = hep.hist.Scatter(
...     (float, "energy", "GeV"), (int, "number of hits"),
...     title="candidate tracks")
\end{verbatim}

Use the \method{accumulate} method to add a sample to the
\class{Scatter}.  The argument must be a two-element sequence containing
the two coordinates.  The left-shift operator \code{<<} is a synonym for
\method{accumulate}.  The sample values are stored in a sequence
attribute \member{points}.

For example,
\begin{verbatim}
>>> for track in tracks:
...     scatter << (track.energy, track.number_of_hits)
... 
\end{verbatim}

