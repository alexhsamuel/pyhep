\documentclass{manual}

\usepackage{xspace}

%-----------------------------------------------------------------------

\newcommand{\codesample}[1]{\verbatiminput{samples/#1}}
\newcommand{\evtgen}{EvtGen\xspace}
\newcommand{\pyhep}{PyHEP\xspace}
\newcommand{\readonly}{(read-only)}

%-----------------------------------------------------------------------

\title{\pyhep Tutorial}
\author{Alex Samuel}

% \input{boilerplate}

\makeindex

\begin{document}

\maketitle

\ifhtml
\chapter*{Front Matter\label{front}}
\fi

% \input{copyright}

\begin{abstract}

\noindent
This manual describes how to use the \pyhep Python tools for High Energy
Physics.  Rather than documenting all APIs exhaustively, it provides
examples of common use cases, and describes APIs and object protocols in
general terms.  See the automatically-generated API documentation, doc
comments, or the source code for detailed information about the APIs.

\warning{This document, and the software it describes, are incomplete.
In particular, the APIs and protocols described here and implemented in
the software may change, as the author gains experience with how the
software is best used.}

\end{abstract}

\tableofcontents

%-----------------------------------------------------------------------

\chapter{Introduction}

This tutorial presents a tour of \pyhep's most important features, and
shows how to perform some common tasks.  We assume you are familiar with
modern versions of the Python language, at least version 2.2.  

\pyhep makes use of several newer Python features, such as true division
and list comprehensions.  While you don't strictly need to be familiar
with these features to use \pyhep, some of them will help you use \pyhep
to its greatest advantage.  In particular, \pyhep is designed to support
a functional programming style, which is well-suited for many common HEP
computing tasks.  Many advanced Python features support functional
programming: the \function{map}, \function{reduce}, and
\function{filter} built-in functions; iterators; \code{lambda}
expressions; generators; and list comprehensions.


\chapter{Histograms}

A histogram is a tool for measuring an \textit{N}-dimensional
statistical distribution by summarizing samples drawn from it.

The histogram divides a rectangular region of the \textit{N}-dimensional
samples space into rectangular sub-regions, called \textit{bins}.  Each
bin records the number of recorded samples whose values fall in that
sub-region; when a sample is added into the histogram, or
\textit{accumulated}, the bin's value is incremented.

The histogram designates an \textit{axis} for each dimension of the
sample space, which specifies the range of coordinate values in that
dimension which the histogram will accept.  Coordinate values that fall
below or above that range are called \textit{underflows} and
\textit{overflows}, respectively.

Each axis also specifies the edges of the bins in the corresponding
dimension, which forms rectangular bins.  For \textit{evenly-binned}
axes, the positions of the bin edges are evenly spaced along the range
of the axis.  For \textit{unevenly-binned} axes, the bin edges may be
positioned arbitrarily; this allows smaller bins to be used in regions
of the coordinate value where samples are more likely to fall more
densely.

In some cases, samples are not drawn from the distribution which is
being measured, but from a modified distribution (for example, when
importance sampling).  The modified distribution can be transformed back
to the original distribution by assigning each sample a \textit{weight}.
The value of the corresponding bin will then be increased by the sample
weight.  When no weight is specified, unit weight is assumed.

If the samples are statistically independent, the contents of the bins
will provide an approximation of the probability density function from
which the samples are drawn.  This approximation is subject to
statistical uncertainty or \textit{error}.  The error is drawn from a
Poisson distribution; for large statistics, a Gaussian distribution is a
good approximation.  

%-----------------------------------------------------------------------

\section{Introductory example}

This example creates a one-dimensional histogram with five evenly-spaced
bins between zero and ten, fills some sample values into it, and dumps
the resulting contents.

Import the histogram module.
\begin{verbatim}
>>> import hep.hist
\end{verbatim}
Create a histogram object.
\begin{verbatim}
>>> histogram = hep.hist.Histogram1D(5, (0, 10))
\end{verbatim}
Show the histogram object we just created.
\begin{verbatim}
>>> print histogram
Histogram(EvenlyBinnedAxis(5, (0, 10)), bin_type=int, error_model='poisson')
\end{verbatim}
Accumulate samples into the histogram.
\begin{verbatim}
>>> for sample in (0, 4, 5, 6, 7, 7, 9, 10, 11):
...   histogram << sample
...
\end{verbatim}
Dump the histogram contents.
\begin{verbatim}
>>> hep.hist.dump(histogram)
Histogram, 1 dimensions
int bins, 'poisson' error model
 
axes: EvenlyBinnedAxis(5, (0, 10))
 
      axis 0             bin value / error
---------------------------------------------
[   None,      0)         0 +  1.148 -      0
[      0,      2)         1 +  1.360 -  1.000
[      2,      4)         0 +  1.148 -      0
[      4,      6)         2 +  1.519 -  2.000
[      6,      8)         3 +  1.724 -  2.143
[      8,     10)         1 +  1.360 -  1.000
[     10,   None)         2 +  1.519 -  2.000
 
\end{verbatim}

%-----------------------------------------------------------------------

\section{Creating histograms}

The class \class{hep.hist.Histogram} provides an arbitrary-dimensional
histogram with binned axes.  Optionally, a histogram can store errors
(uncertainties) of the contents of each bin, or can compute these
automatically from the bin contents.

To construct a \class{Histogram}, provide a specification of each axis,
one for each dimension.  You can specify an \code{Axis} object for each
axis (described later), but the easiest way to specify an evenly-binned
axis is with a tuple containing the number of bins, and a '(lo, hi)'
pair specifying the low edge of the first bin and the high edge of the
last bin.  For example,
\begin{verbatim}
>>> import hep.hist
>>> one_d_histogram = hep.hist.Histogram((20, (0, 100)))
\end{verbatim}
creates a one-dimensional histogram with 20 bins between values 0
and 100.  Why all the parentheses?  The outermost pair are for the function
call; the next pair group parameters of the (single) axis; the innermost
group the low and high boundaries of the axis range.

Similarly,
\begin{verbatim}
>>> two_d_histogram = hep.hist.Histogram((20, (-1.0, 1.0)), (100, (0.0, 10.0)))
\end{verbatim}
creates a two-dimensional histograms with 20 bins on the first axis and
100 bins on the second axis.

To create a one-dimensional histogram you may use
\function{Histogram1D}, which is equivalent except the axis's
parameters are specified directly as arguments instead of wrapped in a
sequence, so that one pair of parentheses may be omitted.  Thus, the
first example above may be written,
\begin{verbatim}
>>> one_d_histogram = hep.hist.Histogram1D(20, (0, 100))
\end{verbatim}

You can access the sequence of axes for a histogram from its \code{axes}
attribute.  You'll see that the tuple arguments you specified have been
converted into \code{EvenlyBinnedAxis} objects.
\begin{verbatim}
>>> two_d_histogram.axes
(EvenlyBinnedAxis(20, (-1.0, 1.0)), EvenlyBinnedAxis(100, (0.0, 10.0)))
\end{verbatim}
For one-dimensional histograms, you may also use the \code{axis} attribute.
\begin{verbatim}
>>> one_d_histogram.axis
EvenlyBinnedAxis(20, (0, 100))
\end{verbatim}

For evenly-binned axes such as these, the \code{range} and
\code{number_of_bins} attributes contain the axis parameters.  The
\code{dimensions} attribute contains the number of dimensions of the 
histogram.  This is always equal to the length of the \code{axes}
attribute.
\begin{verbatim}
>>> one_d_histogram.axis.number_of_bins
20
>>> one_d_histogram.axis.range
(0, 100)
>>> one_d_histogram.dimensions
1
\end{verbatim}

If you specify additional keyword arguments when creating the histogram,
they are added as attributes to the new histogram object.  For example,
\begin{verbatim}
>>> histogram = hep.hist.Histogram(
...   (40, (0.0, 10.0), "momentum", "GeV/c"),
...   (32, (0, 32), "track hits"),
...   title="drift chamber tracks")
>>> histogram.title
'drift chamber tracks'
\end{verbatim}
This creates a two-dimensional histogram of 40 bins of track energy
between 0 and 10 GeV/c, and 32 integer bins counting track hits.  Note
that this histogram has $(40 + 2) \times (32 + 2) = 1428$ bins,
including overflow and underflow bins on each axis.  The histogram has
an additional attribute \member{title} whose value is
\code{"drift chamber tracks"}.  You can, of course, set or modify
additional attributes like \member{title} after creating the histogram.

\subsection{Axis type}

By specifying the range for each axis, you also implicitly specify the
numeric type for values along the axis.  (If the low and high values you
specify for the range are of different types, a common type is obtained
by Python's standard \code{coerce} mechanism.)  So, for instance, if
you specify two integer values for the low and high boundaries for the
axis, the axis will expect integer values, but if you specify one
integer and one floating-point value, the axis will expect
floating-point values.

You can find out the type of an axis by consulting its \code{type}
attribute.  For example,
\begin{verbatim}
>>> histogram = hep.hist.Histogram((20, (-1.0, 1.0)), (20, (-10, 10)))
>>> histogram.axes[0].type
<type 'float'>
>>> histogram.axes[1].type
<type 'int'>
\end{verbatim}

If the type for an axis is \code{int} or \code{long} (because you
specified integer or long values for the axis range), the difference
between the high and low ends of the range must be an even multiple of
the number of bins.  If this is not desired, use \code{float} values for
the axis.  For example, this construction raises an exception:
\begin{verbatim}
>>> histogram = hep.hist.Histogram1D(8, (-50, 50))
[... stack trace ...]
ValueError: number of bins must be a divisor of axis range
\end{verbatim}
because 50 - (-50) = 100 is not a multiple of 8.  However, any of these
will work:
\begin{verbatim}
>>> histogram = hep.hist.Histogram1D(10, (-50, 50))
>>> histogram = hep.hist.Histogram1D(8, (-60, 60))
>>> histogram = hep.hist.Histogram1D(8, (-50.0, 50.0))
\end{verbatim}

\subsection{Additional axis parameters}

In addition to the number of bins and axis range, you may optionally
provide additional elements to a tuple specifying an axis (or as the
arguments to \function{Histogram1D}).  The third argument is the string
name of the axis quantity, and the fourth argument is a string
describing the units of this quantity.  For example,
\begin{verbatim}
h = hep.hist.Histogram1D(30, (0.05, 0.20), "mass", "GeV/$c^2$")
\end{verbatim}
creates a histogram of 30 bins of mass between 0.05 and 0.2 GeV/$c^2$.

\LaTeX markup may be used in the axis name and units; these will be
rendered appropriately when plotting the histogram.  See the section on
\LaTeX markup for details of the syntax.

\subsection{Bin type and error model}

A histogram stores for each bin the total number of samples, or the sum
of weights of samples, that have fallen in this bin.  The keyword
argument \var{bin_type} specifies the numerical type used to store these
totals.  The histogram's \code{bin_type} attribute contains this type.
\begin{verbatim}
>>> histogram = hep.hist.Histogram1D(10, (-50, 50))
>>> histogram.bin_type
<type 'int'>
>>> histogram = hep.hist.Histogram1D(10, (-50, 50), bin_type=float)
>>> histogram.bin_type
<type 'float'>
\end{verbatim}

By default, bin contents are stored as \code{int} values.
\emph{Therefore, if you plan to use non-integer weights, you must
specify} \code{bin_type=float} \emph{when creating the histogram!}
Otherwise, weights will be truncated to integers when filling.  If your
weights are all less than one, you fill find your histogram to be empty.

A histogram can also estimate the statistical counting uncertainty on
each bin.  This error is represented by a 68.2\% confidence interval
around the bin value.  The interval is represented by a pair of values
specifying the sizes of the low and high ``error bars'',
\textit{e.g.} if the bin value is \code{20} and the errors are
\code{(5.5, 4.5)}, the 68.2\% confidence interval is the range (14.5,
24.5).

Several models are available are available that control how the errors
are stored or computed:
\begin{itemize}
 \item \code{"none"}: Each bin is assumed to have zero uncertainty.

 \item \code{"gaussian"}: The errors are computed assuming symmetrical
 Gaussian counting errors.  The bin value is interpreted as a number of
 counts, and the low and high errors are both the square root of the bin
 content.

 \item \code{"poisson"}: The errors are computed assuming Poisson
 counting errors.  The bin value is interpreted as a number of counts,
 and the low and high errors are chosen to cover 68.2\% of the Poisson
 cumulative distribution around the bin value.  The confidence interval
 is chosen to cover 34.1\% on either side of the central value where
 possible.  However, for a central value of zero, one, or two counts,
 this would produce a confidence interval with a lower edge below zero,
 so the lower edge is fixed at zero and the upper edge is chosen to
 capture 68.2\%.

 \item \code{"symmetric"}: For each bin, the histogram stores a single
 value representing both the lower and upper errors.  The error value
 may be specified explicitly for each bin with the \method{setBinError}
 method.  

 When a sample is accumulated into the histogram, the sample weight is
 added in quadrature to the bin error.  The error value on a bin (used
 as both the upper and lower error) are is given by $\sigma=\sqrt{\sum
 w_i^2}$ where $w_i$ are the sample weights accumulated into the bin.

 \item \code{"asymmetric"}: For each bin, the histogram stores two
 values representing the lower and upper errors.  The two error values
 may be specified explicitly for each bin with the \method{setBinError}
 method.  

 Errors are computed from weights as in the \code{"symmetric"} error
 model.  The only difference is that you may specify manually different
 values for the lower and upper errors using the \method{setBinError}
 method.
\end{itemize}

Specify the error model for a histogram with the \var{error_model}
keyword argument.  The default is \constant{"poisson"} if the bin type is
\code{int} or \code{long}, or \constant{"gaussian"} otherwise.  If you will
specify bin errors explicitly using the \method{setBinError} method, you
must specify the \constant{"symmetric"} or the \constant{"asymmetric"}
error model.  The \constant{error_model} attribute contains a histogram's
error model.

For example,
\begin{verbatim}
>>> histogram = hep.hist.Histogram1D(10, (-50, 50))
>>> histogram.error_model
'poisson'
>>> histogram = hep.hist.Histogram1D(10, (-50, 50), bin_type=float)
>>> histogram.error_model
'gaussian'
>>> histogram = hep.hist.Histogram1D(10, (-50, 50), error_model="symmetric")
>>> histogram.error_model
'symmetric'
\end{verbatim}

%-----------------------------------------------------------------------

\section{Other kinds of axes}

The histogram axes we've used up to now have been evenly binned,
\textit{i.e.} all the bins on each axis are the same size.  It is also
possible to create a histogram with unevenly-binned axes, using the
\class{hep.hist.UnevenlyBinnedAxis} class.  

Create an \class{UnevenlyBinnedAxis} instance, specifying a list of bin
edges as its argument.  You can then specify this axis when creating a
histogram.  Note that each histogram axis is independently specified, so
that one, several, or all may be unevenly binned.

For example, to create a 2-D histogram with an unevenly-binned x axis
and one evenly-binned y axis,
\begin{verbatim}
>>> import hep.hist.axis
>>> x_axis = hep.hist.UnevenlyBinnedAxis(
...     [0, 2, 4, 5, 6, 8, 10, 15, 20], name="count")
>>> histogram = hep.hist.Histogram(x_axis, 
...     (10, (0., 5.), "time", "sec"))
\end{verbatim}

%-----------------------------------------------------------------------

\section{Filling histograms}

To ``fill'' a histogram is to accumulate samples from the sample
distribution into it.  A sample is represented by a sequence of
coordinate values, where the length of the sequence is equal to the
number of dimensions of the histogram.   Each item in the sequence is
the coordinate value along the corresponding axis of the histogram.

For each bin, the histogram keeps track of the number of accumulated
samples whose coordinates fall within the bin.  Optionally, samples may
be specified a ``weight''; in that case, the histogram keeps track of
the sum of weights of samples.  The numeric type used to store the
number of samples or sum of weights for each bin is given by the
\code{bin_type} attribute.  The histogram also tracks the total
number of accumulations (the ``number of entries'') that you have made.

The \method{accumulate} method accumulates an event into the histogram.
Specify the coordinates of the sample, and optionally the event weight
(which otherwise is taken to be unity).  The left-shift operator
\code{<<} is shorthand for \method{accumulate} with unit weight.

If any of the coordinate values passed to \method{accumulate} is
\code{None}, the sample is not accumulated into the histogram and the number
of entries is not changed.

For example, to accumulate a the sample whose coordinates are \code{x}
and \code{y} into a two-dimensional histogram \code{histogram},
\begin{verbatim}
>>> histogram.accumulate((x, y))
\end{verbatim}
or
\begin{verbatim}
>>> histogram << (x, y)
\end{verbatim}
To accumulate the same sample with weight \code{weight},
\begin{verbatim}
>>> histogram.accumulate((x, y), weight)
\end{verbatim}

For one-dimensional histograms, you may specify the sample coordinate by
itself, instead of as a one-element sequence.  So, to accumulate the
sample with coordinate \code{x} into one-dimensional \code{histogram},
you may use any of these:
\begin{verbatim}
>>> histogram.accumulate((x, ))
>>> histogram.accumulate(x)
>>> histogram << (x, )
>>> histogram << x
\end{verbatim}

Plotting histograms is described later in the plotting chapter.  A quick
way of displaying histogram contents in text format is with the
\function{hep.hist.dump} function.

This script below creates a one-dimensional histogram with eleven
integer bins between 2 and 12, inclusive, and fills the histogram with
the result of simulating 1000 rolls of two dice.
\codesample{histogram1.py}

%-----------------------------------------------------------------------

\section{Accessing histogram contents}

You may specify a particular bin of a histogram with a ``bin number'',
which is a sequence of bin positions along successive axes.  The length
of the sequence is equal to the histogram's number of dimensions.  Each
item is the index of the bin along the corresponding axis.

Along each axis, the coordinate in the bin number ranges between zero
and one less than the number of bins on the axis.  It may also take the
values \code{"underflow"} and \code{"overflow"}, which denote the
underflow and overflow bins, respectively

For example, consider this histogram:
\begin{verbatim}
>>> histogram = hep.hist.Histogram((20, (-1.0, 1.0)), (24, (0, 24)))
\end{verbatim}
The corner bin numbers are for this histogram \code{(0,~0)},
\code{(9,~0)}, \code{(0,~23)}, and \code{(9,~23)}.  
The bin whose number is \code{(12,~"underflow")} is for any samples
whose first coordinate is between 0.2 and 0.3, and whose second
coordinate is less than 0.  The bin whose number is
\code{("underflow",~"overflow")} is for any sample whose first
coordinate is less than -1 and whose second coordinate is greater than
24.

To get the bin number corresponding to a sample point, use the
\method{map} method, passing the sample coordinates. 

Just as with sample coordinates, for a one-dimensional histogram you may
specify either the bin number as a one-element sequence, or simply the
bin number along the (only) axis.

To obtain the contents of a bin, use the \method{getBinContent} method,
passing the bin number.  To obtain the 68.2\% confidence interval on a
bin, use the \method{getBinError} method, which returns two values
specifying how far the interval extends below and above the central
value.

For example,
\begin{verbatim}
>>> histogram = hep.hist.Histogram1D(10, (0.0, 1.0), error_model="gaussian")
>>> histogram.accumulate(0.64, 17)
>>> histogram.map(0.64)
(6,)
>>> histogram.getBinContent((6, ))
17
>>> histogram.getBinError((6, ))
(4.1231056256176606, 4.1231056256176606)
\end{verbatim}
In the \code{"gaussian"} error model, the errors on the bin are the
square root of the bin contents, here, $\sqrt{17}$=4.123106.  Since the
histogram is one-dimensional, we just as easily could have used,
\begin{verbatim}
>>> histogram.getBinContent(6)
\end{verbatim}

To set the contents of a bin, use the \method{setBinContent} method,
specifying the new value as the second argument.  To set the error
estimate on a bin, use the \method{setBinError} method and specify a
pair for the sizes of lower and upper errors.  Note that you may only
call the \method{setBinError} method of a histogram with
\code{"asymmetric"} or \code{"symmetric"} error model, and in the latter
case, the average of the lower and upper error values you specify is
used as the single symmetric error estimate.

To obtain the range of coordinate values spanned by a single bin, use
the \method{getBinRange} method, passing the bin number.  The return
value is a sequence, each of whose items is a \code{(lo, hi)} pair of
coordinate values along on axis spanned by the bin.  For example, to
print the bin range and value for bins in a one-dimensional histogram,
\begin{verbatim}
>>> for bin in range(histogram.axis.number_of_bins):
...     (lo, hi), = histogram.getBinRange(bin)
...     content = histogram.getBinContent(bin)
...     print "bin (%f,%f): %f" % (lo, hi, content)
\end{verbatim}
Note that the return value from \method{getBinRange} is a one-element
sequence, since the histogram is one-dimensional.  The single element is
the pair \code{(lo, hi)} of the bin's range along the histogram's axis.

A histogram's \member{number_of_samples} attribute contains the number
of times the \method{accumulate} method (or the \code{<<} operator) was
invoked.

%-----------------------------------------------------------------------

\section{More histogram operations}

The following functions are provided in \module{hep.hist}.  Invoke
\code{help(hep.hist.}\textit{function}\code{)} for a description of a
function's parameters.

\begin{itemize}
 \item The function \function{scale} produces a copy of a histogram with
 its contents scaled by a constant factor.  

 \item The function \function{integrate} returns returns the sum over
 all bins of a histogram.  Specify \code{overflows=True} to include
 underflow and overflow bins in the integral.  

 \item The function \function{normalize} returns a copy of a histogram,
 scaled such that its integral a fixed integral value (by default, one).

 \item The function \function{normalizeSlices} returns a copy of a
 histogram in which slices along a specified axis are normalized
 independently to a fixed integral value (by default, one).

 \item The functions \function{add} and \function{divide} create a new
 histogram by adding or dividing, respectively, the corresponding bins
 in two histograms.  The histograms must have the same axis ranges and
 binning.

 \item The function \function{getMoment} computes the Nth moments of a
 histogram in each of its dimensions.

 \item The \function{mean} and \function{variance} functions compute
 those two statistics. 

 \item The function \function{slice} produces an (N-1)-dimensional
 histogram by slicing or projecting out one dimension of an
 N-dimensional histogram.  

 \item The function \function{rebin} produces a copy of a histogram with
 groups of adjacent bins combined together.

 \item The function \function{transform} transforms a histogram axis by
 an arbitrary monotonically increasing function by creating a copy of
 the histogram with adjusted bin boundaries.

 \item The function \function{getRange} determines the range of bin
 values in a histogram.  Optionally, the range can accommodate the error
 intervals on each bin.

 \item The function \function{dump} prints the contents of a histogram
 to standard output or another file.

 \item The function \function{project} accumulates samples
 simultaneously into several histograms from an array of sample events.
 This function is described later in the chapter on tables.

\end{itemize}

You may also add or divide two histogram with the ordinary addition and
division operators, respectively, and you may scale a histogram with the
ordinary multiplication operator.  For example,
\begin{verbatim}
>>> combined_histogram = 3 * histogram1 + histogram2
\end{verbatim}

To iterate over all bins in a histogram, use these functions.  Each
takes an optional second argument \code{overflow}; if true, underflow
and underflow bins are included (by default false).
\begin{itemize}
  \item \function{AxisIterator} takes an \code{Axis}.  It returns an
  iterator that yields the bin numbers for that axis.

  \item \function{AxesIterator} takes a sequence of \code{Axis} objects,
  such as the value of a histogram's \member{axes} attribute.  It return
  s an iterator that yields the bin numbers for the multidimensional bin
  space specified by the axes.

  \item \function{BinValueIterator} takes a histogram.  It returns an
  iterator that yields the contents of each bin in the histogram.

  \item \function{BinErrorIterator} takes a histogram.  It returns an
  iterator that yields the error estimate on each bin in the histogram.

  \item \function{BinIterator} takes a histogram.  It returns an
  iterator that yields triplets \code{(bin_number, contents, error)} for
  each bin in the histogram.

\end{itemize}

For example, the code below shows how you might find the largest bin
value in a histogram, including underflow and overflow bins.  (You could
also use the \function{getRange} function.)
\begin{verbatim}
>>> max(hep.hist.BinValueIterator(histogram, overflow=True))
\end{verbatim}

%-----------------------------------------------------------------------

\section{Scatter plots}

A \class{hep.hist.Scatter} object collects sample points from a
bivariate distribution.  The samples are typically displayed as a
``scatter plot''.

The sample points are not binned in any way; rather, the two coordinates
of each samples is stored.  Unlike with a histogram, the memory usage of
a \class{Scatter} object increases as additional samples are
accumulated.  However, the resulting object contains the full covariance
information for the two coordinate variables, and can be plotted
precisely.

To create a \class{Scatter}, specify information about the two axes.
Each may be an instance of \class{hep.hist.Axis}, or more simply a
sequence '(type, name, units)', where 'type' is the Python type of
coordinate values along this axis, and 'name' and 'units' are strings
describing the coordinate values (which may be omitted).  If no axes are
specified, they are taken to have 'float' coordinates

For example, this code creates a \class{Scatter} object.
\begin{verbatim}
>>> scatter = hep.hist.Scatter(
...     (float, "energy", "GeV"), (int, "number of hits"),
...     title="candidate tracks")
\end{verbatim}

Use the \method{accumulate} method to add a sample to the
\class{Scatter}.  The argument must be a two-element sequence containing
the two coordinates.  The left-shift operator \code{<<} is a synonym for
\method{accumulate}.  The sample values are stored in a sequence
attribute \member{points}.

For example,
\begin{verbatim}
>>> for track in tracks:
...     scatter << (track.energy, track.number_of_hits)
... 
\end{verbatim}


\chapter{Tables}

\pyhep provides a flat-format database facility, similar to ``n-tuples''
in other HEP analysis systems.  

A \pyhep table is similar to a table in a typical database system.  A
table is a sequence of rows, each of which has the same number of
columns and values for each column of the same type.  The number and
types of columns constitutes the table's schema.  

\pyhep's tables are subject to the following restrictions:

\begin{itemize}

 \item A table's schema is set when it is created, and cannot
 subsequently be changed.  (Creating another table with a different
 schema using the same data is straightforward.)

 \item The size in bytes of each table row is a constant.  This limits
 the types of table columns, but allows the implementation of fast seeks
 in a large table.

 \item Rows are appended to the table.  An entire row must be appended
 at one time.  Rows may not be inserted into the middle of a table,
 modified, or removed.

\end{itemize}

\section{Table implementations}

Several table implementations are provided.  Other than for creating new
tables or opening existing tables, the interfaces of these
implementations are identical.

\begin{enumerate}

 \item A extension-class implementation written in C++ in the
 \module{hep.table} module.  The table resides in a disk file, and rows
 are loaded into memory as needed.  The implementation is simple and
 efficient.  Other programs may write and access these tables via a
 simple C++ interface.

 \item An implementation which uses HBOOK n-tuples (column-wise or
 row-wise) stored in HBOOK files.  Not all features of HBOOK n-tuples are
 supported.

 \item An implementation which uses ROOT tress stored in ROOT files.
 Not all features of ROOT trees are supported.

\end{enumerate}

The extension-class implementation in \module{hep.table} is used in this
tutorial. 

\section{Creating and filling tables}

A table schema is represented by an instance of
\class{hep.table.Schema}.  The schema collects together the definitions of the columns in the table.  Each column is identified by a name, which is
a string composed of letters, digits, and underscores.  

A new instance of \class{Schema} has no columns.  Add columns using the
\method{addColumn} method, specifying the column name and type.  For instance, 
\begin{verbatim}
>>> import hep.table
>>> schema = hep.table.Schema()
>>> schema.addColumn("energy", "float64")
>>> schema.addColumn("momentum", "float64")
>>> schema.addColumn("hits", "int32")
\end{verbatim}

The second argument to \function{addColumn} specifies the storage format
used for values in that column.  These column types are supported in
\code{hep.table} (note that other table implementations may not support
all of these):
\begin{itemize}
 \item three signed integer types: \constant{"int8"}, \constant{"int16"},
 and \constant{"int32"}

 \item two floating-point types, \constant{"float32"} and
 \constant{"float64"}

 \item two floating-point complex types, \constant{"complex64"} and
 \constant{"complex128"}
\end{itemize}

More concisely, you may specify columns as keyword arguments, so the
above schema may be constructed with,
\begin{verbatim}
>>> schema = hep.table.Schema(energy="float64", momentum="float64", hits="int32")
\end{verbatim}

You can also load or save schemas in an XML file format using
\function{hep.table.loadSchema} and \function{hep.table.saveSchema}.
The schema above would be represented by the XML file
\begin{verbatim}
<?xml version="1.0" ?>
<schema>
 <column name="energy" type="float64"/>
 <column name="momentum" type="float64"/>
 <column name="hits" type="int32"/>
</schema>
\end{verbatim}

Call the \function{create} function to create a new table.  The
parameters will vary for each implementation; for the \module{hep.table}
implementation, the parameters are the file name for the new table, and
the table's schema.  
\begin{verbatim}
>>> table = hep.table.create("test.table", schema)
\end{verbatim}
The return value is a \emph{connection} to the newly-created table,
which resides on disk.

To add a row to the table, call the table's \method{append} method.  You
may pass it a mapping argument (such as a dictionary) that associates
column values with column names, and/or you may provide column values as
keyword arguments.  One way or another, you must specify values for all
columns in the table.  The return value of \method{append} is the index
of the newly-appended row.

For instance, to add a row to the table created above,
\begin{verbatim}
>>> row = {
...   "energy": 2.7746,
...   "momentum": 1.8725,
...   "hits": 17,
>>> }
>>> table.append(row)
\end{verbatim}
or equivalently,
\begin{verbatim}
>>> table.append(energy=2.7746, momentum=1.8725, hits=17)
\end{verbatim}

\subsection{Example: creating a table from a text file}

The script below converts a table of values in a text file into a
\pyhep table.  The script assumes that the file contains floating-point
values only, except that the first line of the text file contains
headings that will be used as the names of the columns in the table.
\codesample{table1.py}

Here is a table containing parameters for tracks measured in a
detector.  The first column is the track's energy; the other three are
the x, y, and z components of momentum.
\codesample{tracks.txt}

If you save the script as \file{txt2table.py} and the table as
\file{tracks.table}, you would invoke this command to convert it to a
table: 
\begin{verbatim}
> python txt2table.py tracks.txt tracks.table
\end{verbatim}

\section{Using tables}

To open an existing table, use the \function{open} function.  The first
argument is the table file name.  The second argument is the mode in
which to open the table, similar to the built-in \function{open}
function: \constant{"r"} (the default) for read-only, \constant{"w"} for
write.  

For instance, to open the table we created in the last section,
\begin{verbatim}
>>> tracks = hep.table.open("tracks.table")
\end{verbatim}

The table object is a sequence whose elements are the rows.  Each row is
has a row index, which is equal to its position in the table.  Thus, the
\function{len} function returns the number of rows in the table.
\begin{verbatim}
>>> number_of_tracks = len(tracks)
\end{verbatim}
To access a row by its row index, use the normal sequence index
notation.  This returns a \class{Row} object, which is a mapping from
column names to values.
\begin{verbatim}
>>> track = tracks[19]
>>> print track["energy"]
\end{verbatim}

The table's \member{schema} attribute contains the table's schema; the
schema's \member{column} attribute is a sequence of the columns in the
schema (in unspecified order).  You can examine these directly, or use
the \function{dumpSchema} function to print out the schema.
\begin{verbatim}
>>> hep.table.dumpSchema(tracks.schema)
name             type
-----------------------------
energy           float32
p_x              float32
p_y              float32
p_z              float32
 
\end{verbatim}


\section{Iterating over rows}

For most HEP applications, the rows of a table represent independent
measurements, and are processed sequentially.  An \emph{iterator}
represents this sequential processing of rows.  Using iterators instead
of indexed looping constructs simplifies code, opens up powerful
functional-programming methods, and enables automatic optimization of
independent operations on rows.

Since a table satisfies the Python sequence protocol, you can produce an
iterator over its elements (\textit{i.e.} rows) with the \function{iter}
function.  The Python \code{for} construction does even this
automatically.  The simplest idiom for processing rows in a table
sequentially looks like this:
\begin{verbatim}
>>> total_energy = 0
>>> for track in tracks:
...     total_energy += track["energy"]
...
>>> print total_energy
74.7496351004
\end{verbatim}

Since \code{iter(tracks)} is an iterator rather than a sequence of all
rows, each row is loaded from disk into memory only when needed in the
loop, and is subsequently deleted.  This is critical for scanning over
large tables.  (Note that after the loop completes, the last row remains
loaded, until the variable \var{track} is deleted or goes out of scope.
Also, the table object itself is deleted only after any variables that
refer to it, as well as any variables that refer to any of its rows, are
deleted.)

Table iterators may be used to iterate over a subset of rows in a
sequence.  Most obviously, you could implement this by using a
conditional in the loop.  For instance, to print the energy of each
track with an energy greater than 2.5,
\begin{verbatim}
>>> for track in tracks:
...     if track["energy"] > 2.5:
...         print track["energy"]
... 
\end{verbatim}
While this is straightforward, it forces \pyhep to examine each row
every time the program is run.  By using the selection in the iterator,
\pyhep can optimize the selection process, often significantly.
The selection criterion can be any boolean-valued expression involving
the values in the row.  The selection expression is evaluated for each
row, and if the result is true, the iterator yields that row; otherwise,
that row is skipped.  (This is semantically similar to the first
argument of the built-in \function{filter} function.)  The expression
can be a string containing an ordinary Python expression using column
names as if they were variable name (with certain limitations and
special features), or may be specified in other ways.  Expressions are
discussed in greater detail later.

For instance, the same high-energy tracks can be obtained using the
selection expression \constant{"energy > 2.5"}.  Notice that
\var{energy} appears in this expression as if it were a variable defined
when the expression is evaluated.  The \method{select} method returns an
iterator which yields only rows for which the selection is true.
\begin{verbatim}
>>> for track in tracks.select("energy > 2.5"):
...     print track["energy"]
... 
\end{verbatim}

Python's list comprehensions provide a handy method for collecting
values from a table.  For instance, to enumerate all values of
\var{energy} above 2.5 instead of merely printing them,
\begin{verbatim}
>>> energies = [ track["energy"] for track in tracks.select("energy > 2.5") ]
\end{verbatim}
Note here that \code{tracks.select("energy > 2.5")} returns an iterator
object, so it may only be used (i.e. iterated over) once.  However, if
you really want a sequence of \code{Row} objects, you can use the
\function{list} or \function{tuple} functions to expand an iterator into
an actual sequence, as in
\begin{verbatim}
>>> high_energy_tracks = list(tracks.select("energy > 2.5"))
\end{verbatim}
Such a sequence consumes more resources than an iterator.  You can
iterate over such a sequence repeatedly, or perform other sequence
operations. 

 
\section{Projecting histograms from tables}

If you have a sequence or iterator of values, you can use the built-in
\function{map} function to accumulate them into a histogram.  For
instance, if \var{energies} is a sequence of energy values, you could
fill them into a histogram using,
\begin{verbatim}
>>> energy_hist = hep.hist.Histogram1D(20, (0.0, 5.0))
>>> map(energy_hist.accumulate, energies)
\end{verbatim}

Generally, though, you will want to project many histograms at one time
from a table.  Use the \function{hep.hist.project} function to project
multiple histograms in a single scan over a table.  Pass an iterable
over the table rows to project---the table itself, or an iterator
constructed with the \method{select} method---and a sequence of
histogram objects.  Each histogram should have an attribute
\member{expression}, which is the expression whose value is accumulated
into the histogram for each table row.  The expression are similar to
those used with the \method{select} method, except that their values
should be numerical instead of boolean.

For example, this script projects three histograms -- energy, transverse
momentum, and invariant mass, from the table of tracks we constructed
previously.  Only high-energy tracks (those with energy above 2.5) are
included.  The dictionary containing the histograms is stored in a
standard pickle file.
\codesample{project1.py}
Observe that the last histogram is two-dimensional, and its expression
specifies the two coordinates of the sample using a comma expression.

With this scheme, you can determine later how the values in a histogram
were computed, by checking its \member{expression} attribute.


\section{Using expressions with tables}

Expressions are described in more detail in a later chapter.  This
section presents techniques for optimizing expressions used with tables.

As described above, you can use Python expressions encoded in character
strings with \method{select} method and \function{hep.hist.project}
function.  When such an expressions are evaluated, unbound symbols
(\textit{i.e.} variables) are bound to the value of the corresponding
column.

Since a table row satisfies the Python mapping protocol, you can pass a
table row directly to an expression's \method{evaluate} method.  For
example, this constructs an expression object to compute the transverse
momentum of a track in the tracks table constructed above.
\begin{verbatim}
>>> import hep.expr
>>> p_t = hep.expr.asExpression("hypot(p_x, p_y)")
\end{verbatim}
Its \method{evaluate} method computes transverse momentum for a track by
binding \var{p_x} and \var{p_y} to the corresponding values of the table
row. 
\begin{verbatim}
>>> print p_t.evaluate(tracks[0])
0.92102783203
\end{verbatim}
To find the largest transverse momentum in the track table,
\begin{verbatim}
>>> print max(map(p_t.evaluate, tracks))
2.44177757441
\end{verbatim}

The chapter on expressions, below, describes how to compile an
expression into a format for faster evaluation.  When the expression is
used with a table, you can produce an even faster version by compiling
it with the table's \method{compile} method.  This sets the symbol types
correctly based on the table's schema, and applies additional
optimizations specific to tables.  For example,
\begin{verbatim}
>>> p_t = tracks.compile("hypot(p_x, p_y)")
>>> print p_t.evaluate(tracks[0])
0.92102783203
\end{verbatim}
When you evaluate an expression on many rows of a large table,
performance will be substantially better if you compile the expression
first for that table.  Note that an expression compiled for a table
should only be used with that table.


\subsection{Caching expressions in tables}

You can also configure a table to cache the results of evaluating a
boolean expression on the rows.  The first time you evaluate the
expression on a row, the row is loaded and the expression is evaluated
as usual.  On subsequent times, the table reuses the cached value of the
expression, instead of reloading the row and re-evaluating the
expression. 

To instruct a table to cache the value of an expression, pass the
expression to the table's \method{cache} method.  Do not pass a compiled
expression; pass the uncompiled form instead.  You may pass a string or
function here as well.  Once you add the expression to the table's
cache, the expression cache will automatically be used when you compile
the expression or use it with \method{select} and other table functions.

For example, suppose your table file \file{tracks.table} was extremely
large, and you expect to select repeatedly all tracks with energy above
2.5.  You could cache this selection expression like this:
\begin{verbatim}
>>> cut = "energy > 2.5"
>>> tracks.cache(cut)
\end{verbatim}
\begin{verbatim}
>>> cut = tracks.compile(cut)
\end{verbatim}
Optionally, compile the expression and evaluate it on each row once, to
fill the cache.
\begin{verbatim}
>>> cut = table.compile(cut)
>>> for track in tracks:
...   cut.evaluate(track)
... 
\end{verbatim}

Once you have added a cached expression to the table, the cache will be
used automatically whenever you compile the expression for that table:
the compiled \code{cut} above uses the cache.  The cache will also be
used for subexpressions, so if you were to compile the expression
\code{"mass < 1 and energy > 2.5"} would use the cache, too.

\subsection{Row types}

As we have seen above, the object representing one row of a table
satisfies Python's read-only mapping protocol: it maps the name of a
column to the corresponding value in the row.  While in many ways, they
behave like ordinary Python dictionaries (for instance, they support the
\method{keys} and \method{items} methods), they actually instances of
the \class{hep.table.RowDict} class.
\begin{verbatim}
>>> track = tracks[0]
>>> print type(track)
<type 'RowDict'>
\end{verbatim}
If you ever need an actual dictionary object containing the values in a
row, Python will produce that for you:
\begin{verbatim}
>>> dict(track)
{'energy': 1.1803040504455566, 'p_z': -0.73052072525024414, 
 'p_x': -0.47519099712371826, 'p_y': -0.78897768259048462}
\end{verbatim}

For some applications, however, it is more convenient to use a different
interface to access row data.  You can specify another type to use for
row objects by setting the table's \member{row_type} attribute (or with
the \var{row_type} keyword argument to \function{hep.table.open}).  The
default value, as you have seen, is \class{hep.table.RowDict}.

\pyhep includes a second row implementation,
\class{hep.table.RowObject}, which provides access to row values as
object attributes instead of items in a map.  The row has an attribute
named for each column in the table, and the value of the attribute is
the corresponding value in the row.

For example,
\begin{verbatim}
>>> tracks.row_type = hep.table.RowObject
>>> track = tracks[0]
>>> print track.p_x, track.p_y, track.p_z
-0.475190997124 -0.78897768259 -0.73052072525
\end{verbatim}

You may also derive a subclass from \class{RowObject} and use that as
your table's row type.  This is very handy for adding additional
methods, get-set attributes, etc. to the row, for instance to compute
derived values.

For example, you could create a \class{Track} class that provides the
mass and scalar momentum as ``attributes'' that are computed dynamically
from the row's contents.
\begin{verbatim}
>>> from hep.num import hypot
>>> from math import sqrt
>>> class Track(hep.table.RowObject):
...   momentum = property(lambda self: hypot(self.p_x, self.p_y, self.p_z))
...   mass = property(lambda self: sqrt(self.energy**2 - self.momentum**2))
...
\end{verbatim}
Now set this class as the row type.
\begin{verbatim}
>>> tracks.row_type = Track
>>> track = tracks[0]
>>> print type(track)
<class '__main__.Track'>
\end{verbatim}
You can now access the members of \class{Track}:
\begin{verbatim}
>>> print track.momentum
1.17556488438
>>> print track.mass
0.105663873224
\end{verbatim}

Setting a table's row type to \class{RowObject} or a subclass will not
break evaluation of compiled expressions on row objects.  Expressions
look for a \method{get} method, which is provided by both
\class{RowDict} and \class{RowObject}.

Note that computing complicated derived values in this way is less
efficient than using compiled expressions, as described above.  However,
you can create methods or get-set members that evaluate compiled
expressions.  

Only subclasses of \class{RowDict} and \class{RowObject} may be used as
a table's \member{row_type}.


\section{More table functions}

The function \function{hep.table.project} is a generalization of
\function{hep.hist.project}.  Like the latter, it iterates over table
rows and evaluates a set of expressions on each row.  Instead of
accumulating the expression values into histograms, passes them to
arbitrary functions.  Invoke \code{help(hep.table.project)} for usage
information. 

The \class{hep.table.Chain} class concatenates multiple tables into one.
Simply pass the tables as arguments.  Of course, any columns accessed in
the chain must appear in all the included tables.  For example, this
code chains together all table files (files with the ".table" extension)
in the current working directory.
\begin{verbatim}
>>> import os
>>> tables = [ hep.table.open(path)
...            for path in os.listdir(".") 
...            if path.endswith(".table") ]
>>> chain = hep.table.Chain(*tables)
\end{verbatim}



\chapter{Expressions}

We have already seen expressions used as predicates in a table's
\method{select} method, and as formulas for computing the values to
accumulate into histograms.  Generally, operations using expressions
execute much faster than the same logic coded directly as Python, since
expressions are compiled for the specific table into a special format
from which they are evaluated very quickly.

\section{Building and evaluating expressions}

To parse an expression into a Python expression object representing its
parse tree, use \function{hep.expr.asExpression}.
\begin{verbatim}
>>> import hep.expr
>>> ex = hep.expr.asExpression("p_x ** 2 + p_y ** 2")
\end{verbatim}
The expression object's string representation looks similar to the
original formula:
\begin{verbatim}
>>> print ex
(p_x ** 2) + (p_y ** 2)
\end{verbatim}
The expression object's \function{repr} shows its tree structure:
\begin{verbatim}
>>> print repr(ex)
Add(Power(Symbol('p_x', None), Constant(2)), Power(Symbol('p_y', None), Constant(2)))
\end{verbatim}

To evaluate an expression, you must provide the values of all symbols.
For the expression above, the symbols \var{p_x} and \var{p_y} must be
specified.  You may either call the expression object directly, passing
symbol values as keyword arguments:
\begin{verbatim}
>>> ex(p_x=1, p_y=2)
5.0
\end{verbatim}
or you may call its \method{evaluate} method with a map providing values
of the symbols mentioned in the expression:  
\begin{verbatim}
>>> ex.evaluate({"p_x": 1, "p_y": 2})
5.0
\end{verbatim}

\section{Compiling expressions}

Expression objects, as well as expression coded directly in Python,
execute quite a bit slower than similar mathematical expressions
implemented in a compiled language like C or C++.  However, \pyhep can
often execute an expressions much faster, by compiling it to an internal
binary format and then evaluating it with an optimized, stack-based
evaluator implemented in C++.

To compile an expression to this optimized form, use
\function{hep.expr.compile}.  It returns an expression object that can
be used the same way as the original expression, but which executes
faster.
\begin{verbatim}
>>> cex = hep.expr.compile(ex)
>>> cex(p_x=1, p_y=2)
5.0
\end{verbatim}

There is no need to use \code{asExpression} before \code{compile}; just
pass it the expression formula directly.
\begin{verbatim}
>>> cex = hep.expr.compile("p_x ** 2 + p_y ** 2")
>>> cex(p_x=1, p_y=2)
5.0
\end{verbatim}


\subsection{Expression types}

\pyhep attempts to determine the numeric type of the result of an
expression.  To do this, it needs to know the types of the symbol values
in the expression.  For constants, this is obvious, but since Python is
an untyped language, the expression compiler cannot automatically
determine the types of symbolic names in an expression, and treats them
as generic objects.  

\pyhep expressions understand the types \code{int}, \code{float}, and
\code{complex}.  The value \code{None} indicates that the type is not
known, so the value should be treated as a generic Python object.

For example, in these expressions, \pyhep can infer the type of the
expression value entirely from the types of constants.
\begin{verbatim}
>>> print hep.expr.asExpression("10 + 12.5").type
<type 'float'>
>>> print hep.expr.asExpression("3 ** 4").type
<type 'int'>
\end{verbatim}
However, a symbol's type is assumed to be generic, so the whole
expression's type cannot be inferred.
\begin{verbatim}
>>> print hep.expr.asExpression("2 * c + 10").type
None
\end{verbatim}

The expression compiler can do a much better job if it knows the
numerical type of the expression's symbols.  When you call
\function{compile}, you can specify the types of symbols as keyword
arguments.  For example,
\begin{verbatim}
>>> cex = hep.expr.compile("2 * c + 10", c=int)
>>> print cex.type
<type 'int'>
\end{verbatim}
If you provide a second non-keyword argument, this type is used as the
default for all symbols in the expression.
\begin{verbatim}
>>> cex = hep.expr.compile("a**2 + b**2 + c**2", float)
>>> print cex.type
<type 'float'>
\end{verbatim}
By specifying symbol types, you can construct compiled expressions that
execute much faster than expressions with generic types.

You can also construct an uncompiled expression with types specified for
symbols using the \function{hep.expr.setTypes},
\function{hep.expr.setTypesFrom}, and \function{hep.expr.setTypesFixed}
functions.


\section{Using expressions with tables}

Since a table row object is a map from column names to values, you may
specify a row object as the argument to \method{evaluate}.  In the
expression, the name of a column in the table is replaced by the
corresponding value in that row.  Using the \file{tracks.table} table we
created earlier,
\begin{verbatim}
>>> import hep.table
>>> tracks = hep.table.open("tracks.table")
>>> print ex.evaluate(tracks[0])
0.848292267373
\end{verbatim}

This works well for small numbers of rows.  However, if the expression
is to be evaluated on a large number of rows in the same table, it
should be compiled.  Use the table's \method{compile} method, which sets
the symbol types according to the table's schema and performs other
necessary expansions before compiling the expression.  The compiled
expression object behaves just like the original expression object,
except that it runs faster.
\begin{verbatim}
>>> cm = tracks.compile(ex)
>>> print cm.evaluate(tracks[0])
0.848292267373
\end{verbatim}
You may also pass an expression as a string to \method{compile}.
Functions provided by \pyhep which work with expressions, such as a
table's \method{select} method or \function{hep.hist.project} will
compile expressions automatically, where possible.  


\section{Expression syntax}

An expression is specified using Python's ordinary expression syntax,
with the following assumptions:
\begin{itemize}
 \item Arbitrary names may be used in expressions as variable
 quantities, functions, etc.  Other than the built-in names listed
 below, all names must be resolved when the expression is evaluated.

 \item The forward-slash operator for integers is true division, i.e. it
 produces a \code{float} quotient.  Use the double forward-slash
 operator (e.g. ``x // 3'') to obtain the C-style truncated integer
 division.
\end{itemize}

The following names are recognized in expressions:
\begin{itemize}
 \item Built-in Python constants \code{True}, \code{False}, and
 \code{None}.  

 \item Built-in Python types \code{int}, \code{float}, \code{complex},
 and \code{bool}.

 \item Built-in Python functions \code{abs}, \code{min}, and \code{max}.

 \item Constants from the \module{math} module: \code{e} and \code{pi}.

 \item Functions from the \module{math} module: \code{acos},
 \code{asin}, \code{atan}, \code{atan2}, \code{ceil}, \code{cos},
 \code{cosh}, \code{exp}, \code{floor}, \code{log}, \code{sin},
 \code{sinh}, \code{sqrt}, \code{tan}, and \code{tanh}.

 \item From the \module{hep.lorentz} module, \code{Frame} and
 \code{lab}.
\end{itemize}

In addition, expressions may use these numerical convenience functions.
(They are also available in Python programs in the \module{hep.num}
module.)

\begin{funcdesc}{gaussian}{mu, sigma, x}
Returns the probability density at \var{x} from a gaussian PDF with mean
\var{mu} and standard deviation \var{sigma}.
\end{funcdesc}

\begin{funcdesc}{get_bit}{value, bit}
Returns true iff. bit \var{bit} in \var{value} is set.
\end{funcdesc}

\begin{funcdesc}{hypot}{*terms}
A generalization of \function{math.hypot} to arbitrary number of
arguments.  Returns the square root of the sum of the squares of its
arguments.
\end{funcdesc}

\begin{funcdesc}{if_then}{condition, value_if_true, value_if_false}
Returns \var{value_if_true} if \var{condition} is true,
\var{value_if_false} otherwise.  Note that in a compiled expression
(only), the second and third arguments are evaluated lazily, so that
if \var{condition} is true, \var{value_if_false} is not evaluated, and
otherwise \var{value_if_true} is not evaluated.
\end{funcdesc}

\begin{funcdesc}{near}{central_value, half_interval, value}
Returns true if the absolute difference between \var{central_value} and
\var{value} is less than \var{half_interval}. 
\end{funcdesc}


\section{Other ways to make expressions}

You may specify a constant instead of a string when constructing an
expression with \function{asExpression} or \function{compile}.  The
resulting expression simply returns the constant.
\begin{verbatim}
>>> ex = hep.expr.asExpression(15)
>>> print ex
15
>>> print ex.type
<type 'int'>
\end{verbatim}

You may also specify a function that takes only positional arguments.
The resulting expression calls this function, using symbols for the
function arguments matching the parameter names in the function
definition.  If the function has a parameters \code{xyz}, the expression
will evaluate the symbol \code{xyz} and call the function with this
value.  For example,
\begin{verbatim}
>>> def foo(x, a):
...   return x ** a
...
>>> ex = hep.expr.asExpression(foo)
>>> print ex
foo(x, a)
>>> ex(a=8, x=2)
256
\end{verbatim}

The type of the value returned from such a function is not known. 
\begin{verbatim}
>>> ex = hep.expr.asExpression(foo)
>>> print ex.type
None
\end{verbatim}
You may specify it by attaching an attribute \member{type} to the
function, containing the expected type of the function's return value.
\begin{verbatim}
>>> foo.type = float
>>> ex = hep.expr.asExpression(foo)
>>> print ex.type
<type 'float'>
\end{verbatim}

You may also construct expressions pragmatically from the classes used
by \pyhep to represent expressions internally.  Each class represents a
single operation.  Invoke \code{help(hep.expr.classes)} for a list of
these classes and their interfaces.

Here's an example to give you an idea.
\begin{verbatim}
>>> from hep.expr import Add, Divide, Constant, Symbol
>>> mean = Divide(Add(Symbol("a"), Symbol("b")), Constant(2))
>>> print mean
(a + b) / 2
>>> mean(a=16, b=20)
18.0
\end{verbatim}



\chapter{Files and directories}

\pyhep provides a uniform interface for accessing and manipulating
directories and their contents.  The same interface can be used not only
with file system directories, but directories in Root and HBOOK files as
well.  

Directory objects satisfy Python's map protocol, very similar to
built-in \code{dict} objects.  The keys in a directory are the names of
items in the directory.  To find the names in a directory, simply call
the \method{keys} method.  For supported file types (discussed below),
the corresponding values are the contents of the files.  To load a file,
simply get the value using the subscript operator (square brackets) or
\method{get} method.

Here are some examples of using \pyhep's directory objects.  Suppose we
are in a directory with these contents, which are various types of files
containing 
\begin{verbatim}
> ls -l
total 16
-rw-r-----  1 samuel samuel 1257 Dec  6 15:39 histogram.pickle
-rw-r-----  1 samuel samuel 3980 Dec  6 15:39 plot.pickle
-rw-r-----  1 samuel samuel   50 Dec  7 19:11 readme.txt
drwxr-x---  2 samuel samuel 4096 Dec  6 15:43 recodata
> ls -l recodata/
total 12
-rw-r-----  1 samuel samuel 4384 Dec  6 15:43 reco.root
-rw-r-----  1 samuel samuel  683 Dec  6 15:34 tracks.table
\end{verbatim}

Now, let's start Python.  First, we import the file system directory
module, \module{hep.fs}, and construct a directory object, an instance
of \class{FileSystemDirectory}, for the current working directory.
\begin{verbatim}
>>> import hep.fs
>>> cwd = hep.fs.getdir(".")
>>> cwd
FileSystemDirectory('/home/samuel/data', writable=True)
\end{verbatim}

Using the \method{keys} method, we can get the names of files in the
directory.
\begin{verbatim}
>>> print cwd.keys()
['readme.txt', 'plot.pickle', 'recodata', 'histogram.pickle']
\end{verbatim}

Directory objects have an additional \method{list} method, which prints
out directory entries and their types.
\begin{verbatim}
>>> cwd.list()
histogram.pickle    : pickle
plot.pickle         : pickle
readme.txt          : text
recodata            : directory
\end{verbatim}

Directory methods that produce or operate on the keys in the directory
can be given extra arguments.  For instance, the \code{recursive} option
will show keys in subdirectories as well.
\begin{verbatim}
>>> cwd.list(recursive=True)
histogram.pickle            : pickle
plot.pickle                 : pickle
readme.txt                  : text
recodata                    : directory
recodata/reco.root          : Root file
recodata/reco.root/hist1    : 1D histogram
recodata/reco.root/hist2    : 1D histogram
recodata/tracks.table       : table
\end{verbatim}
Notice here that \method{list} now lists the contents of the
\file{recodata} subdirectory.  One of the files in that directory is a
Root file named \file{reco.root}.  Since a Root file has an internal
directory structure, \pyhep treats it as a directory itself and the
listing descends into that file too.

Retrieving an object is as simple as looking up a key in a dictionary.
\pyhep determines the file type from its extension, and loads the object
into memory, creating a Python object of the appropriate type to
represent it.  For instance, a file with the extension \file{.pickle} is
assumed to contain a Python pickle (see the documentation for the
standard \module{pickle} and \module{cPickle} for details).  For
example, to retrieve the contents of \file{histogram.pickle},
\begin{verbatim}
>>> histogram = cwd["histogram.pickle"]
>>> histogram
Histogram(EvenlyBinnedAxis(30, (0.0, 30.0), name='energy', units='GeV'), bin_type=float, error_model='asymmetric')
\end{verbatim}

Obtain an object representing a subdirectory, whether an actual file
system subdirectory or a virtual subdirectory in a Root or HBOOK file,
in the same way.
\begin{verbatim}
>>> subdir = cwd["recodata"]
>>> print subdir.keys()
['reco.root', 'tracks.table']
>>> rootfile = subdir["reco.root"]
>>> print rootfile.keys()
['hist1', 'hist2']
\end{verbatim}
Likewise for retrieving an object from inside a Root file.
\begin{verbatim}
>>> histogram = rootfile["hist2"]
\end{verbatim}

You don't have to hang on to the intermediate directory objects if you
only want one object deep in a hierarchy of subdirectories, Root, and
HBOOK files.
\begin{verbatim}
>>> histogram = cwd["recodata"]["reco.root"]["hist2"]
\end{verbatim}
Even simpler, you can specify multiple levels with a single indexing
operation by separating keys with forward slashes.
\begin{verbatim}
>>> histogram = cwd["recodata/reco.root/hist2"]
\end{verbatim}
Note that you \emph{can not} use \file{..} to move up in the directory
tree, or specify absolute paths, as these operations break the model of
nested map objects.

To store an object to disk, simply assign a new item to the directory
object as you would with a dictionary.
\begin{verbatim}
>>> cwd["reco-hist.pickle"] = histogram
>>> print cwd.keys()
['readme.txt', 'plot.pickle', 'reco-hist.pickle', 'recodata', 'histogram.pickle']
\end{verbatim}
Of course, the file type inferred from the key's extension must support
storing the type of the value object you provide.  Python pickles can
store most Python objects, including collections such as tuples and
dictionaries, and most \pyhep objects.

Generally, the Python object is not associated with the file once it's
loaded.  If you change the Python object and want the changes reflected
in the file, you must store it back.  Directories are an exception to
this: if you add a new item to the directory, it goes immediately on
disk.  Tables are also an exception to this.

%-----------------------------------------------------------------------

\section{File types}

When you retrieve a key from a file system directory (\textit{i.e.} a
directory object corresponding to an actual directory in the file
system, not a directory in a Root or HBOOK file), \pyhep examines the
key's extension to determines how to handle the value.  

Each type has an associated type name.  When retrieving or storing items
in a directory, you may override the determination of the file type with
the keyword argument \code{type}, specifying the type name.  For
example, if you have a Root file named \file{histograms.dat} in a
directory, you may retrieve it using
\begin{verbatim}
>>> histograms = directory.get("histograms.dat", type="Root file")
\end{verbatim}

The file types understood by \pyhep, and their corresponding extensions,
are listed below.

\begin{itemize}
 \item \code{"directory"} (no extension): A directory.  The value is a
 directory object.  

 \item \code{"HBOOK file"} (extension \file{.hbook}): An HBOOK file.
 The value is a directory object representing the root of the RZ
 directory tree inside the file.  Additional information about HBOOK
 files is presented below.

 \item \code{"pickle"} (extension \file{.pickle}): A file containing a
 Python pickle.  The value is whatever Python object was pickled.

 \item \code{"Root file"} (extension \file{.root}): A Root file.  The
 value is a directory object representing the root of the directory tree
 inside the file.  Additional information about Root files is presented
 below.

 \item \code{"symlink"} (no extension): A symbolic link.  The value is a
 string containing the target of the link.  Since this type has no
 associated extension, it can only be used with the \code{type} keyword
 argument described above.

 \item \code{"table"} (extension \file{.table}): A \pyhep table.  The
 value is a handle to the open table.  In contrast to how other file
 types are handled, the table is not loaded into memory.  Changes to the
 table are reflected in the table file.

 \item \code{"text"} (extension \file{.txt}): A text file.  The value is
 a character string of the file's contents.

 \item \code{"unknown"} (all other extensions): Represents all file
 types not recognized by \pyhep.
\end{itemize}

Keys in Root and HBOOK directories are handled differently.  The file
types corresponding to these keys is determined from metadata stored in
the files themselves, not from an extension.  Additional file types are
support in these directories, including \code{"1D histogram"} and
\code{"2D histogram"}.

To determine the file type of a key, use the directory's
\method{getinfo}, method described below.

%-----------------------------------------------------------------------

\section{Methods for accessing keys and values}

As we say above, the \method{keys} method lists all keys in the
directory.  Directory objects also support the standard \method{values}
and \method{items} methods, as well as \method{iterkeys},
\method{itervalues}, and \method{iteritems}.  

For all of these, you may use these keyword arguments to restrict the
keys, values, or items that are returned:
\begin{itemize}
 \item \code{recursive}: If true, include recursively the contents of
 subdirectories. 

 \item \code{not_dir}: If true, don't include items that are directories
 (either in the file system or inside Root or HBOOK files).  

 \item \code{pattern}: Only include items whose keys match the specified
 regular expression.  See the \module{re} module for a description of
 Python's regular expression syntax.

 \item \code{glob}: Only include items whose keys match the specified
 glob pattern.  See the \module{glob} module for a description of
 Python's glob syntax.

 \item \code{only_type}: Only include an item if its type is as
 specified.  Item types are specified by strings; see below for more
 information.

 \item \code{known_types}: True by default, which specifies that only
 items of known types are included.  If you set this to false, all items
 are included.  Note that \pyhep will raise an exception if you try to
 access the value corresponding to a key of unknown type.

\end{itemize}

These options can also be used with \method{list}, which prints a
listing of keys in a directory and their types.

For example,
\begin{verbatim}
>>> print cwd.keys(only_type="pickle")
['plot.pickle', 'reco-hist.pickle', 'histogram.pickle']
>>> cwd.list(glob="hist*", recursive=True)
histogram.pickle            : pickle
recodata/reco.root/hist1    : 1D histogram
recodata/reco.root/hist2    : 1D histogram
\end{verbatim}

As with a dictionary, the \function{len} function returns the number of
keys in the directory.  
\begin{verbatim}
>>> print len(cwd)
5
\end{verbatim}
Iterating over a directory object iterates over its keys.  
\begin{verbatim}
>>> for key in cwd:
...   print key
...
readme.txt
plot.pickle
reco-hist.pickle
recodata
histogram.pickle
\end{verbatim}
The \code{has_key} method and \code{in} operator return true if the
specified key is in the directory.
\begin{verbatim}
>>> print cwd.has_key("readme.txt")
True
>>> print "missing.pickle" in cwd
False
\end{verbatim}

To retrieve an object, use the Python indexing notation (square
brackets) or the \method{get} method.  If the key is not found in the
dictionary, indexing will raise \class{KeyError}; the \method{get}
method will return a default value, which you can specify as the second
argument (the default is \code{None}).
\begin{verbatim}
>>> print cwd["missing.pickle"]
Traceback (most recent call last):
...
KeyError: 'missing.pickle'
>>> print cwd.get("missing.pickle")
None
>>> print cwd.get("missing.pickle", 42)
42
\end{verbatim}

Use the \function{hep.fs.getdir} function to return a directory object
for an arbitrary path.  The path may be absolute or relative to the
current working directory.  The path may descend into Root and HBOOK
files.  For example,
\begin{verbatim}
>>> data_dir = hep.fs.getdir("/nfs/data")
>>> histograms = hep.fs.getir("histograms.root/reco")
\end{verbatim}
You can also use \function{hep.fs.get} to load files of other types by
specifying the path to the file.

%-----------------------------------------------------------------------

\section{Accessing file information}

A directory object's \method{getinfo} method returns an \class{Info}
object containing information about the file.  The \class{Info} object
is constructed without loading the file.

All \class{Info} objects have an attribute \member{type}, which is the
file type for that file.  See above for a discussion of file types.  An
\class{Info} object may have additional attributes, depending on the
type of directory object it was obtained from.

File system \class{Info} objects also contain attributes \member{path},
\member{file_size}, \member{user_id}, \member{group_id},
\member{access_mode}, and \member{modification_time}.

The code below demonstrates the use of \method{getinfo} to print a
listing of a file system directory.
\begin{verbatim}
>>> def listFSDir(directory):
...   for key in directory.keys(known_types=False):
...     info = directory.getinfo(key)
...     print ("%-24s (%-12s)  %8d bytes, mode %06o, owner %4d"
...            % (key, info.type, info.file_size, info.access_mode, info.user_id))
...
>>> listFSDir(cwd) 
readme.txt               (text        )        50 bytes, mode 000640, owner  500
plot.pickle              (pickle      )      3980 bytes, mode 000640, owner  500
reco-hist.pickle         (pickle      )      1252 bytes, mode 000640, owner  500
recodata                 (directory   )      4096 bytes, mode 000750, owner  500
histogram.pickle         (pickle      )      1257 bytes, mode 000640, owner  500
\end{verbatim}

%-----------------------------------------------------------------------

\section{Storing data}

To store data in a file, simply set a key in the dictionary object.  For
file system directories, the file type is determined from the key's
extension.  The file type must support the data type you provide.  For
example,
\begin{verbatim}
>>> text = "Hello, world."
>>> cwd["hello.txt"] = text
>>> cwd["hello.pickle"] = text
>>> cwd["hello.root"] = text
Traceback (most recent call last):
...
AttributeError: 'str' object has no attribute 'keys'
\end{verbatim}
Here, the first assignment stores the text in a plain text file, and the
second stores it as a pickled Python string.  The third assignment
fails, since it doesn't make sense to create a Root file with a string.

You can also use the \method{set} method to set keys.  With
\method{set}, you can specify keyword arguments.  For example, you can
override the file type determination with the \code{type} argument.
For example, this assignment tell \pyhep to store a plain text file,
even though the extension \file{.log} is not known.
\begin{verbatim}
>>> cwd.set("hello.log", text,type="text")
\end{verbatim}

You can other standard map methods to modify the directory's contents as
well: \method{setdefault}, \method{update}, and \method{popitem}.  To
delete files, use the \code{del} statement or the directory's
\method{delete} method.  The \method{clear} method empties the
directory.

These methods can take additional keyword arguments that controls how
directories are modified:
\begin{itemize}
 \item \code{deldirs} (true by default): Allow automatic recursive
 deletion of directories and their contents.

 \item \code{makedirs} (true by default): Create missing intermediate
 subdirectories automatically when setting a key (like ``\code{mkdir
 -p}''). 

 \item \code{replace} (true by default): If true, allow keys to be
 replaced.  Otherwise, raise an exception when setting a key that
 already exists.

 \item \code{replacedirs} (false by default): Like \code{replace} but
 applies to subdirectories.
\end{itemize}

You can also create a directory by setting a key.  Since a directory is
represented by a map, it looks like this:
\begin{verbatim}
>>> cwd["subdir"] = {}
\end{verbatim}
You may populate the map you assign; the items are stored as files in
the new directory.
\begin{verbatim}
>>> cwd["subdir"] = { "contents.txt": "calibration constants", 
...                   "constants.pickle": (10, 11, 12) }
\end{verbatim}
To create a directory if it doesn't exist, or obtain it if it does, use
\method{setdefault}:
\begin{verbatim}
>>> output_dir = cwd.setdefault("output", {})
\end{verbatim}

%-----------------------------------------------------------------------

\section{Working with HBOOK files}

You can use directory objects, as described above, to access contents of
HBOOK files.  An HBOOK file is represented by a directory object, as is
a subdirectory in an HBOOK file.

For example, to create a new HBOOK file in the directory represented by
\code{data_dir} containing two histograms,
\begin{verbatim}
>>> data_dir["histograms.hbook"] = { "hist1": histogram1,
...                                  "hist2": histogram2 }
\end{verbatim}
To create a new, empty HBOOK file or return the existing one if it
exists,
\begin{verbatim}
>>> hbook_file = data_dir.setdefault("histograms.hbook", {})
\end{verbatim}

\subsection{The \module{hep.cernlib.hbook} module}

The \module{hep.cernlib.hbook} module contains \pyhep's implementation
of HBOOK file access.  HBOOK is a part of the CERNLIB library, and
provides a structured file format containing histograms and n-tuples.
Note that not all HBOOK features are supported.

Some details particular to HBOOK directory objects:
\begin{itemize}
 \item \pyhep's module \module{hep.cernlib.hbook} is linked statically
 against CERNLIB 2002.

 \item HBOOK's names are case-insensitive.  \pyhep always uses lower
 case for names in HBOOK files.

 \item To load a histogram from an HBOOK directory, simply obtain it by
 name using the subscript operator or \method{get}.  This returns a
 Python object representing the histogram, which may be modified freely.
 Note that unlike HBOOK itself, where histograms are always stored in a
 global ``PAWC'' memory region, \pyhep constructs ordinary Python
 objects for histograms.  There is no need to manage ``PAWC''
 explicitly.

 \item Not all histogram features supported in HBOOK are also supported
 in \pyhep, and visa versa.  Therefore, if a histogram is saved to an
 HBOOK file and later loaded, it may differ in some of its
 characteristics.  The basic histogram binning, and bin contents and
 errors (including overflow and underflow bins), are stored correctly,
 however.  Note that \pyhep does not provide profile histograms.

 \item An HBOOK file is not closed until the file object is destroyed,
 i.e. all references to it are released.  Especially when writing an
 HBOOK file, be careful to release all references to the file object.

 \item When \pyhep closes an HBOOK file, it ``purges cycles'',
 \textit{i.e.} removes old revisions of all entries stored in the file.
 To disable this behavior, specify \code{purge_cycles=False} as a
 keyword argument when creating or accessing the HBOOK file.

 \item In an HBOOK file, each object is given not only a name
 identifying it in the directory that contains it, but a numerical RZ
 identification as well.  When you retrieve an object from an HBOOK file,
 \pyhep stores this value in the object's \member{rz_id} attribute.
 When you store an objet from an HBOOK file, you may specify the value
 to use either by setting the object's \member{rz_id} attribute or by
 using an \code{rz_id} keyword argument; otherwise, \pyhep chooses an
 available value.  An info object obtained from an HBOOK directory's
 \method{getinfo} method also has an \member{rz_id} attribute.

\end{itemize}


\subsection{N-tuples}

An HBOOK n-tuple is represented in \pyhep by a table.  The table
satisfies the same protocol as the default table implementation (see
\module{hep.table}), except in the method to create or open tables.
Note that because of HBOOK's limitations, certain table features are not
supported.

To access an n-tuple in an HBOOK file, use the file object's subscript
operator or \method{get} method, just access the n-tuple's name, just as
you would for a histogram.  Unlike a histogram, the table object is
still connected to the n-tuple in the HBOOK file.  A new row appended to
the table is incorporated immediately into the n-tuple.  Also, the table
object carries a reference to the HBOOK file, in its \member{file}
attribute, so the HBOOK file is not closed as long as there is an
outstanding table object for an n-tuple in the file.

To create a new n-tuple in an HBOOK file, use the
\function{hep.cernlib.hbook.createTable} function.  The arguments are the
name of the n-tuple, the HBOOK directory object in which to create the
n-tuple, and the schema (as for \function{hep.table.create}).  You may
use the optional \var{rz_id} argument to specify the n-tuple's RZ ID.

By default, a column-wise n-tuple is used for the table; to create a
row-wise n-tuple, pass a false value for the optional \var{column_wise}
argument to \function{createTable}.  When creating a column-wise n-tuple,
the schema may only contain columns of types \constant{"int32"},
\constant{"int64"}, \constant{"float32"}, and \constant{"float64"}.  The
schema for a row-wise n-tuple may use only \constant{"float32"} columns.

This program creates an HBOOK file containing a row-wise n-tuple filled
with random values.  It then re-opens the file, creates a histogram from
the values, and stores it in the file.
\codesample{hbook1.py}

%-----------------------------------------------------------------------

\section{Working with Root files}

Root set of libraries and programs for high energy physics analysis.
Among other things, Root provides a file format for histograms, n-tuples
(which are called ``trees'' in Root), and other data objects.  \pyhep
provides partial data and file compatibility with Root.

You can use directory objects, as described above, to access contents of
Root files.  An Root file is represented by a directory object, as is a
subdirectory in an Root file.

For example, to create a new Root file in the directory represented by
\code{data_dir} containing two histograms,
\begin{verbatim}
>>> data_dir["histograms.root"] = { "hist1": histogram1,
...                                 "hist2": histogram2 }
\end{verbatim}
To create a new, empty Root file or return the existing one if it
exists,
\begin{verbatim}
>>> root_file = data_dir.setdefault("histograms.root", {})
\end{verbatim}

The API and capabilities of \module{hep.root} are very similar to those
of \module{hep.cernlib.hbook}.  A program written for one can be used
with the other with minimal modification, and it is easy to write
functions, scripts, or programs that can work files from either format.


\subsection{The \module{hep.root} module}

The \module{hep.root} contains \pyhep's implementation of Root file
access.  

\pyhep's module \module{hep.root} is built and linked against Root shared
libraries which are distributed with \pyhep.  The module does not depend
on any other version of Root installed on your system.

Some details particular to Root directory objects:
\begin{itemize}
 \item A Root file is not closed until the file object is destroyed,
 i.e. all references to it are released.  Especially when writing a Root
 file, be careful to release all references to the file object.

 \item Not all histogram features supported in Root are also supported
 in \pyhep, and visa versa.  Therefore, if a histogram is saved to a
 Root file and later loaded, it may differ in some of its
 characteristics.  For instance, any additional attributes added to the
 histogram will be lost.  The basic histogram binning, and bin contents
 and errors (including overflow and underflow bins), are stored
 correctly, however.

 \item In a Root file, each object is given not only a name identifying
 it in the directory that contains it, but a title.  When you retrieve
 an object from an Root file, \pyhep stores this value in the object's
 \member{title} attribute.  When you store an object from an Root file,
 you may specify the title to use either by setting the object's
 \member{title} attribute or by using an \code{title} keyword argument.
 An info object obtained from an Root directory's \method{getinfo}
 method also has an \member{title} attribute.

\end{itemize}


\subsection{Trees}

An Root tree is represented in \pyhep by a table.  The table satisfies
the same protocol as the default table implementation (see
\module{hep.table}), except in the method to create or open tables.
Note that because of Root's limitations, certain table features are not
supported.

To open a tree in a Root file as a table, use the file object's
subscript operator or \method{get} method, just as you would for a
histogram.  Note that unlike a histogram returned from \method{load},
though, the table object that \method{load} returns is still connected
to the tree in the Root file.  A new row appended to the table is
incorporated in the tree.  Also, the table object carries a reference to
the Root file, in its \member{file} attribute, so the Root file is not
closed as long as there is an outstanding table object for a tree in the
file.

To create a new tree in a Root file, use the
\function{hep.root.createTable} function.  The arguments are the name of
the new tree, the Root directory in which to create it, and the schema
(as for \function{hep.table.create}).  You may specify a title for the
table with the \var{title} argument.

This program creates a Root file containing a row-wise n-tuple filled
with random values.  It then re-opens the file, creates a histogram from
the values, and stores it in the file.
\codesample{root1.py}


\chapter{Drawing}

\pyhep includes a drawing layer that provides device-independent output
for simple two-dimensional figures and line drawings.  Classes and
functions for producing plots use the drawing layer; these are described
in the next chapter.  The drawing layer is in the module
\module{hep.draw}.

Drawings, plots, and the like are represented in \pyhep by
\textit{figures}.  A figure is a device-independent memory
representation of a drawing in a rectangular region, including all
visual attributes---the \textit{style}---of the drawing.  

A \textit{renderer} produces output by rendering a figure.  Each
renderer represents a particular output channel, such as a display
window or an output file.  The drawing layer supports output to an X11
window, and generation of PostScript files, enhanced Windows metafiles,
and bitmap files.

%-----------------------------------------------------------------------

\section{Units and types}

The unit of measurement in the \pyhep drawing layer is meters,
regardless of the output device; \pyhep attempts to preserve the scale
of output.  All measurements are given as 'float' values.

The module \module{hep.draw} includes the conversion factors
\code{point} and \code{inch} to help you use these units.  For instance,
the length \code{12 * point} is about 17 cm, or a sixth of an inch.

Colors are represented by the \class{Color} class.  Its constructor
arguments are red, green, and blue color components, each between zero
and one.  The function \function{Gray} produces a shade of gray; it
produces a color with the same value for all three components.  The
function \function{HSV} constructs a color from hue, saturation, and
value components.  For example,
\begin{verbatim}
>>> from hep.draw import *
>>> dark_red = Color(0.6, 0.1, 0.1)
>>> medium_gray = Gray(0.6)
>>> print medium_gray
Color(0.600000, 0.600000, 0.600000)
>>> aqua = HSV(0.5, 0.8, 0.5)
>>> print aqua
Color(0.100000, 0.500000, 0.500000)
\end{verbatim}
Also, \module{hep.draw} includes the constants \code{black} and
\code{white}.

The appearance of a line is given by its color, thickness, and dash
pattern.  Specify the thickness in meters as with any other length.  A
dash pattern is a tuple of lengths, that specifies the lengths of
alternating ``on'' and ``off'' segments.  For instance, this dash
pattern produces a line with alternating long (2 mm) and short (1 mm)
dashes, with a fixed small distance (0.5 mm) between the dashes.
\begin{verbatim}
>>> dash = (0.002, 0.0005, 0.001, 0.0005)
\end{verbatim}
For a solid line, specify \code{None} as the dash pattern.  You can also
use these constants for predefined dash patterns: \code{"solid"},
\code{"dot"}, \code{"dash"}, \code{"dot-dash"}, and
\code{"dot-dot-dash"}.

%-----------------------------------------------------------------------

\section{Canvases}

Not documented yet.

%-----------------------------------------------------------------------

\section{Renderers}

A renderer draws a figure to a display or to an output file.  Call the
\method{render} method, passing the figure to render.

\pyhep provides renderers for generating PostScript files, for enhanced
Windows metafiles, bitmap files, and for displaying in X windows.

\subsection{PostScript renderers}

The module \module{hep.draw.postscript} provides renderers for
PostScript files.  

An instance of class \class{PSFile} generates a multi-page ADSC
PostScript file.  Pass the constructor the path to the output file.  

You may specify the page size using the \code{page_size} keyword
argument.  The page size is either a \code{(width, height)} pair.  You
can also specify a page size by name, such as \code{"letter"} or
\code{"A4"}.  Named pages sizes are stored in the dictionary
\code{hep.draw.page_size}.  If the page size is omitted, letter is
assumed.

Each time you invoke the \method{render} method of a \class{PSFile}
object, you generate a new page in the PostScript.  When you are done,
simply delete the \class{PSFile} object to close the file.

This example shows the generatation of a two-page PostScript file.
\begin{verbatim}
>>> import hep.draw.postscript
>>> ps_file = hep.draw.postscript.PSFile("output.ps", page_size="letter")
>>> ps_file.render(figure1)
>>> ps_file.render(figure2)
>>> del ps_file
\end{verbatim}

The class \class{EPSFile} generates an encapsulated PostScript file.
Specify the bounding box size of the image using the \code{size} keyword
argument.  You should only invoke the \method{render} method
once per \class{EPSFile} object.

\subsection{Enhanced metafile renderer}

The module \module{hep.draw.metafile} provides a render for files in the
enhanced Windows metafile (EMF) format.  This is a vector file format
widely supported by Windows applications.  An EMF file essentially
contains a recording of the Windows graphics system calls required to
draw a figure.  The standard file extension for EMF files is
``\code{.emf}''.

The \class{EnhancedMetafile} class renders figures into EMF files.  As
with \class{PSFile}, specify the path to the output file and the image
size when creating an \class{EnhancedMetafile} object.  The
\method{render} method should only be invoked once.

This is how you can quickly save a figure as an EMF file.
\begin{verbatim}
>>> from hep.draw.metafile import EnhancedMetafile
>>> EnhancedMetafile("figure.emf", (0.06, 0.04)).render(figure)
\end{verbatim}

\subsection{Bitmap file renderer}

The module \module{hep.draw.imagefile} provides a function
\function{render} to render a figure as a bitmap image file.  The bitmap
is written using the Imlib library, and all bitmap file types supported
by Imlib can be used.  Depending on how Imlib is installed on your
system, different image file formats will be supported, but at least
PNG, JPG, and TIFF are generally available.  PNG is the recommended
format for line art such as plots of histograms.

Call \function{render} specifying the figure to render and save, the image
file name, and the size of the image in pixels.  The file format is
determined automatically from the file extension.  For example,
\begin{verbatim}
>>> from hep.draw import imagefile
>>> imagefile.render(figure, "figure.png", (640, 480))
\end{verbatim}
You may also specify the virtual size in which to render the figure,
using the \code{virtual_size} argument.

If you specify only the size or only the virtual size, the image other
is computed using the resolution specified by the \code{resolution}
argument, which defaults to 75 dpi.  For example, if you want the image
file to be 5 cm square, assuping 96 dpi,
\begin{verbatim}
>>> imagefile.render(figure, "figure.dpi", virtual_size=(0.05, 0.05),
...                resolution=96/inch)
\end{verbatim}

\subsection{X window renderer}

The class \class{hep.draw.xwindow.Window} is a window that can render
figures in an X window.  Each instance creates a new window.  

When you create a new \class{Window} instance, provide the desired
window size \textit{in meters} when creating a new window.  You can
resize an existing window with its \method{resize} method, or simply by
resizing the window through your window manager.  When creating a
\class{Window}, you may also provide a title for the title bar provided
by your window manager, or set this later by assigning the
\member{title} attribute.

Each time you call the \method{render} method, the window's contents are
replaced by the specified figure.  Note that if the window's contents
are destroyed (for example, if you resize the window), you will have to
call \method{render} again to restore the window's contents.  Also, if
you modify the figure, the window contents do not reflect the changes
until you call \method{render} again.

\class{Window} uses anti-aliasing and subpixel interpolation when
rendering its contents.

The \class{Window} class renders figures in memory and then transfers
the rendered image to your display.  This allows the renderer to use
advanced anti-aliasing and subpixel rendering, and makes sure fonts are
handled uniformly regardless of the configuration of the X server
displaying the window.  However, this means that when you run \pyhep in
a remote X session, the contents of the window are transferred over the
network as a bitmap, which can be slow.  If you tunnel your X connection
through SSH, you will probably find that using compression
(``\code{-C}'' on the \code{ssh} command line) improves display
performance.

\subsection{Figure windows}

A \class{hep.draw.xwindow.FigureWindow} provides a more convenient way
of displaying a figure in a window.  The \class{FigureWindow} knows the
figure it is rendering, and redraws its contents when necessary (when
the window is exposed or resized).

Using a \class{FigureWindow}, it's easy to display a figure.  The second
argument is the window size.
\begin{verbatim}
>>> from hep.draw.xwindow import FigureWindow
>>> window = FigureWindow(figure, (0.16, 0.10))
\end{verbatim}

You can change the figure displayed in the window by setting the
\member{figure} attribute.  If the value is \code{None}, nothing is
drawn in the window.  If the figure changes, call the \method{redraw}
method to force the window to redraw it.

%-----------------------------------------------------------------------

\section{Layouts}

A \textit{layout} is a composite figure that arranges several other
figures.  A layout object is itself a figure, so it can be rendered
directly by renderers, and included in other layouts.  The layout
objects described here are in the \module{hep.draw} module.

A simple layout class is \class{SplitLayout}.  It arranges two figures
next to each other, either horizontally or vertically.  A
\class{SplitLayout} allows you to specify the fractions of the entire
drawing region in which to draw each of the two figures.  When you
create a split layout, specify the orientation of the split, either
\code{"vertical"} or \code{"horizontal"}, and the two figures.  Either
figure may be \code{None}.  You may also specify the fraction of the
first figure (the default is 0.5); the remainder is used to draw the
second figure.

For example, this code creates a layout displaying \code{fig1} on the
left, occupying three-quarters of the layout, and \code{fig2} on the
right, occypting the remaining quarter.
\begin{verbatim}
>>> from hep.draw import *
>>> layout = SplitLayout("vertical", fig1, fig2, fraction=0.75)
\end{verbatim}

You can create more flexible layouts with a \class{BrickLayout} object.
A \textit{brick layout} resembles a row of bricks: the region is divided
into equally-spaced rows, each of which is divided into equally-spaced
cells.  Each row may have a different number of cells.  To specify the
arrangement of cells, provide a sequence containing the number of cells
in each row.  The number of elements in the sequence is the number of
rows.  For instance, to create a layout with three rows, of which the
first and third are divided in half and the middle is divided into
thirds,
\begin{verbatim}
>>> layout = BrickLayout((2, 3, 2))
\end{verbatim}
You may also specify style attributes as keyword arguments; as with
other figures, a layout's style attributes are stored in a dictionary
attribute \member{style}.  For example, the \code{margin} style
attribute controls the size of an empty margin inserted between rows and
cells in a row.
\begin{verbatim}
>>> layout = BrickLayout((2, 3, 2), margin=12*point)
\end{verbatim}

You can index the figures in a brick layout by column and row index, for
instance 
\begin{verbatim}
>>> layout[0, 0] = fig1
>>> layout[1, 2] = fig2
\end{verbatim}
Any entry that contains \code{None} (the default) is left empty.

A \class{GridLayout} is a subclass of \class{BrickLayout} in which all
rows have the same number of cells.  This produces a grid of
evenly-sized cells.  Specify the number of columns and rows when
creating an instance.
\begin{verbatim}
>>> layout = GridLayout(2, 3)
\end{verbatim}

%% All layout classes also have an attribute \member{figures}, which is a
%% list of figures displayed in the layout.  The order of figures depends
%% on the layout.  Also, the \member{appender} method provides convenient
%% way to set figures sequentially in a complex layout.  It produces a
%% function that, when called repeated, sets sequential positions of
%% \member{figures} to its argument.  Each \member{appender} filles the
%% figures starting from index zero, so you should hang on to it and call
%% the same one repeatedly.  For example,
%% \begin{verbatim}
%% >>> append_figure = layout.appender
%% >>> append_figure(fig1)
%% >>> append_figure(fig2)
%% \end{verbatim}

\chapter{Plotting}

\pyhep includes classes and function for producing plots of histograms,
functions, and scatter plots.  The \class{hep.hist.plot.Plot} class is
the basic class used for producing plots.  An instance of \class{Plot}
is a figure object and can be used with renderers and layouts, as
described in the previous chapter.

Here's a basic example.  First, create a histogram:
\begin{verbatim}
>>> import hep.hist
>>> hist = hep.hist.makeSample1D("flat", 500)
\end{verbatim}
Now create a \class{Plot} object, which is a figure depicting the
histogram.
\begin{verbatim}
>>> from hep.hist.plot import Plot
>>> plot = Plot(1, hist)
\end{verbatim}
When creating the plot, you specified the number of dimensions
(independent variables) that are depicted in the plot, and the histogram
we want to plot.  Finally, create a window, and render the plot in
the window.
\begin{verbatim}
>>> from hep.draw.xwindow import Window
>>> window = Window((0.2, 0.1))
>>> window.render(plot)
\end{verbatim}
The argument to \class{Window} is the width and height of the window in
meters. 

By default, a bin of a one-dimensional histograms is drawn as crosses.
The horizontal line shows the bin value and bin width, and the vertical
line shows the range of errors on the bin content.  On the left and
right sides are displayed the underflow and overflow bins, labelled
``UF'' and ``OF'' respectively.

You can just as easily generate a PostScript file containing the plot.
Just use a PostScript renderer instead of the \code{Window} object.
\begin{verbatim}
>>> from hep.draw.postscript import PSFile
>>> ps_file = PSFile("plot.ps")
>>> ps_file.render(plot)
>>> del ps_file
\end{verbatim}
By deleting the \class{PSFile} object, you generate the output and close
the file. 

%-----------------------------------------------------------------------

\section{Plots and series}

A \class{hep.hist.plot.Plot} object is a figure that displays a plot of
histograms, functions, and similar data.  A plot can display either data
with one independent variable, such as one-dimensional histograms and
functions of one variable, or data with two independent variables, such
as two-dimensional histograms and scatter plots.  More than one
histogram, function, or scatter plot can be shown at the same time; each
is called a \textit{series}.  All series in the same plot share the same
vertical and horizontal axes (and z range, for two-dimensional plots).

To create a plot object, specify the number of dimensions in the plot,
either 1 or 2.  You may specify the series to plot as additional
arguments.  For example, to create a plot overlaying a histogram and a
function,
\begin{verbatim}
>>> from hep.hist.plot import Plot
>>> plot = Plot(1, histogram, function)
\end{verbatim}
You can also use the \method{append} method to add a series to a plot,
so the following code is equivalent:
\begin{verbatim}
>>> plot = Plot(1)
>>> plot.append(histogram)
>>> plot.append(function)
\end{verbatim}

A \class{Plot} object has a \member{series} attribute, which is a list
of series in the plot.  The series are not the histograms and functions
themselves, but figure objects representing the series.  (Generally, you
wouldn't to render these figures directly.)  
\begin{verbatim}
>>> plot.series[0]
<hep.hist.plot.Histogram1DPlot object at 0xf6d01fcc>
\end{verbatim}
The plot object for a histogram references the histogram itself in its
\member{histogram} attribute.
\begin{verbatim}
>>> plot.series[0].histogram
Histogram(EvenlyBinnedAxis(20, (0.0, 1.0)), bin_type=int, error_model='poisson')
\end{verbatim}

You can remove series from a plot simply by removing the corresponding
element from the \member{series} list.  Similarly, you can reorder
elements in the list to change the stacking order of series in the
plot. 

%-----------------------------------------------------------------------

\section{Plot styles}

\textit{Styles} control the visual attributes of plots.  Style
information is not stored in a histogram, which only contains
statistical data and annotations (such as units).  Instead, style is
stored in a plot.

Each plot has a \member{style} attribute, which is a dictionary
containing style items.  Keys in the style dictionary are names of style
attributes.  Any style attributes may be stored in the style dictionary;
different plot types use different style attributes to control their
output.

A \class{Plot} object has a \member{style} attribute; in addition, the
plot object representing each series in a plot has a \member{style}
attribute as well.  This allows different styles to be used for each
series in a plot (for instance, to assign a different color to each
series).  If a particular style attribute is missing from a series's
style dictionary, it uses the value from the parent \class{Plot}
object's style dictionary.  Thus, if you set a style attribute in a
\class{Plot}, it will apply to each series in that plot, except for a
series which overrides it in its own style dictionary.

Suppose you have a plot showing two histograms.
\begin{verbatim}
>>> plot = Plot(1, histogram1, histogram2)
\end{verbatim}
To draw 2 mm dots at bin contents for both series in the plot, set the
\code{"marker"} and \code{"marker size"} attributes in the plot's
style dictionary.
\begin{verbatim}
>>> plot.style["marker"] = "filled dot"
>>> plot.style["marker_size"] = 0.002
\end{verbatim}
To draw the second series (but not the first) in red, set the
\code{"color"} attribute in that series's style dictionary.
\begin{verbatim}
>>> plot.series[1].style["color"] = hep.draw.Color(0.7, 0, 0)
\end{verbatim}

You can set style attributes for a \class{Plot} when you create it, by
adding keyword arguments.  Similarly, you can set style attributes for a
series by adding keyword arguments to \method{append}.  For example,
this code produces the same plot as the code above does:
\begin{verbatim}
>>> plot = Plot(1, marker="filled dot", marker_size=0.002)
>>> plot.append(histogram1)
>>> plot.append(histogram2, color=Color(0.7, 0, 0))
\end{verbatim}

The sections below list style attributes used by \pyhep's plot classes
to determine the visual style of plots.  For an explanation of how to
specify fonts, colors, marker styles, \textit{etc.}, see the chapter on
drawing. 

\subsection{Styles for plot objects}

These style attributes control the visual style of \code{Plot} objects.

\begin{itemize}
 \item \code{"bottom_margin"}: The size of the margin on the bottom of
 the plot.

 \item \code{"color"}: Specifies the color for the entire plot,
 including axes and labels.

 \item \code{"font"}: Specifies the font to use for titles and labels
 on the plot.

 \item \code{"font_size"}: Specifies the font size to use for titles and
 labels on the plot.

 \item \code{"left_margin"}: The size of the margin on the left side of
 the plot.

 \item \code{"log_scale"}: If true, a logarithmic scale is used for the
 dependent axis (the y axis in one-dimensional plots, the z range in
 two-dimensional plots).

 \item \code{"normalize_bin_size"}: How to normalize bin contents to a
 common bin size.  The contents of bins of different sizes are
 normalized to a common effective bin width (for one-dimensional
 histograms) or bin area (for two-dimensional histograms) to facilitate
 the interpretation of bin contents as an approximation of probability
 density.  The contents of overflow bins are never normalized.

 If the value of this style attribute is a number, the content of each
 bin is normalized to this bin size.  If the value is \code{"auto"} (the
 default), a bin size is automatically chosen (for a histogram with
 evenly-binned axes, the histogram's actual bin size is used).  If the
 value is \code{None}, bins are not normalized, and the actual values of
 bin contents are plotted.

 \item \code{"overflows"}: If true, show underflow and overflow bins for
 the x axis (for one-dimensional plots) or the x and y axis (for
 two-dimensional plots).  For two-dimensional plots, underflow and
 overflow bins can be enabled for the x and y axes individually with the
 \code{"x_axis_overflows"} and \code{"y_axis_overflows"} style
 attributes.

 \item \code{"overflow_line"}, \code{"overflow_line_color"},
 \code{"overflow_line_dash"}, \code{"overflow_line_thickness"}: Whether
 to draw lines separating the overflow bins from the rest of the bin
 data, and the color, dash pattern, and thickness of the line.

 \item \code{"right_margin"}: The size of the margin on the right side
 of the plot.

 \item \code{"title"}: A title to draw on the plot.

 \item \code{"top_margin"}: The size of the margin on the top of the
 plot.

 \item \code{"x_axis_font"}, \code{"x_axis_font_size"}: The font and
 font size to use for labelling the x axis and its ticks.  Likewise for
 the y axis.

 \item \code{"x_axis_line"}, \code{"x_axis_color"}, 
 \code{"x_axis_thickness"}: Whether to draw the x axis, and its color
 and thickness.  Likewise for the y axis.

 \item \code{"x_axis_offset"}: The distance between the x axis and the
 edge of the bin data.  Likewise for the y axis.

 \item \code{"x_axis_position"}, \code{"y_axis_position"}: The position
 of the axes on the plot.  For the x axis, the value may be either
 \code{"bottom"} or \code{"top"}; for the y axis, either \code{"left"}
 or \code{"right"}.

 \item \code{"x_axis_range"}: The range of values to display along the x
 axis, as a \code{(lo, hi)} pair.  Likewise for the y axis.

 \item \code{"x_axis_ticks"}: Specifies the tick marks to draw along the
 x axis.  The value may be the approximate number of tick marks to use
 (chosen heuristally from the axis range); or a sequence of tick
 positions; or \code{None}.  Likewise for the y axis.

 \item \code{"x_axis_tick_size"}, \code{"x_axis_tick_thickness"}: The
 length and thickness of tick marks on the x axis.  Likewise for the y
 axis. 

 \item \code{"z_range"}: For two-dimensional plots, the range of values
 to display on the (virtual) z axis, as a \code{(lo, hi)} pair.

 \item \code{"zero_line"}, \code{"zero_line_color"},
 \code{"zero_line_thickness"}: In one-dimensional plots, whether to draw
 the zero line, and the zero line's color and thickness.

\end{itemize}

In addition, the style attributes that control the border, aspect ratio,
and size of layouts can be specified for plot objects as well.  See the
section on layout styles in the chapter on drawing.

\subsection{Styles for all plot series}

These style attribute control the visual styles of individual series in
plots.  If a series's style dictionary doesn't contain a particular
attribute, the value from the parent \code{Plot} object's style
dictionary is used.  Certain style attributes are used only for some bin
styles.

\begin{itemize}
 \item \code{"color"}: The color in which to draw this series.  

 \item \code{"dash"}: The dash pattern to use for lines.

 \item \code{"errors"}: If true, display bin errors for histograms that
 contain error information.  The representation of bin errors depends on
 the bin style.

 \item \code{"marker"}: The marker to use when drawing points at bin
 contents.  If \code{None}, no markers are drawn.

 \item \code{"marker_size"}: The marker size to use when drawing points
 at bin contents.

 \item \code{"thickness"}: The line thickness to use.

 \item \code{"suppress_zero_bins"}: If true, a bins with zero contents
 will not be drawn.

\end{itemize}


\subsection{Styles for 1D histograms plot series}

These style attributes are specific to plot series of one-dimensional
histograms.

\begin{itemize}
 \item \code{"bins"}: The style to use to draw bins.  These bin styles
 may be used:
 \begin{itemize}
  \item \code{"points"} (the default): Draws a marker and/or cross at
  the value of each bin.

  \item \code{"skyline"}: Draws the traditional outline or filled region
  representing bin contents.

 \end{itemize}

 \item \code{"bin_center"}: In the \code{"points"} bin style, the
 fractional horizontal location within the bin to draw the marker and
 the vertical error bar.  The value should be between zero and one.  The
 default is 0.5, which centers the marker and error bar in the bin.
 When superimposing two series with identical binning and similar bin
 contents, it is useful to offset the error bars of one or both to
 prevent them from overlapping.

 \item \code{"cross"}: In the \code{"points"} bin style, whether to
 draw a horizontal line in each bin.  The position of the line shows the
 bin contents, and the length of the line shows the bin width.

 \item \code{"error_hatch_color"}, \code{"error_hatch_pitch"},
 \code{"error_hatch_thickness"}: In the \code{"skyline"} bin style, bin
 errors are depicted with a 45-degree hatch pattern.  These style
 attributes control the color, spacing, and thickness of the hacth
 pattern.  If the color is \code{None}, the hatch pattern uses the fill
 pattern for errors above the bin contents, and the background color
 below the bin contents.

 \item \code{"fill_color"}: In the \code{"skyline"} bin style, the color
 with which to fill the bins.  If \code{None}, the bins are not filled.

 \item \code{"line_color"}: In the \code{"skyline"} bin style, the color
 for the outline of the bin contetns.  If \code{None}, no outline is
 drawn.

\end{itemize}


\subsection{Styles for 2D histograms plot series}

These style attributes are specific to plot series of two-dimensional
histograms.

\begin{itemize}
 \item \code{"bins"}: The style to use to draw bins.  These bin styles
 may be used:
 \begin{itemize}
  \item \code{"box"}: Draws a box for each bin with area proportional to
  the value of the bin content.

  \item \code{"density"}: Shades each bin with a color reprenting the
  bin content.
 \end{itemize}

 \item \code{"negative_color"}: If a color is specified, an empty bin
 represents zero bin contents, and this color is used to draw the
 contents of negative bins.  If this style attribute is \code{None}, an
 empty bin represents the low end of the z range, and all values are
 represented relative to this value.

 \item \code{"overrun_color"}, \code{"underrun_color"}: Colors to use to
 draw bins whose values are outside the z range of the plot.
\end{itemize}


\subsection{Styles for function plot series}

These style attributes are specific to plot series of one-dimensional
functions.  

\begin{itemize}
 \item \code{"bins"}: The style to use to draw bins.  These bin styles
 may be used:
 \begin{itemize}
  \item \code{"curve"}: Draw the function value as a curve.
 \end{itemize}

 \item \code{"number_of_samples"}: The number of points at which to
 sample the function.

\end{itemize}

%-----------------------------------------------------------------------

\section{Annotations and decorations}

Using the class \class{hep.hist.plot.Annotation}, you can add textual
annotations to a plot.  The annotation contains one or more lines of
text.  You may specify the text when constructing the object, or
incrementally using the \method{append} method or left-shift operator.
If the text contains newline characters, it is split into multiple
lines.

The annotations are drawn left-justified in the upper-left corner of the
plot, or right-justified in the upper-right corner of the plot.  Specify
the position with the \code{"position"} style, which may be either
\code{"left"} or \code{"right"}.  The styles \code{"annotation_font"},
\code{"annotation_font_size"}, \code{"annotation_color"}, and
\code{"annotation_leading"} control how the text is drawn.

Simply add the \class{Annotation} object as a series to the plot.

For example, this code makes a plot of a histogram with an annotation
describing its bin and axis type
\begin{verbatim}
>>> plot = hep.hist.plot.Plot(1, histogram)
>>> annotation = hep.hist.plot.Annotation("title: %s" % histogram.title)
>>> annotation << "bin type: %s" % histogram.bin_type.__name__
>>> annotation << "axis type: %s" % histogram.axis.type.__name__
>>> plot.append(annotation)
\end{verbatim}

The class \class{hep.hist.plot.Statistics} is a subclass of
\class{Annotation} which annotates histogram statistics.  To create one,
provide a sequence of statistic names to include, and one or more
histograms.  Statistic names can be \code{"sum"}, \code{"mean"},
\code{"variance"}, \code{"sd"}, and \code{"overflows"}.  Then simply add
the statistics annotation object as a series to the plot.

For example, to plot a histogram and display its sum (integral), mean,
and standard deviation,
\begin{verbatim}
>>> plot = hep.hist.plot.Plot(1, histogram)
>>> plot.append(hep.hist.plot.Statistics(("sum", "mean", "sd"), histogram))
\end{verbatim}

If you want to include the statistics with other annotations, you can
generate the text of the statistics annotation with the function
\function{hep.hist.plot.formatStatistics}.

%-----------------------------------------------------------------------

\section{Prepackaged plot functions}

\chapter{Lorentz geometry and kinematics}

\pyhep provides an implementation of Lorentz vectors, momenta, and
transformations.  The implementation classes do not simply provide
quadruplets of numbers equipped with a Minkowski inner product; instead,
they represent coordinate-independent objects.  Ultimately, though, the
coordinates of a vector must be specified and obtained; this is done
using a \emph{reference frame}, which specifies the coordinate system to
use. 

Coordinates are given in the order $(t, x, y, z)$, the metric signature
$(+,-,-,-)$ is employed, and the speed of light $c$ is assumed to be
unity.

\section{Reference frames}

A four-vector is a geometric object, which can be used to represent, for
instance, the space-time position of an event, or the energy-momentum of
a particle.  The typical representation of a four-vector is a quadruplet
of four coordinates, but the coordinate values for a particular
four-vector depends on the basis used, or equivalently on the reference
frame in which the coordinates are specified.

The Python class \class{hep.lorentz.Frame} represents a reference
frame.  The principle of relativity implies that there is no absolute
way of specifying a frame; the frame may only be specified in relation
to another.  In \pyhep, a frame is specified in relation to a special
frame, the canonical \emph{lab frame}.  The lab frame is
\member{hep.lorentz.lab}, an instance of \class{Frame}.  

\section{Vectors}

A four-vector is represented by an instance of
\class{hep.lorentz.Vector} class.  A \class{Vector} instance represents
the geometric object, which is independent of reference frame, so you
can specify or obtain its coordinate values only in reference to a
frame.  You may use the lab frame for this, or any other frame which you
create.

To create a vector by specifying its coordinates in a particular frame,
use the \method{Vector} method of that frame.  For instance,
\begin{verbatim}
from hep.lorentz import lab
vector = lab.Vector(5.0, 1.0, 0.0, -2.0)
\end{verbatim}
creates a four-vector whose time coordinate is 5.0 and whose space
coordinates are (1.0, 0.0, -2.0) in the lab frame.

To obtain the coordinates of a four-vector in a reference frame, use the
\method{coordinatesOf} method of that frame.  For instance,
\begin{verbatim}
t, x, y, z = lab.coordinatesOf(vector)
\end{verbatim}

You may negate (invert) a vector or scale it by a constant.  You may
also add or subtract two vectors.  Since these are geometric operations,
no frame is specified.  For example,
\begin{verbatim}
vector3 = - vector1 / 2 + 3 * vector2
\end{verbatim}
Use the \code{\^} operator to obtain the inner product of two vectors.
Each vector also has an attribute \member{norm}, its Lorentz-invariant
normal.  For example,
\begin{verbatim}
c = (vector1 ^ vector2) / (vector1.norm * vector2.norm)
\end{verbatim}

You may also use \class{hep.lorentz.Momentum}, a subclass of
\class{Vector} that represents a four-momentum.  It provided an
additional attribute \member{mass}, which is equivalent to
\member{norm}, plus an attribute \member{rest_frame}, which is a
\class{Frame} object representing the rest frame of a particle with that
four-momentum. 

\section{Transformations and frames}

A \class{hep.lorentz.Transformation} object represents a general Lorentz
transformation.  It can be used to transform either a geometric object,
such as a four-vector, or a reference frame.

Typically, a transformation is specified as a rotation or a boost.  A
rotation is specified by the Euler angles $\phi, \theta, \psi$ in a
particular reference frame.  A boost is specified by the vector
$\vec\beta$ in a particular reference frame.  The frame object's
\method{Rotation} and \method{Boost} methods, respectively, create these
transformations.   For example, 
\begin{verbatim}
from hep.lorentz import lab
from math import pi
rotation = lab.Rotation(pi / 4, pi / 4, 0)
boost = lab.Boost(0.0, 0.0, 0.5)
\end{verbatim}
The arguments to \method{Rotation} are the Euler angles, and the
arguments to \method{Boost} are the components of $\vec\beta$.

Transformations may be composed using the \code{\*} operation.  Be
careful about the frame in which you specify each one; generally, for
sequential transformations, you will want to apply the previous
transformation to your starting frame of reference before specifying the
next one.

You can apply a transformation to a four-vector using the \code{\^}
operator; this returns a different geometric four-vector.  

FIXME

A transformation can also be used to create a new reference frame.


\chapter{Particle properties}

The \module{hep.pdt} module provides code to access a \emph{particle
data table}, which contains measured properties of particles studied in
high energy physics.  

\pyhep includes a particle data table in \code{hep.pdt.default}.  This
table contains particle data from the XXXX edition of the \textit{Review
of Particle Properties} published by the Particle Data Group (PDG).  The
function \function{hep.pdt.loadPdtFile} can be used to load particle
data from a file in the PDG's text file format.

A particle data table is represented by a \class{hep.pdt.Table}
instance, which acts as a dictionary keyed with the plain-text names of
particles.  The text (ASCII) names are set by the PDG.  These names are
generally the same as the conventional particle designations, with Greek
letters spelled out and underscores used to denote subscripts.  For
particle names associated with more than one charge state, the charge
muse be indicated.  Neutral antiparticles conventionally designated with
an overbar are denoted by the ``\code{anti-}'' prefix.  For example, a
positron is spelled ``\code{e+}'', and the short-lived kaon eigenstate
``\code{K_S0}''.  The usual \method{keys} method will return the
plain-text names of all particles in the table.  A particle's name may
have alternate common text spellings; these are stored as aliases, and a
particle may be accessed in the table by any of its aliases.

The values in a particle data table are \class{hep.pdt.Particle}
instances.  Each represents a single species of particle, with unique
quantum numbers.  (Some special particle codes representing bound
states, particles that have not been established, Monte Carlo internal
constructs, etc. are also included.)  

A \class{Particle} object has these attributes.  Central values are
given for measured properties.

\begin{itemize}

  \item \member{name} is the particle's text name.

  \item \member{aliases} is a sequence of alternate text names for the
  particle.

  \item \member{charge_conjugate} is the \class{Particle} object for the
  particle's charge conjugate.

  \item \member{mass} is the particle's nominal mass, in GeV.

  \item \member{width} is the particle's width, in GeV.

  \item \member{charge} is the particle's electric charge.

  \item \member{spin} is the particle's spin, in units of half h-bar.

  \item \member{is_stable} is true if the particle is considered
  stable. 

  \item \member{id} is the particle's Monte Carlo ID number in the PDG's
  numbering scheme.

\end{itemize}

The \class{Table} object also as a method \method{findId}, which returns
the particle corresponding to a Monte Carlo ID number.

The following is a demonstration of using the default particle data
table to look up some particle properties.
\begin{verbatim}
>>> from hep.pdt import default as particle_data
>>> print particle_data["e-"].mass
0.000510999
>>> print particle_data["J/psi"].spin
1.0
>>> print particle_data.findId(22).name
gamma
\end{verbatim}

%-----------------------------------------------------------------------

\chapter{Using \evtgen}

\pyhep provides a simple interface to the \evtgen event generator.
You can easily generate randomized decays of particles, and examine the
decay products.  The \evtgen interface is in the \module{hep.evtgen}
module.

To create a particle decay, follow these steps:
\begin{enumerate}

 \item Create a \class{Generator} instance.  The two arguments to its
 constructor are the path to the particle data listing file, which
 contains particle property information, and the path to the main decay
 file, which contains decays and branching fractions.  The default
 \evtgen particle data and decay files are used if these arguments are
 omitted.  You may specify paths to user decay files, which override the
 main decay file, as additional arguments.  See the \evtgen
 documentation for information about these files.

 \item Create a \class{Particle} object to represent the initial-state
 particle.  Specify the name of the particle, as listed in the particle
 data file, as the argument.  The particle is originally at rest at the
 origin of the lab frame.

 \item Produce the decay by calling the generator's \method{decay}
 method on the particle object.

\end{enumerate}

A \class{Particle} object's momentum is stored in its \member{momentum}
attribute, as a \class{hep.lorentz.FourMomentum} object.  Use
\method{hep.lorentz.lab.coordinatesOf} to obtain its lab-frame
coordinates.  Similarly, its production position is stored in its
\member{position} attribute, as a \class{hep.lorentz.FourVector}.
The name of the Particle's species is in its \member{species} attribute.

Use the \member{decay_products} attribute to access the decay products
of a decayed particle.  That value is a sequence of \class{Particle}
objects representing the particle's decay products.

The following script produces a single decay of an Upsilon(4S), using
a particle data listing file and a decay file in the current directory.
It prints out the decay tree of the Upsilon(4S), with the lab components
of each product's momentum, using the \function{printParticleTree}
function.
\codesample{evtgen1.py}


\chapter{Interactive PyHEP}

%-----------------------------------------------------------------------

This chapter describes how to use interactive \pyhep.  Interactive
\pyhep is simply the ordinary Python interpreter with certain \pyhep
modules preloaded, commonly-used names imported into the global
namespace, and some additional interactive functions and variables are
provided.


\section{Invoking interactive \pyhep}

The \code{pyhep} script launches interactive \pyhep.  This is installed
by default in \code{/usr/bin}; check your installation for your specific
location.  

In addition to the Python interpreter's usual startup message, the
\pyhep version is displayed.

\section{Imported names}

These names are imported into the global namespace:
\begin{itemize}

 \item The contents of the standard \module{math} module.

 \item The contents of the \module{hep.num} module.

 \item If your version of Python doesn't include built-in \code{bool},
 \code{True}, and \code{False}, these names are added.

 \item The \code{hep.lorentz.lab} reference frame object, as \code{lab}.

 \item The \code{default} particle data dictionary from
 \module{hep.pdt}, as \code{pdt}.

\end{itemize}


\section{Interactive functions and variables}

The names of interactive \pyhep functions always begin with `\code{i}',
and the names of variables always begin with `\code{i_}'.

The \function{ihelp} function displays some help information for
interactive \pyhep.


\subsection{Plotting functions}

Interactive \pyhep plots histograms in a pop-up plot window, similar to
PAW.  The plot window may be divided into rectangular ``zones'', each
displaying one plot.  All plots are shown in the same window, replacing
other plots where necessary.

\begin{funcdesc}{iplot}{histogram\optional{, **style}}
 Show a new plots of a histogram or scatter plot in the plot window.
 The argument is a histogram or \class{Scatter} object.  Any keyword
 arguments are used as style attributes for the new plot.
\end{funcdesc}

\begin{funcdesc}{iseries}{histogram\optional{, **style}}
 Add a series to the current plot in the plot window.  The argument is a
 histogram or \class{Scatter} object.  Any keyword arguments are used as
 style attributes for the new plot.
\end{funcdesc}

\begin{funcdesc}{igrid}{columns, rows}
 Divide the plot window into a grid of rectangular plot zones.  The
 arguments are the number of columns and rows in the grid.  Each call to
 \function{iplot} uses the next plot zone, proceeding left-to-right and
 then top-to-bottom.  Note that calling \function{igrid} always
 removes all plots and clears the plot window.
\end{funcdesc}

\begin{funcdesc}{iselect}{column, row}
 Select a plot zone.  Its arguments are the column and row coordinates
 of the zone.  The next plot is drawn in this zone. 
\end{funcdesc}

\begin{funcdesc}{ishow}{figure}
 Show \code{figure} in the plot window.  If you have divided the figure
 with \function{igrid}, shows \code{figure} in the current cell.
\end{funcdesc}

\begin{funcdesc}{iprint}{file_name}
 Print the contents of the plot window to a file.  The type of the file
 is inferred from the file name extension: ``.ps'' produces a PostScript
 file, and ``.eps'' produces an encapsulated PostScript file.
\end{funcdesc}

The global variable \code{ifig} always refers to the current \code{Plot}
object, the current figure (if you added one with \function{ishow}, or
\code{None} (if the current plot zone is empty).  The global variable
\code{i_plots} is a sequence of sequences of \code{Plot} objects for all
zones in the plot window.

The global variable \code{iwin} is a draw object for the plot window.

The global variable \code{istyle} is a style dictionary that is used for
plots and series created with \function{iplot} and \function{iseries}.
If you set a style attribute in \code{istyle}, it will become the
default for subsequent plots.  You can also call the \method{setall}
method of \code{istyle}, which sets a style attribute not only in this
dictionary, but in all of the plots currently shown in the plot window. 
For example, to set a log scale for the plots currently shown as well as
subsequent plots,
\begin{verbatim}
>>> istyle.setall("log_scale", True)
\end{verbatim}
To set a style attribute in the current plot only, access its
\member{style} member through \code{ifig}, like this:
\begin{verbatim}
>>> ifig.style["log_scale"] = True
\end{verbatim}
\pyhep keeps track of when a plot's style dictionary is changed, and
redraws the plot automatically.

You may modify any plot displayed in any plot zone.  Select the zone
with \function{iselect}, access and modify the plot with \code{i_plot}
(for instance, adjust its style or add a series), and then call
\function{iredraw} to redraw it.

\subsection{Table functions}

\begin{funcdesc}{iproject}{table, expression\optional{, selection,
      number_of_bins, range}}
 Project a histogram from a table.  The first argument is the table
 object, and the second argument is the expression to accumulate in the
 histogram.  The optional \code{selection} argument is a selection
 expression; only table rows for which this expression is true are
 projected into the histogram.  The number of bins and histogram range
 are automatically determined, but to override these, use the
 \code{number_of_bins} and \code{range} (a pair of values) arguments,
 respectively.  The return value is a histogram object.

 To project a one-dimensional histogram, use an ordinary expression.  To
 project a two-dimensional histogram, specify two expressions in
 \code{expression} separated by commas; for instance \code{"p_x, p_y"}.
 You can also use the \function{iproj1} function to project a
 one-dimensional histogram from a table.
\end{funcdesc}

\begin{funcdesc}{idump}{rows, *expressions}
 Dumps values from a table.  The first argument is a table or an
 iterator over table rows.  One or more additional arguments are
 expressions whose values are to be displayed.  The output is a table
 with the row number followed by the values of the specified
 expressions.
\end{funcdesc}



% FIXME: The reference manual is out-of-date.
%% \chapter{Reference Manual}
%% \section{\module{hep.lorentz} -- Lorentz geometry and kinematics}

\declaremodule{extension}{hep.lorentz}
\modulesynopsis{Lorentz geometry and kinematics.}

\subsection{Kinematics}

The \module{hep.lorentz} module provides these functions to compute
kinematic quantities.

\begin{funcdesc}{twoBodyDecayMomentum}{mass_a, mass_b, mass_c}
 For the decay \emph{a -> b + c}, compute the magnitude of each of the
 momenta of \emph{b} and \emph{c} in the rest frame of \emph{a}, given
 the masses of the three particles.
\end{funcdesc}

%% \section{\module{hep.hist} --- Histograms}

\declaremodule{extension}{hep.hist}
\modulesynopsis{Histograms.}

This module provides a way of collecting and manipulating histograms.  A
histogram is used to measure a statistical distribution by collecting
data sampled from this distribution.  This module supports binned
histograms, in which the data are divided into rectangular bins spanning
the range of possible values.  The values themselves may be one- or
higher-dimensional, and the value's component in each dimension may be
of a different numerical type.

A \class{Histogram} object represents a histogram of a distribution of
one or more dimensions.  Each dimension is represented by an
\class{Axis} object.  The \class{Axis} specifies the minimum and maximum
value of each sample point in that dimensions, and divides the range
into evenly-sized bind.

\begin{funcdesc}{Axis}{num_bins, min, max\optional{, axis_type,
 **kw_args}}
 Create an object representing an evenly-binned dimension in sample
 space with values greater than or equal to \var{min} and less than
 \var{max}.  This range of values is divided into \var{num_bins} bins of
 even size.  Values in this dimension are specified by the numerical
 type \var{axis_type}, which is inferred from \var{min} and \var{max} if
 omitted.  If \var{axis_type} is integral, the number of bins must
 divide evenly into \var{max} - \var{min}.

 Any additional \var{**kw_args} are added to the \class{Axis} object as
 attributes.
\end{funcdesc}

A histogram also records the number of samples which fall above or below
the specifies axis range in each dimension.  A value below the minimum
is stored as an ``underflow'' value along that axis, and a value equal
to or above the maximum is stored as an ``overflow''.

For each bin, a histogram stores the accumulated weight of samples whose
sample values fall in the bin.  Often, samples are accumulated with unit
weight, so that the content of a bin is the number of samples whose
values fall in the bin.  

The histogram also estimates the error on each bin.  The error is
computed as the square root of the sum of the squares of weights of
sample whose sample values fall in the bin.  If samples are accumulated
with unit weight, this is equal to the square root of the number of
samples whose values fall in the bin.  Note that this estimate of the
error is not always statistically valid; use your judgement when
interpreting this error value.

To create a histogram, use

\begin{funcdesc}{Histogram}{*axes\optional{, value_type, **kw_args}}
 Create a histogram with evenly-binned axes that implements the
 histogram protocol.  

 The number of dimensions of the histogram is equal to the number of
 \var{axes} arguments.  Each argument may be either
 \begin{itemize}
  \item An \class{Axis} instance.

  \item A sequence of the form \code{(num_bins, min, max\optional{,
  axis_type, name, units})}.  If specified, \var{axis_type} is the type
  used for bin contents; if omitted, the type for that axis is inferred
  from the values of \var{min} and \var{max}.  If specified, \var{name}
  and \var{units} are set as attributes of the same name on the
  \class{Axis} object.
 \end{itemize}

 The keyword argument \var{bin_type} specifies the numeical type used to
 store the value in each bin.  \emph{If omitted, the default is
 \code{int}}.  If any additional \var{**kw_args} are provided, they are
 set as attributes to the returned histogram object.
\end{funcdesc}

Histograms of a one-dimenstional independent variable are commonly
used.  A subclass of \class{Histogram} for one-dimensional histograms,
\class{Histogram1D}, provides a more convenient interface and more
efficient implementation.

\begin{funcdesc}{Histogram1D}{num_bins, min, max\optional{, axis_type,
name, units, **kw_args}} 
 Arguments are as for \function{Histogram}.  As with
 \function{Histogram}, the numerical type used to represent the contents
 of a bin is \code{int} unless the \var{bin_type} keyword argument is
 provided.
\end{funcdesc}

For example, this call creates a two-dimensional histogram.  The first
dimension, named ``energy,'' takes \code{float} values between 0 GeV and
4 GeV, divided into 40 bins.  The second dimension, named ``drift
chamber hits,'' takes integer values between 0 and 60, with one bin for
each possible value.  Bin contents are stored as integers.  The title of
this histogram and the units for bin contents are also specified.
\begin{verbatim}
histogram = Histogram(
    (40, 0, 4, float, "energy", "GeV"),
    (60, 0, 60, int, "drift chamber hits"),
    title="Fiducial track distribution",
    units="tracks")
\end{verbatim}

This example creates a one-dimenstional histogram over \code{float}
values between -1 and 1.  Bin contents are stored as \code{float}.  
The histogram title is also specified.
\begin{verbatim}
histogram = Histogram1D(20, -1.0, 1.0, bin_type=float, title="cos(theta)")
\end{verbatim}

%-----------------------------------------------------------------------

\subsection{The histogram protocol}

The histogram protocol is used for histograms with arbitrary binning
along one or more axes.  

A position in the histogram is represented by a sequence whose length is
the number of dimensions of the histogram.  Each element of the sequence
is the position along the corresponding axis.  The type of each element
must be the type of the corresponding axis, or coercible to it.  Such a
position sequence corresponds to exactly one bin in the histogram, which
may be an overflow or underflow bin if one or more elements fall outside
the ranges of their corresponding bins.

A bin may also be specified by a bin number, which is also a sequence
whose length is the number of dimensions of the histogram.  Each element
is a bin number between zero and one less than the number of bins of the
corresponding axis, or the strings \code{"underflow"} or
\code{"overflow"}.  For example, \code{(0, 4, "overflow")} is a bin
number for a three-dimensional histogram, specifying the first bin in
the first dimension, the fifth bin in the second dimension, and a value
larger than the axis range in the third dimension.

\begin{memberdesc}{dimensions}
 \readonly The number of dimensions.
\end{memberdesc}

\begin{memberdesc}{axes}
 \readonly A sequence of \class{Axis} instances describing the
 dimensions of the histogram's sample space.
\end{memberdesc}

\begin{memberdesc}{bin_type}
 \readonly The type used to for bin contents.
\end{memberdesc}

\begin{methoddesc}{accumulate}{value\optional{, weight=1}}
 Add \var{weight} to the contents of the bin at \var{value}; update the
 bin error accordingly.
\end{methoddesc}

\begin{methoddesc}{__lshift__}{value}
 A synonym for \method{accumulate}, with unit \var{weight}.
\end{methoddesc}

\begin{methoddesc}{getBin}{bin_number}
 Return the contents of bin \var{bin_number}.
\end{methoddesc}

\begin{methoddesc}{setBin}{bin_number, bin_contents}
 Set the contents of bin \var{bin_number} to \var{bin_contents}. 
\end{methoddesc}

\begin{methoddesc}{setError}{bin_number, bin_contents}
 Set the error on bin \var{bin_number} to \var{bin_contents}. 
\end{methoddesc}

\begin{methoddesc}{getError}{bin_number}
 Return the error on the bin \var{bin_number}.
\end{methoddesc}

\begin{methoddesc}{getBinRange}{bin_number}
 Return the range of values spanned by the bin \var{bin_number}.  The
 result a sequence of pairs, where each pair contains the minimum and
 maximum value for the bin along that axis dimension.
\end{methoddesc}

\begin{methoddesc}{map}{value}
 Return the bin number corresponding to \var{value}.
\end{methoddesc}

An \class{Axis} instance represents a one-dimensional axis of a
histogram.  Bins are numbered from zero to one less than the number of
bins; the bin number for an underflow value is \code{"underflow"}, and
the bin number for an overflow value is \code{"overflow"}.

\begin{memberdesc}{number_of_bins}
 \readonly The number of bins along this axis, not including underflow
 and overflow bins.
\end{memberdesc}

\begin{memberdesc}{range}
 \readonly A pair \code{(min, max)} of the range of values spanned by
 the bins of the histogram.  A value less than \var{min} is considered
 an underflow, and a value greater than or equal to \var{max} is
 considered an overflow.
\end{memberdesc}

\begin{memberdesc}{type}
 \readonly The type for values along this axis.
\end{memberdesc}

\begin{methoddesc}{__call__}{value}
 Return the bin number for a value along the axis.  
\end{methoddesc}

\begin{methoddesc}{getBinRange}{bin_number}
 Return the range \code{(min, max)} for a bin.
\end{methoddesc}

%-----------------------------------------------------------------------

\subsection{The simplified one-dimensional histogram protocol}

\class{Histogram1D} uses a simplified protocol to ease manipulation of
one-dimensional histograms.  It is identical to the general histogram
protocol, except that bin values and bin numbers are represented by
single values instead of sequences, and bin ranges are represented by
\var{(min, max)} pairs instead of sequences of such pairs.

\class{Histogram1D} is a subclass of \class{Histogram}, and provides the
following additional methods and attributes:

\begin{memberdesc}{axis}
 \readonly The sample axis; equivalent to \code{axes[0]}.
\end{memberdesc}

\begin{methoddesc}{__getitem__}{bin_number}
 Equivalent to \function{getBin}.
\end{methoddesc}

\begin{methoddesc}{__setitem__}{bin_number, bin_contents}
 Equivalent to \function{setBin}.
\end{methoddesc}

For example, this code creates and fills a one-dimensional histogram
with \class{Histogram}:
\begin{verbatim}
histogram = Histogram((10, 0.0, 1.0), bin_type=float)
for value in data_samples:
    histogram << (value, )

for bin_number in range(10):
    print "bin %d: %f" % (bin_number, histogram.getBinContent((bin_number, )))
\end{verbatim}
Since \class{Histogram} represents a histogram with a sample space in
arbitrary dimensions, the sample value and bin_number for the
one-dimensional case must be specified as a sequence with one element,
in this case \code{(value, )} and \code{(bin_number, )} respectively.

Using \class{Histogram1D}, this code is simplified:
\begin{verbatim}
histogram = Histogram1D(10, 0.0, 1.0, bin_type=float)
for value in data_samples:
    histogram << value

for bin_number in range(10):
    print "bin %d: %f" % (bin_number, histogram[bin_number])
\end{verbatim}

%-----------------------------------------------------------------------

\subsection{Projecting histograms}

To accumulate into multiple histograms from arbitrary functions of a
sequence of values, use the \function{hep.hist.project} function.

\begin{funcdesc}{project}{events, projections\optional{, weight}}
 Project multiple histograms from a collection of \var{events}.

 The \var{events} argument is a sequence or iterator.  Each item is a
 map containing variable values for that event: each key is the name of
 a variable, and the corresponding value is the event's value for that
 variable.  Each event in \var{events} should have the same keys.

 The \var{projections} argument is a sequence describing the histograms
 to project.  Each sequence element is of the form \code{(expression,
 histogram)}, where \var{expression} is an expression over the variables
 in the events; for each event, it is evaluated and the result is
 accumulated into \var{histogram}.  The expression may be a string, a
 callable, or an expression object (see the description of
 \module{hep.expr} elsewhere in this manual).  

 The \var{weight} argument is an expression over the event variables
 which yields the weight to use for that event.  For each event, the
 same weight value is used for accumulating into all histograms.  If
 \var{weight} is omitted, unit weight is assumed.

 The function returns the sum of weights (which is the number of events,
 if unit weight is used) projected into the histograms.
\end{funcdesc}

Note that \function{project} is designed to work well with tables:
simply pass a table iterator or a sequence of row objects as
\var{events}.  In this case, \function{project} determines that a table
row is in use, and will use special table features to perform the
projections efficiently.  To project a subset of rows in a table, use
the selection feature of table iterators.

%-----------------------------------------------------------------------

%% \section{\module{hep.table} --- Tables}

\declaremodule{extension}{hep.table}
\modulesynopsis{Tables.}

A \pyhep \emph{table} is a flat database table with columns of fixed
type and storage size.  Tables are append-only: new rows may be added to
the end of the table, but rows may not be inserted elsewhere, nor can
rows be modified or deleted.  

Each table has a fixed \emph{schema}, which determines the data items
stored in the table and their types.  The schema includes an unordered
set of \emph{columns}, each identified by a name.  Each column has a
fixed data type.  A schema may also include \emph{expressions}, which
are derived quantities which are computed for each row from values of
columns and other expressions for that row.  

The table's data is an ordered sequence of \emph{rows}, each of which
contains exactly one value for each column in the table's schema.

Multiple table implementations are provided.  Implementations provide
different interfaces for creating and opening tables.  However, objects
representing tables and rows satisfy the same protocol across all
implementations.


\subsection{Schema objects}

An instance of \class{hep.table.Schema} represents a table schema.  

\begin{memberdesc}{columns}
 \readonly A sequence of columns in the schema.  Each element is a
 \class{Column} instance.  The order of the columns is unspecified.
\end{memberdesc}

\begin{memberdesc}{expressions}
 \readonly A sequence of expressions in the schema.  Each element is
 expression object (see \module{hep.expr}).  The order of the
 expressions is unspecified.
\end{memberdesc}

\begin{methoddesc}{addColumn}{name, type\optional{, **attributes}}
 Add a column to the sequence.  \var{name} is a character string, and
 \var{type} is a column type (see below).  If additional keyword
 arguments are provided, they are added to the column as attributes.
 New columns should not be added to a schema after the schema has been
 used to create a table.
\end{methoddesc}

\begin{methoddesc}{addExpression}{name, expression\optional{, **attributes}}
 Add an expression to the sequence.  \var{name} is a character string,
 and \var{expression} is an expression as a string or expression object.
 If additional keyword arguments are provided, they are added to the
 expression object as attributes.
\end{methoddesc}

An instance of \class{hep.table.Column} describes a column in a schema.
Do not instantiate this class directly; instead, use the
\method{addColumn} memthod of class \class{Schema}.

\begin{memberdesc}{name}
 \readonly The column's name.
\end{memberdesc}

\begin{memberdesc}{type}
 \readonly The column's data type.
\end{memberdesc}

The column's data type is specified by one of these names:
\begin{tableiii}{ccc}{code}{name}{C type}{Python Type}
  \lineiii{"int8"}{\code{signed char}}{\code{int}}
  \lineiii{"int16"}{\code{signed short}}{\code{int}}
  \lineiii{"int32"}{\code{signed int}}{\code{int}}
  \lineiii{"float32"}{\code{float}}{\code{float}}
  \lineiii{"float64"}{\code{double}}{\code{float}}
\end{tableiii}


An instance of \class{hep.table.Expression} describes a column in a
schema.  Do not instantiate this class directly; instead, use the
\method{addExpression} memthod of class \class{Schema}.

\begin{memberdesc}{name}
 \readonly The expressions's name.
\end{memberdesc}

\begin{memberdesc}{expression}
 \readonly An \class{Expression} instance; see \module{hep.expr}.
\end{memberdesc}

\begin{memberdesc}{formula}
 \readonly A string representation of the expression.
\end{memberdesc}



\subsection{Table objects}

A table object represents an in-memory table or a connection to an
externally-stored table.  In the former case, the table is erased when
the table object is deleted.  In the latter case, the connection is
closed but the table persists when the table object is deleted.

A table object satisfies the Python read-only sequence protocol.  The
sequence elements are the rows of the sequence.  Thus, the
\function{len} function returns the number of rows in the table, and
the \method{__getitem__} method extracts a row by its position in the
sequence. 

In addition, these methods are provided:

\begin{methoddesc}{append}{row\optional{, **kw_args}}
 (Read-write tables only.)  Appends a row to the end of the table.  The
 \var{row} must is a dictionary or other mapping object which maps
 column names to values.  Additional column values may be specified as
 keyword arguments.  If a column is specified both in \var{row} and in
 \var{**kw_args}, the value from latter is used.  A value must be
 specified for each column, and no other names may be specified.
 Returns the row index of the new row.
\end{methoddesc}

\begin{methoddesc}{__iter__}{}
 Returns an iterator over all rows in the table.
\end{methoddesc}

\begin{memberdesc}{rows}
 An interator over all rows in the table.
\end{memberdesc}

\begin{methoddesc}{select}{expr}
 Returns an iterator over rows in the table for which expression
 \var{expr} is true.  \var{expr} may be a string expression formula or
 an expression object.
\end{methoddesc}


\subsection{Row objects}

A row object may represent a row in a table, or a row that is compatible
with a table's schema (but not one of the rows in that table).

The row object satisfies the Python map protocol.  Each key is the name
of a column in the table's schema.  The corresponding map value is the
row's value for that column.  Row objects should be treated as
read-only; it is not possible to modify a table row by changing values
in a row object.

In addition, a row object provides these members and methods:

\begin{methoddesc}{asDict}{}
 Return a new Python dictionary of the column names and corresponding
 values in the row.
\end{methoddesc}

\begin{memberdesc}{index}
 \readonly The index of the row in the table.  The value of this
 attribute is undefined if the row was obtained from the table's
 \method{newRow} method and has not yet been passed to \method{append}.
\end{memberdesc}

\begin{memberdesc}{table}
 \readonly The table to which the row belongs.
\end{memberdesc}


\subsection{Standard implementation}

\pyhep's standard table implementation, in \module{hep.table}, uses a
simple, binary, on-disk representation of the table's data.  Table rows
are automatically loaded from disk as needed.

The \module{hep.table} module provides functions \function{create} and
\function{open} to create and open new tables, respectively.

\begin{funcdesc}{create}{filename, schema}
 Create a new table.  The table is stored in a file named by
 \var{filename}, which is created or overwritten.  The \var{schema}
 argument is a map from column names to \class{Column} instances,
 specifying the columns in the table.

 The return value is a table object, which is opened to the new, empty
 in write mode.
\end{funcdesc}

\begin{funcdesc}{open}{filename\optional{, mode="r"}}
 Open an existing table stored in the file named by \var{filename}.  The
 \var{mode} argument specifies the mode in which to open the table:
 \constant{"r"} to open the table in read-only mode, or \constant{"w"}
 to open the table in read-write mode.

 The return value is a table object.
\end{funcdesc}


%% \section{\module{hep.cernlib} -- CERNLIB interface}

\declaremodule{extension}{hep.cernlib}
\modulesynopsis{CERNLIB interface}

The \module{hep.cernlib} module provides access from Python to features
in CERNLIB, the CERN program library for high energy physics.  The
applicable components of CERNLIB are linked into \pyhep and need not be
provided separately.

\subsection{\module{hep.cernlib.hbook} -- HBOOK}

\declaremodule{extension}{hep.cernlib.hbook}
\modulesynopsis{HBOOK interface}

HBOOK is a system for managing histograms and ntuples.  It implements a
file format for storing these in structured disk files.  \pyhep provides
read and write access to histograms and ntuples stored in HBOOK files.

The module \module{hep.cernlib.hbook} provides these functions:

\begin{funcdesc}{open}{path\optional{, mode="r", record_length=1024,
purge_cycles=1}}
 Create or open an HBOOK file at \var{path}.  

 \var{mode} is the access mode, analogous to Python's built-in
 \function{file} and \function{open} functions.  If \code{"r"}, opens an
 existing HBOOK file for reading.  If \code{"r+"}, \code{"w"},
 \code{"a"}, or \code{"a+"}, opens an existing file for reading and
 writing, or creates the file if it doesn't exist.  If \code{"w+"},
 creates a file or overwrites an existing file.

 \var{record_length} is the RZ record length to use for this file.  

 If \var{purge_cycles} is set to a true value and the file is open for
 writing, the entire file will be purged when it is closed.
\end{funcdesc}

The return value from \function{open} is a \class{File} object.  The
file remains open until it is deleted.  

\subsubsection{Paths, IDs, and cycles}

In \pyhep, all paths in HBOOK files are specified relative to the root
directory of the HBOOK file.  The root directory itself is represented
by an empty string, and the path separator is a forward slash.  Note
that there is no notion of working directory or relative paths.

The last component of a path is the title of a histogram, ntuple, or
directory.  HBOOK does not require titles to be unique, even in the same
directory.  However, each entry is also assigned a positive integer ID
number (which is called the RZ ID, because HBOOK's file format is built
on top of the RZ input/output system).  \pyhep's functions for creating
tables and directories and saving histograms will choose an available RZ
ID by default, but you can specify a different value to use (except for
directories).

Thus, HBOOK allows you to place multiple items with the same title in
the same directory, but each must have a unique RZ ID.  When loading or
deleting histograms or ntuples, you may specify either the RZ ID instead
of the title in the path.  For example, a histogram with title MYHIST
and RZ ID 17 in directory DIR/SUBDIR can be accessed as
\code{"DIR/SUBDIR/MYHIST"} or \code{"DIR/SUBDIR/17"}.  When saving
histograms, though, you must specify the title in the path.

Note that because of how HBOOK keeps ntuples in memory while they're
being accessed, you may not access two ntuples with the same RZ ID at
the same time, even if the ntuples are stored in different HBOOK files
or directories.  Be sure to delete the \class{Table} object for one
before creating or loading another.

HBOOK files also support revision cycles; see the HBOOK or RZ
documentation for details.  Unless \function{hbook.open} is invoked with
\code{purge_cycles=0} (or the file is opened read-only), \pyhep
``purges'' the entire file when the file is closed.  This removes cycles
except the most recent for each histogram and ntuple in the file.

\subsubsection{\class{File} objects}

A \class{File} object has the following methods and attributes:

\begin{memberdesc}{writeable}
 True if the file is open for writing as well as reading.
\end{memberdesc}

\begin{methoddesc}{listdir}{\optional{path}}
 Return the contents of the directory specified by \var{path}.  The
 return value is a sequence of \code{(id, type, title)} tuples, where
 \var{id} is the RZ ID of the entry; \var{type} is one of
 \code{"1d~histogram"}, \code{"2d~histogram"}, \code{"ntuple"}, or
 \code{"directory"}; and \var{title} is the title of the entry.
\end{methoddesc}

\begin{methoddesc}{mkdir}{path}
 Create a new directory at \var{path}.
\end{methoddesc}

\begin{methoddesc}{rmdir}{path}
 Remove the directory at \var{path}.
\end{methoddesc}

\begin{methoddesc}{save}{path, object\optional{, rz_id}}
 Store a histogram \var{object} at \var{path}.  One- and two-dimensional
 histograms are supported.  An unused RZ ID is chosen automatically,
 unless \var{rz_id} is specified or \var{object} has an \member{rz_id}
 attribute.  The function returns the RZ ID under which 'object' was
 saved. 
\end{methoddesc}

\begin{methoddesc}{load}{path}
 Load a histogram or connect to an ntuple at \var{path}.  If \var{path}
 is a one- or two-dimensional histogram, returns a histogram object with
 its contents.  If \var{path} is an ntuple, returns a table object (see
 \module{hep.table}) accessing it.
\end{methoddesc}

\begin{methoddesc}{createTable}{path, schema\optional{, rz_id, column_wise=1}}
 Create a table as an ntuple at \var{path} in the HBOOK file, using
 columns from \var{schema}, a \class{hep.table.Schema} instance.  If
 \var{rz_id} is not provided, an unused RZ ID is chosen automatically.
 The table is stored as a column-wise ntuple, unless a false value is
 passed for \var{column_wise}.  

 Column-wise ntuples support these column types: \constant{"int32"},
 \constant{"int64"}, \constant{"float32"}, and \constant{"float64"}.

 Row-wise ntuples support only \constant{"float32"} column types.
\end{methoddesc}

\begin{methoddesc}{remove}{path}
 Remove the ntuple or histogram at \var{path}.  Use \method{rmdir} to
 remove a directory.
\end{methoddesc}

\begin{methoddesc}{normpath}{path}
 Return the canonical form of absolute path \var{path} in the HBOOK
 file.
\end{methoddesc}

\begin{methoddesc}{join}{*components}
 Join together \var{components} to form an absolute path.
\end{methoddesc}

\begin{methoddesc}{split}{path}
 Return \code{(dir, name)}, where \var{dir} is the portion of \var{path}
 up to the last directory separator, and \var{name} is the remainder.
\end{methoddesc}


%% \section{\module{hep.root} -- Root interface}

\declaremodule{extension}{hep.root}
\modulesynopsis{Root interface}

The \module{hep.root} module provides access from Python to features in
Root, a framework for high energy physics data management and analysis.
The applicable components of Root are included with \pyhep and need not
be provided separately.

\subsection{Root files}

Root includes an implementation of histograms and ntuples (which are
called ``trees'' in Root).  It implements a file format for storing
these in structured disk files.  \pyhep provides read and write access
to histograms and trees stored in Root files.

The \function{open} function access a Root file:

\begin{funcdesc}{open}{path\optional{, mode="r", purge_cycles=1}}
 Create or open a Root file at \var{path}.  

 \var{mode} is the access mode, analogous to Python's built-in
 \function{file} and \function{open} functions.  If \code{"r"}, opens an
 existing file for reading.  If \code{"r+"}, \code{"w"}, \code{"a"}, or
 \code{"a+"}, opens an existing file for reading and writing, or creates
 the file if it doesn't exist.  If \code{"w+"}, creates a file or
 overwrites an existing file.

 If \var{purge_cycles} is set to a true value and the file is open for
 writing, the entire file will be purged when it is closed.  
\end{funcdesc}

The return value from \function{open} is a \class{File} object.  The
file remains open until it is deleted.  

In \pyhep, all paths in Root files are specified relative to the root
directory of the Root file.  The root directory itself is represented by
an empty string, and the path separator is a forward slash.  Note that
there is no notion of working directory or relative paths.

Root files also support revision cycles; see the Root documentation for
details.  Unless \function{root.open} is invoked with
\code{purge_cycles=0} (or the file is opened read-only), \pyhep
``purges'' the entire file when the file is closed.  This removes cycles
except the most recent for each object in the file, whether or not the
object was written or revised in the current session.

\pyhep stores additional information along with tables and histograms in
Root files.  This information, called \emph{metadata}, consists of
useful information about histograms and tables that is not handled
directly by the Root file format.  \pyhep stores metadata in
subdirectories named ``_pyhep_metadata'' in the Root file.  These
subdirectories and their contents may be removed safely without injuring
the primary data in the histograms and tables they are associated with,
but doing so will delete configuration and annotations accessible from
\pyhep.

\subsubsection{\class{File} objects}

A \class{File} object has the following methods and attributes:

\begin{memberdesc}{writeable}
 True if the file is open for writing as well as reading.
\end{memberdesc}

\begin{methoddesc}{listdir}{\optional{path}}
 Return the contents of the directory specified by \var{path}.  The
 return value is a sequence of \code{(name, title, class_name)} tuples,
 where \var{name} is the object's name, \var{title} is the object's
 title, and \var{class_name} is the object's class name.
\end{methoddesc}

\begin{methoddesc}{mkdir}{path\optional{, title}}
 Create a new directory at \var{path}.  Optionally, a title may be
 specified in the \var{title} argument.
\end{methoddesc}

\begin{methoddesc}{rmdir}{path}
 Remove the directory at \var{path}.
\end{methoddesc}

\begin{methoddesc}{save}{path, object}
 Store a histogram \var{object} at \var{path}.  One- and two-dimensional
 histograms are supported.  
\end{methoddesc}

\begin{methoddesc}{load}{path}
 Load a histogram or connect to an ntuple at \var{path}.  If \var{path}
 is a one- or two-dimensional histogram, returns a histogram object with
 its contents.  If \var{path} is a tree, returns a table object (see
 \module{hep.table}) accessing it.
\end{methoddesc}

\begin{methoddesc}{createTable}{path, schema\optional{, title,
separate_branches=0, branch_name="default"}}
 Create a table as a tree at \var{path} in the Root file, using
 columns from \var{schema}, a \class{hep.table.Schema} instance.  

 By default, column values is stored in a leaves grouped in a single
 branch, named "default".  Another name may be specified for the branch
 with the \var{branch_name} argument.  Alternately, if
 \var{separate_branches} is true, each leaf is placed in a separate
 branch, whose name is the column name.

 These column types are supported: \constant{"int8"},
 \constant{"int16"}, \constant{"int32"}, \constant{"float32"}, and
 \constant{"float64"}.
\end{methoddesc}

\begin{methoddesc}{remove}{path}
 Remove the tree or histogram at \var{path}.  Use \method{rmdir} to
 remove a directory.
\end{methoddesc}

\begin{methoddesc}{normpath}{path}
 Return the canonical form of absolute path \var{path} in the Root
 file.
\end{methoddesc}

\begin{methoddesc}{join}{*components}
 Join together \var{components} to form an absolute path.
\end{methoddesc}

\begin{methoddesc}{split}{path}
 Return \code{(dir, name)}, where \var{dir} is the portion of \var{path}
 up to the last directory separator, and \var{name} is the remainder.
\end{methoddesc}



%-----------------------------------------------------------------------

\appendix

% \chapter{Reporting Bugs}
% \input{reportingbugs}

% \chapter{History and License}
% \input{license}

% \input{test.ind}			% Index -- must be last

%-----------------------------------------------------------------------

\end{document}
