\section{\module{hep.table} --- Tables}

\declaremodule{extension}{hep.table}
\modulesynopsis{Tables.}

A \pyhep \emph{table} is a flat database table with columns of fixed
type and storage size.  Tables are append-only: new rows may be added to
the end of the table, but rows may not be inserted elsewhere, nor can
rows be modified or deleted.  

Each table has a fixed \emph{schema}, which determines the data items
stored in the table and their types.  The schema includes an unordered
set of \emph{columns}, each identified by a name.  Each column has a
fixed data type.  A schema may also include \emph{expressions}, which
are derived quantities which are computed for each row from values of
columns and other expressions for that row.  

The table's data is an ordered sequence of \emph{rows}, each of which
contains exactly one value for each column in the table's schema.

Multiple table implementations are provided.  Implementations provide
different interfaces for creating and opening tables.  However, objects
representing tables and rows satisfy the same protocol across all
implementations.


\subsection{Schema objects}

An instance of \class{hep.table.Schema} represents a table schema.  

\begin{memberdesc}{columns}
 \readonly A sequence of columns in the schema.  Each element is a
 \class{Column} instance.  The order of the columns is unspecified.
\end{memberdesc}

\begin{memberdesc}{expressions}
 \readonly A sequence of expressions in the schema.  Each element is
 expression object (see \module{hep.expr}).  The order of the
 expressions is unspecified.
\end{memberdesc}

\begin{methoddesc}{addColumn}{name, type\optional{, **attributes}}
 Add a column to the sequence.  \var{name} is a character string, and
 \var{type} is a column type (see below).  If additional keyword
 arguments are provided, they are added to the column as attributes.
 New columns should not be added to a schema after the schema has been
 used to create a table.
\end{methoddesc}

\begin{methoddesc}{addExpression}{name, expression\optional{, **attributes}}
 Add an expression to the sequence.  \var{name} is a character string,
 and \var{expression} is an expression as a string or expression object.
 If additional keyword arguments are provided, they are added to the
 expression object as attributes.
\end{methoddesc}

An instance of \class{hep.table.Column} describes a column in a schema.
Do not instantiate this class directly; instead, use the
\method{addColumn} memthod of class \class{Schema}.

\begin{memberdesc}{name}
 \readonly The column's name.
\end{memberdesc}

\begin{memberdesc}{type}
 \readonly The column's data type.
\end{memberdesc}

The column's data type is specified by one of these names:
\begin{tableiii}{ccc}{code}{name}{C type}{Python Type}
  \lineiii{"int8"}{\code{signed char}}{\code{int}}
  \lineiii{"int16"}{\code{signed short}}{\code{int}}
  \lineiii{"int32"}{\code{signed int}}{\code{int}}
  \lineiii{"float32"}{\code{float}}{\code{float}}
  \lineiii{"float64"}{\code{double}}{\code{float}}
\end{tableiii}


An instance of \class{hep.table.Expression} describes a column in a
schema.  Do not instantiate this class directly; instead, use the
\method{addExpression} memthod of class \class{Schema}.

\begin{memberdesc}{name}
 \readonly The expressions's name.
\end{memberdesc}

\begin{memberdesc}{expression}
 \readonly An \class{Expression} instance; see \module{hep.expr}.
\end{memberdesc}

\begin{memberdesc}{formula}
 \readonly A string representation of the expression.
\end{memberdesc}



\subsection{Table objects}

A table object represents an in-memory table or a connection to an
externally-stored table.  In the former case, the table is erased when
the table object is deleted.  In the latter case, the connection is
closed but the table persists when the table object is deleted.

A table object satisfies the Python read-only sequence protocol.  The
sequence elements are the rows of the sequence.  Thus, the
\function{len} function returns the number of rows in the table, and
the \method{__getitem__} method extracts a row by its position in the
sequence. 

In addition, these methods are provided:

\begin{methoddesc}{append}{row\optional{, **kw_args}}
 (Read-write tables only.)  Appends a row to the end of the table.  The
 \var{row} must is a dictionary or other mapping object which maps
 column names to values.  Additional column values may be specified as
 keyword arguments.  If a column is specified both in \var{row} and in
 \var{**kw_args}, the value from latter is used.  A value must be
 specified for each column, and no other names may be specified.
 Returns the row index of the new row.
\end{methoddesc}

\begin{methoddesc}{__iter__}{}
 Returns an iterator over all rows in the table.
\end{methoddesc}

\begin{memberdesc}{rows}
 An interator over all rows in the table.
\end{memberdesc}

\begin{methoddesc}{select}{expr}
 Returns an iterator over rows in the table for which expression
 \var{expr} is true.  \var{expr} may be a string expression formula or
 an expression object.
\end{methoddesc}


\subsection{Row objects}

A row object may represent a row in a table, or a row that is compatible
with a table's schema (but not one of the rows in that table).

The row object satisfies the Python map protocol.  Each key is the name
of a column in the table's schema.  The corresponding map value is the
row's value for that column.  Row objects should be treated as
read-only; it is not possible to modify a table row by changing values
in a row object.

In addition, a row object provides these members and methods:

\begin{methoddesc}{asDict}{}
 Return a new Python dictionary of the column names and corresponding
 values in the row.
\end{methoddesc}

\begin{memberdesc}{index}
 \readonly The index of the row in the table.  The value of this
 attribute is undefined if the row was obtained from the table's
 \method{newRow} method and has not yet been passed to \method{append}.
\end{memberdesc}

\begin{memberdesc}{table}
 \readonly The table to which the row belongs.
\end{memberdesc}


\subsection{Standard implementation}

\pyhep's standard table implementation, in \module{hep.table}, uses a
simple, binary, on-disk representation of the table's data.  Table rows
are automatically loaded from disk as needed.

The \module{hep.table} module provides functions \function{create} and
\function{open} to create and open new tables, respectively.

\begin{funcdesc}{create}{filename, schema}
 Create a new table.  The table is stored in a file named by
 \var{filename}, which is created or overwritten.  The \var{schema}
 argument is a map from column names to \class{Column} instances,
 specifying the columns in the table.

 The return value is a table object, which is opened to the new, empty
 in write mode.
\end{funcdesc}

\begin{funcdesc}{open}{filename\optional{, mode="r"}}
 Open an existing table stored in the file named by \var{filename}.  The
 \var{mode} argument specifies the mode in which to open the table:
 \constant{"r"} to open the table in read-only mode, or \constant{"w"}
 to open the table in read-write mode.

 The return value is a table object.
\end{funcdesc}

